\documentclass[addpoints,spanish, 12pt,a4paper]{exam}
%\documentclass[answers, spanish, 12pt,a4paper]{exam}
\printanswers
\pointpoints{punto}{puntos}
\hpword{Puntos:}
\vpword{Puntos:}
\htword{Total}
\vtword{Total}
\hsword{Resultado:}
\hqword{Ejercicio:}
\vqword{Ejercicio:}

\usepackage[utf8]{inputenc}
\usepackage[spanish]{babel}
\usepackage{eurosym}
%\usepackage[spanish,es-lcroman, es-tabla, es-noshorthands]{babel}


\usepackage[margin=1in]{geometry}
\usepackage{amsmath,amssymb}
\usepackage{multicol}
\usepackage{yhmath}
\usepackage{pdflscape}

\pointsinrightmargin % Para poner las puntuaciones a la derecha. Se puede cambiar. Si se comenta, sale a la izquierda.
\extrawidth{-2.4cm} %Un poquito más de margen por si ponemos textos largos.
\marginpointname{ \emph{\points}}

\usepackage{graphicx}
\graphicspath{{../img/}} 

\newcommand{\class}{1º Bachillerato}
\newcommand{\examdate}{\today}
\newcommand{\examnum}{Extraordinario de septiembre}
\newcommand{\tipo}{A}


\newcommand{\timelimit}{80 minutos}

\renewcommand{\solutiontitle}{\noindent\textbf{Solución:}\enspace}


\pagestyle{head}
\firstpageheader{\includegraphics[width=0.2\columnwidth]{header_left}}{\textbf{Departamento de Matemáticas\linebreak \class}\linebreak \examnum}{\includegraphics[width=0.1\columnwidth]{header_right}}
\runningheader{\class}{\examnum}{Página \thepage\ of \numpages}
\runningheadrule


\begin{document}

\noindent
\begin{tabular*}{\textwidth}{l @{\extracolsep{\fill}} r @{\extracolsep{6pt}} }
\textbf{Fecha:}\makebox[1in]{\hrulefill} \textbf{Nombre:} \makebox[.6in]{\hrulefill} @@alumno \makebox[.6in]{\hrulefill} \\
 & \\
\textbf{Tiempo: \timelimit} & Tipo: \tipo 
\end{tabular*}
\rule[2ex]{\textwidth}{2pt}
Esta prueba tiene \numquestions\ ejercicios. La puntuación máxima es de \numpoints. 
La nota final de la prueba será la parte proporcional de la puntuación obtenida sobre la puntuación máxima. 

\begin{center}


\addpoints
 %\gradetable[h][questions]
	\pointtable[h][questions]
\end{center}

\noindent
\rule[2ex]{\textwidth}{2pt}

\begin{questions}

\question[2] Opera:
		\begin{multicols}{3}
		\begin{parts}  
				\part $$\frac{{\sqrt[5]{a}  \cdot \sqrt {a} }}{{{a^{\frac{1}{3}}}}}$$ \begin{solution}   $a^{\frac{11}{30}}$  \end{solution}
			\part $$\frac{x}{{1 + \frac{1}{{1 + \frac{1}{x}}}}}$$ \begin{solution}   $\frac{x^{2} + x}{2 x + 1}$  \end{solution}
		        \part $$\log ( {7 - \sqrt {22} } ) + \log ( {7 + \sqrt {22} } ) - 3\log 3 $$  \begin{solution}   $0$ \end{solution}
		        \end{parts}
		        \end{multicols}


\question[2] Calcula: 
		\begin{multicols}{2}
		\begin{parts}  
				\part $${\binom{x}{2}} = \frac{x!}{(x-1)!}$$ \begin{solution}   $3$  \end{solution}
			\part $$ \left\{\begin{matrix}\dfrac{{x - 1}}{3} - \dfrac{{x + 3}}{2} \leq x\\ \dfrac{{4x - 2}}{4} - \dfrac{{x - 1}}{3} \geq x\end{matrix}\right.$$ \begin{solution}   $\left[- \dfrac{11}{7}, - \dfrac{1}{2}\right]$ \end{solution}
		        
		        \end{parts}
		        \end{multicols}



\question[1] Resuelve por Gauss indicando el tipo de sistema:
 $$\left\{\begin{matrix}x + 2y - 3z = 9\\ 2x - y = 6\\ 4x + 3y - 6z = 24\\ \end{matrix}\right.$$
		 \begin{solution}   $\left[\begin{matrix}1 & 2 & -3 & 9\\0 & -5 & 6 & -12\\0 & 0 & 0 & 0\end{matrix}\right] \rightarrow  \\ \left \{ x : \frac{3 z}{5} + \frac{21}{5}, \quad y : \frac{6 z}{5} + \frac{12}{5}\right \} $   \end{solution}
		 
		 
		         \question[2] Dado el triángulo de vértices A=(-2, -1), B=(0, -3) y  C=(2, 1) que es acutángulo. Calcula:
        \begin{multicols}{1}
        \begin{parts}
        \part la longitud de sus lados   \begin{solution}   $2\sqrt{2}$, $2\sqrt{5}$, $2\sqrt{5}$  \end{solution} 
        \part sus ángulos \begin{solution}   $36'87$ y dos de $71'57$\   \end{solution} 
         \end{parts}
        \end{multicols}


        \question[2] Si $\sen\alpha=-\frac{5}{13}\land \alpha \in III$ (tercer cuadrante), calcula "sin usar la calculadora":
        \begin{multicols}{4}
        \begin{parts} 
        \part $\cos\alpha$  \begin{solution}   $- \frac{12}{13}$\   \end{solution}
                \part $\tan\alpha$  \begin{solution}   $\frac{5}{12}$\   \end{solution}  
                \part $\cos(\pi+\alpha)$  \begin{solution}   $\frac{12}{13}$\   \end{solution}
                \part $\sen(2\alpha)$  \begin{solution}   $\frac{120}{169}$\   \end{solution} 
        \end{parts}
        \end{multicols}


\question La temperatura media en los meses de invierno en varias ciudades y el gasto medio por habitante en
calefacción ha sido:\\
\\
\begin{tabular}{|c||c|c|c|c|c|c|}
\hline 
Temperatura (ºC) & 10 & 12 & 14 & 16  \\ 
\hline 
Gasto (\euro ) & 150 & 120 & 102 & 90  \\ 
\hline 
\end{tabular} \\
\begin{parts}
\part[1] Halla el coeficiente de correlación lineal 
\begin{solution}
\begin{tabular}{rrrrrr}
\hline
    &   x &     y &   xy &   x2 &    y2 \\
\hline
  0 &  10 & 150   & 1500 &  100 & 22500 \\
  1 &  12 & 120   & 1440 &  144 & 14400 \\
  2 &  14 & 102   & 1428 &  196 & 10404 \\
  3 &  16 &  90   & 1440 &  256 &  8100 \\
  \hline
  4 &  52 & 462   & 5808 &  696 & 55404 \\
  \hline
  5 &  13 & 115.5 & 1452 &  174 & 13851 \\
\hline
\end{tabular}
\\

covarianza	-49.5 \\
desvx	2.23606797749979 \\
desvy	22.599778759979046 \\
coefcorr -0.9795260923726159 \\
\end{solution}
\part[1] Estima, razonadamente, el gasto medio por habitante de una ciudad si la temperatura media hubiera sido de 11ºC. ¿Es fiable la estimación obtenida? 

\begin{solution}
$y = -9.9x + 244.2$ \\ Valor estimado para 11: 135.3 \euro 
\end{solution}
\end{parts}


\question Dos máquinas se usan para producir tornillos. La máquina A produce el 70\% de todos los tornillos.
El 2\% de todos los tornillos producidos por la máquina A son defectuosos, mientras que el 3\% de los
 producidos por la máquina B son defectuosos. Se selecciona un tornillo al azar de entre
todos los producidos. Calcular:

        \begin{parts} 
        
        \part[1] La probabilidad de que sea defectuoso \begin{solution}   $\frac{23}{1000}$   \end{solution} 
        \part[1] Si sabemos que el tornillo es defectuoso, calcula la probabilidad de que haya sido producido por la
máquina A. \begin{solution}   $\frac{14}{23}$   \end{solution} 
        \end{parts}


\question Un jugador de baloncesto tiene un porcentaje de acierto en tiros de 3 del 40 \%. Si tira seis
veces:
    

        \begin{parts} 
        
        \part[1] Calcula la probabilidad de que enceste 4  \begin{solution}   $P{\left(X = 4 \right)}=0.1382$   \end{solution} \part[1] Calcula la probabilidad de que enceste al menos 1  \begin{solution}   $P{\left(X \geq 1 \right)}=1-P{\left(X = 0 \right)}=0.9533$   \end{solution} \part[1] Calcula la probabilidad de que enceste más de 3 si ha fallado los dos primeros \begin{solution}   $P{\left(X' = 4 \right)}=0.0256$   \end{solution}
        \end{parts}





\question Dadas las funciones $$f(x)= x^2+5$$ $$g(x)= \frac{{x - 1}}{{x + 3}}$$ 
%y $h(x)= \sqrt{x}$
Calcula:

	\begin{parts}  \part[1] $f \circ g$. Es decir, $g$ compuesta con $f$   \begin{solution} $f{\left (g{\left (x \right( )} \right )}=\frac{\left(x - 1\right)^{2}}{\left(x + 3\right)^{2}} + 5$
	\end{solution}
	        \part[1] $g^{-1}(x)$. Es decir, la inversa de $g$  \begin{solution}   $g^{-1}(x)=- \frac{3 x + 1}{x - 1}$ \end{solution}
	        \end{parts}


        
\question[2] Dada la función:$f(x)=\dfrac{- x^{2} - x + 3}{x^{2} + x - 2}$. Calcular:

        \begin{parts} \part Dominio de $f(x)$  \begin{solution}   $Dom(f)=\left(-\infty, -2\right) \cup \left(-2, 1\right) \cup \left(1, \infty\right)$\\  \end{solution} \part Asíntotas verticales, horizontales y oblicuas, en caso que existan  \begin{solution}   Asíntotas:\\A.V. $x=-2$\\, A.V. $x=1$\\A.H. $y=-1$\\A.H. $y=-1$\\A.O. $y=-1$ \\A.O. $y=-1$ \\   \end{solution}
        \end{parts}

        

        

\addpoints




\end{questions}


\end{document}

