
        \documentclass[addpoints,spanish, 12pt,a4paper]{exam}
        %\documentclass[answers, spanish, 12pt,a4paper]{exam}
        \printanswers
        \pointpoints{punto}{puntos}
        \hpword{Puntos:}
        \vpword{Puntos:}
        \htword{Total}
        \vtword{Total}
        \hsword{Resultado:}
        \hqword{Ejercicio:}
        \vqword{Ejercicio:}

        \usepackage[utf8]{inputenc}
        \usepackage[spanish]{babel}
        \usepackage{eurosym}
        %\usepackage[spanish,es-lcroman, es-tabla, es-noshorthands]{babel}


        \usepackage[margin=1in]{geometry}
        \usepackage{amsmath,amssymb}
        \usepackage{multicol, xparse}

        \usepackage{yhmath}

        \usepackage{verbatim}
        %\usepackage{pstricks}


        \usepackage{graphicx}
        \graphicspath{{../img/}}




        \let\multicolmulticols\multicols
        \let\endmulticolmulticols\endmulticols
        \RenewDocumentEnvironment{multicols}{mO{}}
         {%
          \ifnum#1=1
            #2%
          \else % More than 1 column
            \multicolmulticols{#1}[#2]
          \fi
         }
         {%
          \ifnum#1=1
          \else % More than 1 column
            \endmulticolmulticols
          \fi
         }
        \renewcommand{\solutiontitle}{\noindent\textbf{Sol:}\enspace}

        \newcommand{\samedir}{\mathbin{\!/\mkern-5mu/\!}}

        \newcommand{\class}{1º Bachillerato}
        \newcommand{\examdate}{\today}

        %\newcommand{\tipo}{A}


        \newcommand{\timelimit}{50 minutos}

        \renewcommand{\solutiontitle}{\noindent\textbf{Solución:}\enspace}


        \pagestyle{head}
        \firstpageheader{\includegraphics[width=0.2\columnwidth]{header_left}}{\textbf{Departamento de Matemáticas\linebreak \class}\linebreak \examnum}{\includegraphics[width=0.1\columnwidth]{header_right}}
        \runningheader{\class}{\examnum}{Página \thepage\ of \numpages}
        \runningheadrule
        
        \pointsinrightmargin % Para poner las puntuaciones a la derecha. Se puede cambiar. Si se comenta, sale a la izquierda.
        \extrawidth{-2.4cm} %Un poquito más de margen por si ponemos textos largos.
        \marginpointname{ \emph{\points}}

        \newcommand{\tipo}{B}\newcommand{\examnum}{Autoevaluación}
        \begin{document}
        \noindent
        \begin{tabular*}{\textwidth}{l @{\extracolsep{\fill}} r @{\extracolsep{6pt}} }
        \textbf{Nombre:} \makebox[3.5in]{\hrulefill} & \textbf{Fecha:}\makebox[1in]{\hrulefill} \\
         & \\
        \textbf{Tiempo: \timelimit} & Tipo: \tipo 
        \end{tabular*}
        \rule[2ex]{\textwidth}{2pt}
        Esta prueba tiene \numquestions\ ejercicios. La puntuación máxima es de \numpoints. 
        La nota final de la prueba será la parte proporcional de la puntuación obtenida sobre la puntuación máxima. 

        \begin{center}


        \addpoints
             %\gradetable[h][questions]
            \pointtable[h][questions]
        \end{center}

        \noindent
        \rule[2ex]{\textwidth}{2pt}

        \begin{questions}
        \question Resolver las siguientes inecuaciones:
        \begin{multicols}{1} 
        \begin{parts} \part[1]  $ \left|{x - 4}\right| - 2< 0 $  \begin{solution}  $ 2 < x \wedge x < 6 $  \end{solution}
        \end{parts}
        \end{multicols}
        \question Calcula :
        \begin{multicols}{1} 
        \begin{parts} \part[1]  $ \frac{{{5^{ - 3}} \cdot {5^{ - 1}} \cdot {5^2}}}{{{5^0} + {5^6}}} $  \begin{solution}  $ \frac{1}{390650} $  \end{solution} \part[1]  $ {( {\frac{7}{4}} )^5} \cdot \frac{{{2^6}}}{{{7^2}}} $  \begin{solution}  $ \frac{343}{16} $  \end{solution} \part[1]  $ \frac{{{3^{ - 3}} \cdot {3^6} \cdot {2^3}}}{{{{( {3 \cdot 2} )}^5}}} $  \begin{solution}  $ \frac{1}{36} $  \end{solution}
        \end{parts}
        \end{multicols}
        \question Calcula:
        \begin{multicols}{1} 
        \begin{parts} \part[1]  $ 2\sqrt {3125}  + 3\sqrt {20}  - 12\sqrt {45} $  \begin{solution}  $ 20 \sqrt{5} $  \end{solution}
        \end{parts}
        \end{multicols}
        \question Realiza los desarrollos de los siguientes binomios:
        \begin{multicols}{1} 
        \begin{parts} \part[1]  $ ( {1 + 3\sqrt{ 2} })^3 $  \begin{solution}  $ 55 + 63 \sqrt{2} $  \end{solution} \part[1]  $ ( {5\sqrt {2} - 2\sqrt {3} } )^3 $  \begin{solution}  $ - 324 \sqrt{3} + 430 \sqrt{2} $  \end{solution} \part[1]  $ ( {\frac{2}{\sqrt {2} } + \sqrt {2} } )^3 $  \begin{solution}  $ \frac{12 \sqrt{2}}{2} + 10 \sqrt{2} $  \end{solution} \part[1]  $ (3 + x)^4 $  \begin{solution}  $ x^{4} + 12 x^{3} + 54 x^{2} + 108 x + 81 $  \end{solution}
        \end{parts}
        \end{multicols}
        \question Descomponer en factores
        \begin{multicols}{1} 
        \begin{parts} \part[1]  $ 2 x^{3} + 2 x^{2} - 12 x $  \begin{solution}  $ 2 x \left(x - 2\right) \left(x + 3\right) $  \end{solution}
        \end{parts}
        \end{multicols}
        
    \end{questions}
    \end{document}
    