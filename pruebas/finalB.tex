
        \documentclass[addpoints,spanish, 12pt,a4paper]{exam}
        %\documentclass[answers, spanish, 12pt,a4paper]{exam}
        
        \pointpoints{punto}{puntos}
        \hpword{Puntos:}
        \vpword{Puntos:}
        \htword{Total}
        \vtword{Total}
        \hsword{Resultado:}
        \hqword{Ejercicio:}
        \vqword{Ejercicio:}

        \usepackage[utf8]{inputenc}
        \usepackage[spanish]{babel}
        \usepackage{eurosym}
        %\usepackage[spanish,es-lcroman, es-tabla, es-noshorthands]{babel}


        \usepackage[margin=1in]{geometry}
        \usepackage{amsmath,amssymb}
        \usepackage{multicol, xparse}

        \usepackage{yhmath}

        \usepackage{verbatim}
        %\usepackage{pstricks}


        \usepackage{graphicx}
        \graphicspath{{../img/}}




        \let\multicolmulticols\multicols
        \let\endmulticolmulticols\endmulticols
        \RenewDocumentEnvironment{multicols}{mO{}}
         {%
          \ifnum#1=1
            #2%
          \else % More than 1 column
            \multicolmulticols{#1}[#2]
          \fi
         }
         {%
          \ifnum#1=1
          \else % More than 1 column
            \endmulticolmulticols
          \fi
         }
        \renewcommand{\solutiontitle}{\noindent\textbf{Sol:}\enspace}

        \newcommand{\samedir}{\mathbin{\!/\mkern-5mu/\!}}

        \newcommand{\class}{1º Bachillerato}
        \newcommand{\examdate}{\today}

        %\newcommand{\tipo}{A}


        \newcommand{\timelimit}{50 minutos}

        \renewcommand{\solutiontitle}{\noindent\textbf{Solución:}\enspace}


        \pagestyle{head}
        \firstpageheader{\includegraphics[width=0.2\columnwidth]{header_left}}{\textbf{Departamento de Matemáticas\linebreak \class}\linebreak \examnum}{\includegraphics[width=0.1\columnwidth]{header_right}}
        \runningheader{\class}{\examnum}{Página \thepage\ of \numpages}
        \runningheadrule
        
        \pointsinrightmargin % Para poner las puntuaciones a la derecha. Se puede cambiar. Si se comenta, sale a la izquierda.
        \extrawidth{-2.4cm} %Un poquito más de margen por si ponemos textos largos.
        \marginpointname{ \emph{\points}}

        %\printanswers
            \newcommand{\tipo}{B}\newcommand{\examnum}{Final 1ª evaluación}
        \begin{document}
        \noindent
        \begin{tabular*}{\textwidth}{l @{\extracolsep{\fill}} r @{\extracolsep{6pt}} }
        \textbf{Nombre:} \makebox[3.5in]{\hrulefill} & \textbf{Fecha:}\makebox[1in]{\hrulefill} \\
         & \\
        \textbf{Tiempo: \timelimit} & Tipo: \tipo 
        \end{tabular*}
        \rule[2ex]{\textwidth}{2pt}
        Esta prueba tiene \numquestions\ ejercicios. La puntuación máxima es de \numpoints. 
        La nota final de la prueba será la parte proporcional de la puntuación obtenida sobre la puntuación máxima. 

        \begin{center}


        \addpoints
             %\gradetable[h][questions]
            \pointtable[h][questions]
        \end{center}

        \noindent
        \rule[2ex]{\textwidth}{2pt}

        \begin{questions}
        \question Justifica si los siguientes pares de vectores forman base de $\mathbb{R}^2$:
        \begin{multicols}{1} 
        \begin{parts} \part[1]  $ \overrightarrow{u}=(1, 2) \, \ \overrightarrow{v}=(3, 4) $  \begin{solution}  $ True $  \end{solution} \part[1]  $ \overrightarrow{u}=(4, 12) \, \ \overrightarrow{v}=(2, 6) $  \begin{solution}  $ False $  \end{solution}
        \end{parts}
        \end{multicols}
        \question Determinar el ángulo formado por las rectas:
        \begin{multicols}{1} 
        \begin{parts} \part[1]  $ r\equiv2x-y-2=0\  y \ s\equiv3x+2y-4=0 $  \begin{solution}  $ 119.74488129694222 $  \end{solution}
        \end{parts}
        \end{multicols}
        \question Calcula el vértice C de un triángulo isósceles ABC, sabiendo que:
        \begin{multicols}{1} 
        \begin{parts} \part[1]  $  A(2, -3)\  , \ B=(5, 2) \ y \  C \in r\equiv -x+3y-16=0 $  \begin{solution}  $ \left [ \left \{ x : -4, \quad y : 4\right \}\right ] $  \end{solution}
        \end{parts}
        \end{multicols}
        \question Dado el triángulo ABC, hallar: \\ 
\begin{itemize} \item la mediana correspondiente al vértice A \item la mediatriz correspondiente al lado AB \item y el área \end{itemize} Siendo:
        \begin{multicols}{1} 
        \begin{parts} \part[1]  $ A=(2, 1), \ B=(4, 3),\  y \ C=(6, -1) $  \begin{solution}  $ \left [ y - 1 = 0, \quad - 2 x - 2 y + 10 = 0, \quad \left [ 3 \sqrt{2}, \quad 6\right ]\right ] $  \end{solution}
        \end{parts}
        \end{multicols}
        \question Resolver las siguientes ecuaciones:
        \begin{multicols}{1} 
        \begin{parts} \part[1]  $ \cos{2x}-3\cos{x}+1=0 $  \begin{solution}  $ \left [ -90, \quad 90\right ] $  \end{solution} \part[1]  $ 2\cos^2{x}-\sqrt{3}\cos{x}=0 $  \begin{solution}  $ \left [ 30, \quad 90, \quad 270, \quad 330\right ] $  \end{solution}
        \end{parts}
        \end{multicols}
        \question Calcula:
        \begin{multicols}{1} 
        \begin{parts} \part[1]  $ \frac{(1+2i)i^7}{(3-2i)-(2+i)} $  \begin{solution}  $ \frac{1}{2} + \frac{i}{2} $  \end{solution}
        \end{parts}
        \end{multicols}
        \question Escribe los siguientes números complejos en forma polar con el argumento en radianes:
        \begin{multicols}{1} 
        \begin{parts} \part[1]  $ 2i $  \begin{solution}  $ 2_{\frac{\pi}{2}} $  \end{solution} \part[1]  $ 2 - 2\sqrt {3}i $  \begin{solution}  $ 4_{- \frac{\pi}{3}} $  \end{solution} \part[1]  $ -4 $  \begin{solution}  $ 4_{\pi} $  \end{solution}
        \end{parts}
        \end{multicols}
        
    \end{questions}
    \end{document}
    