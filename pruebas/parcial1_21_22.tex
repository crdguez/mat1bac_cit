
        \documentclass[addpoints,spanish, 12pt,a4paper]{exam}
        %\documentclass[answers, spanish, 12pt,a4paper]{exam}
        % \printanswers
        \pointpoints{punto}{puntos}
        \hpword{Puntos:}
        \vpword{Puntos:}
        \htword{Total}
        \vtword{Total}
        \hsword{Resultado:}
        \hqword{Ejercicio:}
        \vqword{Ejercicio:}

        \usepackage[utf8]{inputenc}
        \usepackage[spanish]{babel}
        \usepackage{eurosym}
        %\usepackage[spanish,es-lcroman, es-tabla, es-noshorthands]{babel}


        \usepackage[margin=1in]{geometry}
        \usepackage{amsmath,amssymb}
        \usepackage{multicol, xparse}

        \usepackage{yhmath}

        \usepackage{verbatim}
        %\usepackage{pstricks}


        \usepackage{graphicx}
        \graphicspath{{../img/}}




        \let\multicolmulticols\multicols
        \let\endmulticolmulticols\endmulticols
        \RenewDocumentEnvironment{multicols}{mO{}}
         {%
          \ifnum#1=1
            #2%
          \else % More than 1 column
            \multicolmulticols{#1}[#2]
          \fi
         }
         {%
          \ifnum#1=1
          \else % More than 1 column
            \endmulticolmulticols
          \fi
         }
        \renewcommand{\solutiontitle}{\noindent\textbf{Sol:}\enspace}

        \newcommand{\samedir}{\mathbin{\!/\mkern-5mu/\!}}

        \newcommand{\class}{1º Bachillerato}
        \newcommand{\examdate}{\today}

        %\newcommand{\tipo}{A}


        \newcommand{\timelimit}{45 minutos}

        \renewcommand{\solutiontitle}{\noindent\textbf{Solución:}\enspace}


        \pagestyle{head}
        \firstpageheader{\includegraphics[width=0.2\columnwidth]{header_left}}{\textbf{Departamento de Matemáticas\linebreak \class}\linebreak \examnum}{\includegraphics[width=0.1\columnwidth]{header_right}}
        \runningheader{\class}{\examnum}{Página \thepage\ of \numpages}
        \runningheadrule
        
        \pointsinrightmargin % Para poner las puntuaciones a la derecha. Se puede cambiar. Si se comenta, sale a la izquierda.
        \extrawidth{-2.4cm} %Un poquito más de margen por si ponemos textos largos.
        \marginpointname{ \emph{\points}}

        %\printanswers
            \newcommand{\tipo}{B}\newcommand{\examnum}{Parcial 1ª evaluación}
        \begin{document}
        \noindent
        \begin{tabular*}{\textwidth}{l @{\extracolsep{\fill}} r @{\extracolsep{6pt}} }
        \textbf{Nombre:} \makebox[3.5in]{\hrulefill} & \textbf{Fecha:}\makebox[1in]{\hrulefill} \\
         & \\
        \textbf{Tiempo: \timelimit} & Tipo: \tipo 
        \end{tabular*}
        \rule[2ex]{\textwidth}{2pt}
        Esta prueba tiene \numquestions\ ejercicios. La puntuación máxima es de \numpoints. 
        La nota final de la prueba será la parte proporcional de la puntuación obtenida sobre la puntuación máxima. 

        \begin{center}


        \addpoints
             %\gradetable[h][questions]
            \pointtable[h][questions]
        \end{center}

        \noindent
        \rule[2ex]{\textwidth}{2pt}

        \begin{questions}
        \question Dados los siguientes conjuntos:
        $$A=\left\{ x \in \mathbb{R}| -1 \leq x < 5 \right\}$$
        $$ B=\left(-\infty, -2\right) \cup \left(2, \infty\right) $$
        $$C=\left\{ x \in \mathbb{R}| \left|{x + 4}\right|\geq7 \right\}$$ 
        \begin{solution}
        $\rightarrow Solución:  C=\left(-\infty, -11\right] \cup \left[3, \infty\right) \ \ A \cup  B = \left(-\infty, -2\right) \cup \left[-1, \infty\right)  \\  A \cap B= \left(2, 5\right)   \\  A \cap B  \cap C= \left[3, 5\right)$
        \end{solution}

        \begin{multicols}{1} 
        \begin{parts}
        \part[2] Representa en la recta real y en forma de intervalos el conjunto $C$
        \part[1] Calcula $A \cup  B$ , $A \cap B$ y $A \cap B \cap C$
        \part[1] Encuentra, si existen, el supremo, el ínfimo, el máximo y el mínimo de cada uno de los conjuntos $A$, $B$ y $C$
       
        \end{parts}
        \end{multicols}
        
        
        
        \question[1] Usando la definición y las propiedades de los números combinatorios, resolver la ecuación:
$$ {\binom{23}{x}} = \binom{23}{3x-1} $$  \begin{solution}  $ \left\{6\right\} $  \end{solution}
        \question[2] Calcula, sin hacer todo el desarrollo, el coeficiente del término asociado a:
  $$ P(x)=\left(3 x^{2} + \frac{1}{x}\right)^{7} \  \ y \ parte \ literal \ x^{5} $$  \begin{solution}  $ 2835 $  \end{solution}

        \question[2] Efectúa:
        \begin{multicols}{2} 
        \begin{parts} \part  $$ \frac{{2 - \sqrt {3} }}{{1 - \sqrt {3} }}-\frac{1}{2\sqrt {3}}+ \frac{3}{2-\sqrt{3} } $$  \begin{solution}  $ \frac{14 \sqrt{3} + 39}{6}\to\frac{7 \sqrt{3}}{3} + \frac{13}{2} $  \end{solution} \part  $$ \frac{\sqrt{7}-\sqrt{5}}{\sqrt{7}+\sqrt{5}}-\frac{\sqrt{7}+\sqrt{5}}{\sqrt{7}-\sqrt{5}} $$  \begin{solution}  $ \frac{-12 - 2 \sqrt{35} + \left(- \sqrt{7} + \sqrt{5}\right)^{2}}{2}\to- 2 \sqrt{35} $  \end{solution}
        \end{parts}
        \end{multicols}
        \question[2] Calcula el valor de $k$ para que el resto de dividir
   $P(x)=x^{27}-kx+3k-4$  entre  $x+1$  sea $11$    \begin{solution}  $ 4 $  \end{solution}
        % \question Halla el m.c.d. y el m.c.m. de los polinomios: 
        % \begin{multicols}{1} 
        % \begin{parts} \part[3]  $  A(x)= x^{5} - 6 x^{3} + 2 x^{2} + 9 x - 6\  y \\  B(x)= x^{5} + 3 x^{4} - 3 x^{3} - 13 x^{2} + 12 \\  $  \begin{solution}  $ Descomposici \acute on:(\left(x - 1\right)^{2} \left(x + 2\right) \left(x^{2} - 3\right) \ y  \ \left(x - 1\right) \left(x + 2\right)^{2} \left(x^{2} - 3\right))\\x^{4} + x^{3} - 5 x^{2} - 3 x + 6= \left(x - 1\right) \left(x + 2\right) \left(x^{2} - 3\right) \ MCD\ y \\ x^{6} + 2 x^{5} - 6 x^{4} - 10 x^{3} + 13 x^{2} + 12 x - 12= \left(x - 1\right)^{2} \left(x + 2\right)^{2} \left(x^{2} - 3\right) \ MCM $  \end{solution}
        % \end{parts}
        % \end{multicols}
        
    \end{questions}
    \end{document}
    