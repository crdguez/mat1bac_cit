
        \documentclass[addpoints,spanish, 12pt,a4paper]{exam}
        %\documentclass[answers, spanish, 12pt,a4paper]{exam}
        
        \printanswers
        
        \pointpoints{punto}{puntos}
        \hpword{Puntos:}
        \vpword{Puntos:}
        \htword{Total}
        \vtword{Total}
        \hsword{Resultado:}
        \hqword{Ejercicio:}
        \vqword{Ejercicio:}

        \usepackage[utf8]{inputenc}
        \usepackage[spanish]{babel}
        \usepackage{eurosym}
        %\usepackage[spanish,es-lcroman, es-tabla, es-noshorthands]{babel}


        \usepackage[margin=1in]{geometry}
        \usepackage{amsmath,amssymb}
        \usepackage{multicol, xparse}

        \usepackage{yhmath}

        \usepackage{verbatim}
        %\usepackage{pstricks}


        \usepackage{graphicx}
        \graphicspath{{../img/}}




        \let\multicolmulticols\multicols
        \let\endmulticolmulticols\endmulticols
        \RenewDocumentEnvironment{multicols}{mO{}}
         {%
          \ifnum#1=1
            #2%
          \else % More than 1 column
            \multicolmulticols{#1}[#2]
          \fi
         }
         {%
          \ifnum#1=1
          \else % More than 1 column
            \endmulticolmulticols
          \fi
         }
        \renewcommand{\solutiontitle}{\noindent\textbf{Sol:}\enspace}

        \newcommand{\samedir}{\mathbin{\!/\mkern-5mu/\!}}

        \newcommand{\class}{1º Bachillerato}
        \newcommand{\examdate}{\today}

        %\newcommand{\tipo}{A}


        \newcommand{\timelimit}{50 minutos}

        \renewcommand{\solutiontitle}{\noindent\textbf{Solución:}\enspace}


        \pagestyle{head}
        \firstpageheader{\includegraphics[width=0.2\columnwidth]{header_left}}{\textbf{Departamento de Matemáticas\linebreak \class}\linebreak \examnum}{\includegraphics[width=0.1\columnwidth]{header_right}}
        \runningheader{\class}{\examnum}{Página \thepage\ of \numpages}
        \runningheadrule
        
        \pointsinrightmargin % Para poner las puntuaciones a la derecha. Se puede cambiar. Si se comenta, sale a la izquierda.
        \extrawidth{-2.4cm} %Un poquito más de margen por si ponemos textos largos.
        \marginpointname{ \emph{\points}}

        \newcommand{\tipo}{A}\newcommand{\examnum}{Recuperación 2ª evaluación}
        \begin{document}
        \noindent
        \begin{tabular*}{\textwidth}{l @{\extracolsep{\fill}} r @{\extracolsep{6pt}} }
        \textbf{Nombre:} \makebox[3.5in]{\hrulefill} & \textbf{Fecha:}\makebox[1in]{\hrulefill} \\
         & \\
        \textbf{Tiempo: \timelimit} & Tipo: \tipo 
        \end{tabular*}
        \rule[2ex]{\textwidth}{2pt}
        Esta prueba tiene \numquestions\ ejercicios. La puntuación máxima es de \numpoints. 
        La nota final de la prueba será la parte proporcional de la puntuación obtenida sobre la puntuación máxima. 

        \begin{center}


        \addpoints
             %\gradetable[h][questions]
            \pointtable[h][questions]
        \end{center}

        \noindent
        \rule[2ex]{\textwidth}{2pt}

        \begin{questions}
        % \question[1] Justifica si los siguientes pares de vectores forman base de $\mathbb{R}^2$:
        % \begin{multicols}{2} 
        % \begin{parts} \part  $ \overrightarrow{u}=(1, 2) \, \ \overrightarrow{v}=(3, 4) $  \begin{solution}  $ True $  \end{solution} \part  $ \overrightarrow{u}=(4, 12) \, \ \overrightarrow{v}=(2, 6) $  \begin{solution}  $ False $  \end{solution}
        % \end{parts}
        % \end
        
        % \question Dados los vectores $\overrightarrow{a}=(-2, -3)$, $\overrightarrow{b}=(9, -6)$ y $\overrightarrow{c}=(-7, -17)$ expresados en la base canónica:
        % \begin{parts}
        % \part[1] Demuestra que $B=\left\{\overrightarrow{a},\overrightarrow{b}\right\}$ es una base del plano. ¿Es ortogonal?¿Y ortonormal?
        % \begin{solution}
        % % Point(-2,-3).dot(Point(9,-6))
        % % Point(-2,-3).distance(Point(0,0)
        % $\overrightarrow{a}\cdot\overrightarrow{b}=0 \to $ ortogonal y $|\overrightarrow{a}|=\sqrt{13} \to $ no ortonormal
        % \end{solution}
        % \part[1] Calcula las coordenadas de $\overrightarrow{c}$ respecto de la base $B=\left\{\overrightarrow{a},\overrightarrow{b}\right\}$\begin{solution}
        % $(5,\frac{1}{3})$
        % \end{solution}
        % \end{parts}
        
        \question ¿Son equipolentes los vectores  $\overrightarrow {AB} $  y  $\overrightarrow {CD} $ siendo A, B, C y D los puntos de coordenadas:?
        % \begin{multicols}{2} 
        \begin{parts} \part[1] A(3, 4), B(7, 2), C(-1, 0) y D(3, -2)  \begin{solution}  $ Point2D(4, -2), Point2D(4, -2): True $  \end{solution}
        \part[1] Demuestra si los vectores forman base de $\mathbb{R}^2$  \begin{solution}  No  \end{solution}
        \end{parts}
        % \end{multicols}

        
        % \question[2] Desde un punto del suelo se ve la copa de un pino bajo un ángulo de 42°. Si nos alejamos 2,5 m hacia otro punto del suelo, alineado con el anterior y con el pie del pino, vemos la copa bajo un ángulo de 24°. Calcula la altura del pino. \begin{solution}
        % $2,2 m$
        % \end{solution}

        % \question[1] Calcula el área del paralelogramo cuyos lados miden 10 y 15 cm, respectivamente, si uno de sus ángulos mide $35^\circ$\begin{solution}$\frac{1}{2}\cdot 10 \cdot 15 \cdot \sen 35 = 86,04 \  cm^2$ \end{solution}
        
        % \question[2] Calcula el área de un triángulo sabiendo que $a=16$ cm, $b=25$ cm y $c=15$ cm.\begin{solution} $\sen(A)=0,6105$ y $A=114,17cm^2$ \end{solution}
        
        \question Calcula el área de un triángulo sabiendo que:
        \begin{multicols}{1} 
        \begin{parts} \part[2] El lado a=8, y los ángulos B=30º, y C=105º  \begin{solution}  $ 2 \sqrt{2} \left(2 \sqrt{2} + 2 \sqrt{6}\right) \to 21.856406460551018 $  \end{solution}
        \end{parts}
        \end{multicols}
        
        
        % \question Calcula los lados de un triángulo sabiendo que:
        % \begin{multicols}{1} 
        % \begin{parts} \part[2] Su área mide=18 cm², y los ángulos A=30º, y B=45º  \begin{solution}  $ 6, 6, 6 \sqrt{2} $  \end{solution}
        % \end{parts}
        % \end{multicols}
        
          
        % \question Dado el triángulo de vértices A ( 2 , 1 ) , B ( 6 , 5 ) y C ( 5 , 1 ) , calcula:
        % El área del triángulo. El ángulo A \begin{solution}
        % $14 ud^2$
        % \end{solution}
        

%         \question Determinar el ángulo formado por las rectas:
%         \begin{multicols}{1} 
%         \begin{parts} \part[1]  $ r\equiv2x-y-2=0\  y \ s\equiv3x+2y-4=0 $  \begin{solution}  $ 119.74488129694222 $  \end{solution}
%         \end{parts}
%         \end{multicols}
%         \question Calcula el vértice C de un triángulo isósceles ABC, sabiendo que:
%         \begin{multicols}{1} 
%         \begin{parts} \part[2]  $  A(2, -3)\  , \ B=(5, 2) \ y \  C \in r\equiv -x+3y-16=0 $  \begin{solution}  $ \left [ \left \{ x : -4, \quad y : 4\right \}\right ] $  \end{solution}
%         \end{parts}
%         \end{multicols}
%         \question[3] Dado el triángulo ABC, siendo $ A=(2, 1), \ B=(4, 3),\  y \ C=(6, -1) $, hallar: 

%         \begin{multicols}{1} 
%         \begin{parts} \part  la mediana correspondiente al vértice A  
%         \part la mediatriz correspondiente al lado AB
%         \part el área del triángulo
%         \end{parts}
% \begin{solution}  $ \left [ y - 1 = 0, \quad - 2 x - 2 y + 10 = 0, \quad \left [ 3 \sqrt{2}, \quad 6\right ]\right ] $  \end{solution}
%         \end{multicols}
        \question Resolver las siguientes ecuaciones:
        \begin{multicols}{1} 
        \begin{parts} 
        % \part[2]  $ \cos{2x}-3\cos{x}+1=0 $  \begin{solution}  $ \left [ -90, \quad 90\right ] $  \end{solution} 
        % \part[2] $\sin{\left(x \right)} + \cos{\left(x \right)} = - \sqrt{2}$ \begin{solution}     $\left[ - \frac{3 \pi}{4}\right]$ \end{solution}
        % \part[2]  $ 2\cos^2{x}-\sqrt{3}\cos{x}=0 $  \begin{solution}  $ \left [ 30, \quad 90, \quad 270, \quad 330\right ] $  \end{solution}
        
        % \part[2]        $\sen x + \cos 2x = 1 $ \begin{solution}
        % map(deg,solve(sin(x)+cos(2*x)-1))
        % $[0, 30, 150, 180]$\end{solution}
        \part[2]  $ \sen{x}\cos{x}=2\sen{x} $  \begin{solution}  $ \left [ 0\right ] $  \end{solution}
        
        \part[2]  $ \tan{x}-\sen{x}=0 $  \begin{solution}  $ \left [ 0, \quad -180, \quad 180, \quad 360\right ] $  \end{solution}
        \end{parts}
        \end{multicols}
        
        % \question Resolver las siguientes ecuaciones:
        % \begin{multicols}{1} 
        % \begin{parts} \part[1]  $ 2\sin{x}+\csc{x}=2\sqrt{2} $  \begin{solution}  $ \left [ 45, \quad 135\right ] $  \end{solution} \part[1]  $ \sin{x}=\cos^2{x}+1 $  \begin{solution}  $ \left [ 90\right ] $  \end{solution} \part[1]  $ \sin{x}\cos{x}=0 $  \begin{solution}  $ \left [ 0, \quad 90, \quad 180, \quad 270\right ] $  \end{solution} \part[1]  $ \tan{x}-\sin{x}=0 $  \begin{solution}  $ \left [ 0, \quad -180, \quad 180, \quad 360\right ] $  \end{solution} \part[1]  $ \sin{x}\cos{x}=2\sin{x} $  \begin{solution}  $ \left [ 0\right ] $  \end{solution} \part[1]  $ 2\cos{x}-3\tan{x}=0 $  \begin{solution}  $ \left [ 150, \quad 30, \quad - \frac{180 i \log{\left (- i \left(- \sqrt{3} + 2\right) \right )}}{\pi}, \quad - \frac{180 i \log{\left (- i \left(\sqrt{3} + 2\right) \right )}}{\pi}\right ] $  \end{solution} \part[1]  $ \sin{2x}=2\cos{x} $  \begin{solution}  $ \left [ -90, \quad 90\right ] $  \end{solution} \part[1]  $ 4\tan{x}=\frac{\sqrt{3}}{\cos^2{x}} $  \begin{solution}  $ \left [ -120, \quad -150, \quad 60, \quad 30\right ] $  \end{solution} \part[1]  $ \sin{x}+\cos{x}=\sqrt{2} $  \begin{solution}  $ \left [ 45\right ] $  \end{solution} \part[1]  $ \sin{2x}\cos{x}=6\sin^3{x} $  \begin{solution}  $ \left [ 0, \quad 180, \quad -150, \quad 150, \quad -30, \quad 30\right ] $  \end{solution} \part[1]  $ 4\sin{\frac{x}{2}}\cos{x}=3 $  \begin{solution}  $ \left [ \right ] $  \end{solution} \part[1]  $ \tan{x}\tan{2x}=1 $  \begin{solution}  $ \left [ -150, \quad 150, \quad -30, \quad 30\right ] $  \end{solution} \part[1]  $ 4\cos{2x}+3\cos{x}=1 $  \begin{solution}  $ \left [ 180, \quad - \frac{180 i \log{\left (\frac{5}{8} - \frac{\sqrt{39} i}{8} \right )}}{\pi}, \quad - \frac{180 i \log{\left (\frac{5}{8} + \frac{\sqrt{39} i}{8} \right )}}{\pi}\right ] $  \end{solution} \part[1]  $ \tan{x}+3\cot{x}=4 $  \begin{solution}  $ \left [ 45, \quad \frac{180 \operatorname{atan}{\left (3 \right )}}{\pi}\right ] $  \end{solution} \part[1]  $ 4\sin{(x-30)}\cos{(x-30)}=\sqrt{3} $  \begin{solution}  $ \left [ \frac{180 \left(- \frac{2 \pi}{3} + 30\right)}{\pi}, \quad \frac{180 \left(\frac{\pi}{6} + 30\right)}{\pi}, \quad \frac{180 \left(\frac{\pi}{3} + 30\right)}{\pi}, \quad \frac{180 \left(- 2 \operatorname{atan}{\left (\sqrt{3} + 2 \right )} + 30\right)}{\pi}\right ] $  \end{solution}
        % \end{parts}
        % \end{multicols}
        
        
        \question Calcula:
        \begin{multicols}{1} 
        \begin{parts} 
        % \part[2]  $ \dfrac{(1+2i)i^7}{(3-2i)-(2+i)} $  \begin{solution}  $ \frac{1}{2} + 
        % \frac{i}{2} $  \end{solution}
        \part[2]  $ ({\sqrt {2}  - i})\dfrac{{\sqrt{ 2} +i}}{{1 - 2i}} $  \begin{solution}  $ \frac{3}{5} + \frac{6 i}{5} $  \end{solution}
        \end{parts}
        \end{multicols}
        
        
        % \question Calcula
        % \begin{multicols}{2} 
        % \begin{parts} \part[1]  $ (5-i)(3+2i) $  \begin{solution}  $ 17 + 7 i $  \end{solution} \part[1]  $ (2+\frac{1}{3}i)(-5-i) $  \begin{solution}  $ - \frac{29}{3} - \frac{11 i}{3} $  \end{solution} \part[1]  $ (2-i)(2+i) $  \begin{solution}  $ 5 $  \end{solution} \part[1]  $ (3-\frac{1}{4}i)(2-i)(3+2i) $  \begin{solution}  $ \frac{97}{4} + i $  \end{solution} \part[1]  $ \frac{2-i}{1+3i} $  \begin{solution}  $ - \frac{1}{10} - \frac{7 i}{10} $  \end{solution} \part[1]  $ \frac{{\sqrt{2} - 3i}}{{2 + i}} $  \begin{solution}  $ - \frac{3}{5} + \frac{2 \sqrt{2}}{5} - \frac{6 i}{5} - \frac{\sqrt{2} i}{5} $  \end{solution} \part[1]  $ \frac{1}{{3 - i}} $  \begin{solution}  $ \frac{3}{10} + \frac{i}{10} $  \end{solution} \part[1]  $ \frac{{3i}}{{2 - 4i}} $  \begin{solution}  $ - \frac{3}{5} + \frac{3 i}{10} $  \end{solution} \part[1]  $ \frac{{5 - i}}{i} $  \begin{solution}  $ -1 - 5 i $  \end{solution} \part[1]  $ \frac{1 + 2i}{3 + 3i} $  \begin{solution}  $ \frac{1}{2} + \frac{i}{6} $  \end{solution} \part[1]  $ ({\sqrt {2}  - i})\frac{{\sqrt{ 2} +i}}{{1 - 2i}} $  \begin{solution}  $ \frac{3}{5} + \frac{6 i}{5} $  \end{solution} \part[1]  $ ( {2\sqrt {3} - i})\frac{{\sqrt {3} i}}{{1 + i}} $  \begin{solution}  $ \frac{\sqrt{3}}{2} + 3 - \frac{\sqrt{3} i}{2} + 3 i $  \end{solution} \part[1]  $ \frac{{1 - i}}{{3 + 2i}}\frac{{2i}}{{1 + i}} $  \begin{solution}  $ \frac{6}{13} - \frac{4 i}{13} $  \end{solution} \part[1]  $ \frac{{\sqrt {2} }}{{ - 2 - i}}\frac{1}{{2 + 3i}} $  \begin{solution}  $ - \frac{\sqrt{2}}{65} + \frac{8 \sqrt{2} i}{65} $  \end{solution}
        % \end{parts}
        % \end{multicols}
        
        \question[1] Escribe en forma binómica los siguientes números complejos:
        \begin{multicols}{2} 
        \begin{parts} 
        % \part[1]  $ 2_{\frac{\pi}{4}} $  \begin{solution}  $ \sqrt{2} + \sqrt{2} i $  \end{solution} 
        \part  $ 3_{\frac{\pi}{6}} $  \begin{solution}  $ \frac{3 \sqrt{3}}{2} + \frac{3 i}{2} $  \end{solution} 
        \part  $ \sqrt{3}_{\pi} $  \begin{solution}  $ - \sqrt{3} $  \end{solution} 
        % \part[1]  $ 17_{0} $  \begin{solution}  $ 17 $  \end{solution} \part[1]  $ 1_{\frac{\pi}{2}} $  \begin{solution}  $ i $  \end{solution} \part[1]  $ 5_{\frac{3 \pi}{2}} $  \begin{solution}  $ - 5 i $  \end{solution} \part[1]  $ 1_{\frac{5 \pi}{6}} $  \begin{solution}  $ - \frac{\sqrt{3}}{2} + \frac{i}{2} $  \end{solution} \part[1]  $ 4_{\frac{2 \pi}{3}} $  \begin{solution}  $ -2 + 2 \sqrt{3} i $  \end{solution}
        \end{parts}
        \end{multicols}
        
        
        % \question[1] Escribe los siguientes números complejos en forma polar 
        % % con el argumento en radianes
        % :
        % \begin{multicols}{3} 
        % \begin{parts} 
        % \part  $ -4 $\begin{solution}  $ 4_{\pi} $\end{solution} 
        % % \part $ 2 - 2\sqrt {3}i $  \begin{solution}  $ 4_{- \frac{\pi}{3}} $  \end{solution}
        % %  \part  $ 2i $  \begin{solution}  $ 2_{\frac{\pi}{2}} $  \end{solution}
        % \part[1]  $ 2-2i $  \begin{solution}  $ 2 \sqrt{2}_{- \frac{\pi}{4}} $  \end{solution}
        % \end{parts}
        % \end{multicols}
        
    \end{questions}
    \end{document}
    