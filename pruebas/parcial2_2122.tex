
        \documentclass[addpoints,spanish, 12pt,a4paper]{exam}
        %\documentclass[answers, spanish, 12pt,a4paper]{exam}
        
        \usepackage{pgf,tikz}
        \usetikzlibrary{arrows}
        
        \printanswers
        
        
        \pointpoints{punto}{puntos}
        \hpword{Puntos:}
        \vpword{Puntos:}
        \htword{Total}
        \vtword{Total}
        \hsword{Resultado:}
        \hqword{Ejercicio:}
        \vqword{Ejercicio:}

        \usepackage[utf8]{inputenc}
        \usepackage[spanish]{babel}
        \usepackage{eurosym}
        %\usepackage[spanish,es-lcroman, es-tabla, es-noshorthands]{babel}


        \usepackage[margin=1in]{geometry}
        \usepackage{amsmath,amssymb}
        \usepackage{multicol, xparse}

        \usepackage{yhmath}

        \usepackage{verbatim}
        %\usepackage{pstricks}


        \usepackage{graphicx}
        \graphicspath{{../img/}}

        \let\multicolmulticols\multicols
        \let\endmulticolmulticols\endmulticols
        \RenewDocumentEnvironment{multicols}{mO{}}
         {%
          \ifnum#1=1
            #2%
          \else % More than 1 column
            \multicolmulticols{#1}[#2]
          \fi
         }
         {%
          \ifnum#1=1
          \else % More than 1 column
            \endmulticolmulticols
          \fi
         }
        \renewcommand{\solutiontitle}{\noindent\textbf{Sol:}\enspace}

        \newcommand{\samedir}{\mathbin{\!/\mkern-5mu/\!}}

        \newcommand{\class}{1º Bachillerato}
        \newcommand{\examdate}{\today}

        %\newcommand{\tipo}{A}


        \newcommand{\timelimit}{50 minutos}

        \renewcommand{\solutiontitle}{\noindent\textbf{Solución:}\enspace}


        \pagestyle{head}
        \firstpageheader{\includegraphics[width=0.2\columnwidth]{header_left}}{\textbf{Departamento de Matemáticas\linebreak \class}\linebreak \examnum}{\includegraphics[width=0.1\columnwidth]{header_right}}
        \runningheader{\class}{\examnum}{Página \thepage\ of \numpages}
        \runningheadrule
        
        \pointsinrightmargin % Para poner las puntuaciones a la derecha. Se puede cambiar. Si se comenta, sale a la izquierda.
        \extrawidth{-2.4cm} %Un poquito más de margen por si ponemos textos largos.
        \marginpointname{\emph{\points}}

        \newcommand{\tipo}{A}
        \newcommand{\examnum}{Parcial 2ª evaluación}
        
        \begin{document}
        \noindent
        \begin{tabular*}{\textwidth}{l @{\extracolsep{\fill}} r @{\extracolsep{6pt}} }
        \textbf{Nombre:} \makebox[3.5in]{\hrulefill} & \textbf{Fecha:}\makebox[1in]{\hrulefill} \\
         & \\
        \textbf{Tiempo: \timelimit} & Tipo: \tipo 
        \end{tabular*}
        \rule[2ex]{\textwidth}{2pt}
        Esta prueba tiene \numquestions\ ejercicios. La puntuación máxima es de \numpoints. 
        La nota final de la prueba será la parte proporcional de la puntuación obtenida sobre la puntuación máxima. 

        \begin{center}


        \addpoints
             %\gradetable[h][questions]
            \pointtable[h][questions]
        \end{center}

        \noindent
        \rule[2ex]{\textwidth}{2pt}


        \begin{questions}
        
        \question Dados los vectores $\overrightarrow{u}=\left(2, -3\right)$ y $\overrightarrow{v}=\left(3, 6\right)$:
        \begin{parts}
        \part[1] Demuestra que $B=\left\{\overrightarrow{u},\overrightarrow{v}\right\}$ forma una base del plano
        \begin{solution}$\frac{2}{3}\neq \frac{-3}{6}$\end{solution}
        \part[1] Calcula las coordenadas del vector $\overrightarrow{w}=\left(6, -\frac{11}{12}\right)$ respecto de la base anterior
        \begin{solution}$(\frac{5}{2},\frac{1}{3})$\end{solution}
        \end{parts}
        
        \question Respecto de una base ortonormal tenemos dos vectores $\overrightarrow{u}$ y $\overrightarrow{v}$.
Calcular $\overrightarrow{u}\cdot\overrightarrow{v}$, $|\overrightarrow{u}| \ y \ |\overrightarrow{v}|$ 
y $\angle(\overrightarrow{u},\overrightarrow{v})$ siendo:
        \begin{multicols}{1} 
        \begin{parts} \part[1]  $ \overrightarrow{u}=(8, 2) \ , \ \overrightarrow{v}=(-3, 4) $  \begin{solution}  $ \left [ -16, \quad \left [ 2 \sqrt{17}, \quad 5\right ], \quad 112.833654177918\right ] $  \end{solution}
        \end{parts}
        \end{multicols}
        
        % \question[2] Calcula los valores de $k$ para que el ángulo formado por $\overrightarrow{v}=\left(k, -k\right)$ y $\overrightarrow{w}=\left(k+1, k-1\right)$ sea de $45^\circ$
        % \begin{solution}
        % $k = 1 \ , k= -1$   
        % \end{solution}
   
        \question Demuestra que:
        \begin{parts}
        \part[1] $\sec^2 \alpha + \cosec^2 \alpha = \dfrac{1}{\sec^2 \alpha \cdot \cosec^2 \alpha}$
        % \part[1] $\tg \alpha + \cotg \alpha = \sec \alpha \cdot \cosec \alpha$
        \end{parts}
        
        % \question Si $\sin\alpha=-\frac{3}{5}\land \alpha \in III$, calcula (sin usar la calculadora, con identidades trigonométricas y reducción al primer cuadrante):
        % \begin{multicols}{1} 
        % \begin{parts} \part[1]  $ \sin(2\alpha) $  y $ \cos\alpha $ \begin{solution}  $ \frac{24}{25} $ y $ - \frac{4}{5} $ \end{solution} 
        % \part[1]  $ \tan\alpha $ y $ \cos(\pi+\alpha) $   \begin{solution}  $ \frac{3}{4} $ y $ \frac{3}{5} $  \end{solution} 
        % \end{parts}
        % \end{multicols}
        
        
        \question Sabiendo que $\tg \alpha = 3$, con $0^\circ < \alpha < 90^\circ$, calcula (sin utilizar las teclas trigonométricas de la calculadora):
        \begin{parts}
        \part[1] $\tg \left( 2 \alpha \right)$\begin{solution} $-  \frac{3}{4}$\end{solution}
        \part[1] $\sen \left( 30^\circ + \alpha \right)$\begin{solution} $  \frac{3\sqrt{10}}{10}$\end{solution}
        \end{parts}
        
        \question Resolver las siguientes ecuaciones:
        \begin{multicols}{1} 
        \begin{parts} 
        % \part[2]  $ \sen{x}+\cos{x}=\sqrt{2} $  \begin{solution}  $ \left [ 45.0\right ] $  \end{solution}
        \part[2]  $ \sen{x}+\cos{x}=0 $  \begin{solution}  $ \left [ 135, 315\right ] $  \end{solution}
        % \part[2]  $ \sen{2x}+\sen{x}=0 $  \begin{solution}  $ \left [ 0, 60, 180, 300, 360\right ] $  \end{solution}
        \end{parts}
        \end{multicols}
        
        
        % \question Calcular el área de un triángulo sabiendo que:
        % \begin{multicols}{1} 
        % \begin{parts} \part[3] El lado a=20, y los ángulos B=30º, y C=105º  \begin{solution}  $ 5 \sqrt{2} \left(5 \sqrt{2} + 5 \sqrt{6}\right) \to 136.60254037844388 $  \end{solution}
        % \end{parts}
        % \end{multicols}
        % \question Dado el siguiente número $z$, calcula $\frac{z+\overline{z}}{z-\overline{z}}$ dando el resultado en forma compleja binómica:
        % \begin{multicols}{1} 
        % \begin{parts} \part[2]  $ 12\sqrt{3}-3i $  \begin{solution}  $ 4 \sqrt{3} i $  \end{solution}
        % \end{parts}
        % \end{multicols}
        
    \end{questions}
    
    
    % \begin{tikzpicture}[>=triangle 45]\draw [->,red] (0,0) -- (1,1);\end{tikzpicture}

    \end{document}
    