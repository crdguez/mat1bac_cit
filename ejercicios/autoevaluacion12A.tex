
        \documentclass[addpoints,spanish, 12pt,a4paper]{exam}
        %\documentclass[answers, spanish, 12pt,a4paper]{exam}
        
        \pointpoints{punto}{puntos}
        \hpword{Puntos:}
        \vpword{Puntos:}
        \htword{Total}
        \vtword{Total}
        \hsword{Resultado:}
        \hqword{Ejercicio:}
        \vqword{Ejercicio:}

        \usepackage[utf8]{inputenc}
        \usepackage[spanish]{babel}
        \usepackage{eurosym}
        %\usepackage[spanish,es-lcroman, es-tabla, es-noshorthands]{babel}


        \usepackage[margin=1in]{geometry}
        \usepackage{amsmath,amssymb}
        \usepackage{multicol, xparse}

        \usepackage{yhmath}

        \usepackage{verbatim}
        %\usepackage{pstricks}


        \usepackage{graphicx}
        \graphicspath{{../img/}}




        \let\multicolmulticols\multicols
        \let\endmulticolmulticols\endmulticols
        \RenewDocumentEnvironment{multicols}{mO{}}
         {%
          \ifnum#1=1
            #2%
          \else % More than 1 column
            \multicolmulticols{#1}[#2]
          \fi
         }
         {%
          \ifnum#1=1
          \else % More than 1 column
            \endmulticolmulticols
          \fi
         }
        \renewcommand{\solutiontitle}{\noindent\textbf{Sol:}\enspace}

        \newcommand{\samedir}{\mathbin{\!/\mkern-5mu/\!}}

        \newcommand{\class}{1º Bachillerato}
        \newcommand{\examdate}{\today}

        %\newcommand{\tipo}{A}


        \newcommand{\timelimit}{50 minutos}

        \renewcommand{\solutiontitle}{\noindent\textbf{Solución:}\enspace}


        \pagestyle{head}
        \firstpageheader{\includegraphics[width=0.2\columnwidth]{header_left}}{\textbf{Departamento de Matemáticas\linebreak \class}\linebreak \examnum}{\includegraphics[width=0.1\columnwidth]{header_right}}
        \runningheader{\class}{\examnum}{Página \thepage\ of \numpages}
        \runningheadrule
        
        \pointsinrightmargin % Para poner las puntuaciones a la derecha. Se puede cambiar. Si se comenta, sale a la izquierda.
        \extrawidth{-2.4cm} %Un poquito más de margen por si ponemos textos largos.
        \marginpointname{ \emph{\points}}

        %\printanswers
            \newcommand{\tipo}{A}\newcommand{\examnum}{Autoevaluación12}
        \begin{document}
        \noindent
        \begin{tabular*}{\textwidth}{l @{\extracolsep{\fill}} r @{\extracolsep{6pt}} }
        \textbf{Nombre:} \makebox[3.5in]{\hrulefill} & \textbf{Fecha:}\makebox[1in]{\hrulefill} \\
         & \\
        \textbf{Tiempo: \timelimit} & Tipo: \tipo 
        \end{tabular*}
        \rule[2ex]{\textwidth}{2pt}
        Esta prueba tiene \numquestions\ ejercicios. La puntuación máxima es de \numpoints. 
        La nota final de la prueba será la parte proporcional de la puntuación obtenida sobre la puntuación máxima. 

        \begin{center}


        \addpoints
             %\gradetable[h][questions]
            \pointtable[h][questions]
        \end{center}

        \noindent
        \rule[2ex]{\textwidth}{2pt}

        \begin{questions}
        \question Resolver las siguientes inecuaciones:
        \begin{multicols}{1} 
        \begin{parts} \part[1]  $ \left|{2 x + 6}\right| - 0.5\geq 0 $  \begin{solution}  $ \left(-2.75 \leq x \wedge x < \infty\right) \vee \left(x \leq -3.25 \wedge - \infty < x\right) $  \end{solution} \part[1]  $ \left|{2 x + 5}\right| - 2\leq 0 $  \begin{solution}  $ - \frac{7}{2} \leq x \wedge x \leq - \frac{3}{2} $  \end{solution} \part[1]  $ \left|{x - 4}\right| - 2< 0 $  \begin{solution}  $ 2 < x \wedge x < 6 $  \end{solution}
        \end{parts}
        \end{multicols}
        \question Efectúa simplificando el resultado si es posible:
        \begin{multicols}{1} 
        \begin{parts} \part[1]  $ \frac{3x^2+1}{x^2+x}  - \frac{2x}{x+1} $  \begin{solution}  $ \frac{x^{2} + 1}{x^{2} + x} $  \end{solution} \part[1]  $ \frac{1}{x^2-x} + \frac{2x-1}{x-1} - \frac{3x-1}{x} $  \begin{solution}  $ - \frac{x - 3}{x - 1} $  \end{solution} \part[1]  $ \frac{\frac{x^2-2x+1}{x + 3}}{\frac{x-1}{x^2 -9 }} $  \begin{solution}  $ x^{2} - 4 x + 3 $  \end{solution}
        \end{parts}
        \end{multicols}
        \question Resuelve mediante expresiones algebraicas:
        \begin{multicols}{1} 
        \begin{parts} \part[1] En un corral hay conejos y gallinas, en total 50 cabezas y 134 patas. 
    ¿Cuántos animales hay de cada clase?  \begin{solution}  $ \left\{\begin{matrix}50=x+y\\ 134=4x+2y\\ \end{matrix}\right.  \rightarrow  \\\left[\begin{matrix}1 & 1 & 50\\0 & -2 & -66\end{matrix}\right] \rightarrow  \left \{ x : 17, \quad y : 33\right \} $  \end{solution} \part[1] Un librero vendió 84 libros, unos a 45 euros y otros a 36 y obtuvo de la venta 3.105 euros. ¿Cuántos vendió de
cada clase?  \begin{solution}  $ \left\{\begin{matrix}3105=45x+36y\\ 84=x+y\\ \end{matrix}\right.  \rightarrow  \\\left[\begin{matrix}45 & 36 & 3105\\0 & \frac{1}{5} & 15\end{matrix}\right] \rightarrow  \left \{ x : 9, \quad y : 75\right \} $  \end{solution}
        \end{parts}
        \end{multicols}
        \question Resuelve los sistemas:
        \begin{multicols}{1} 
        \begin{parts} \part[1]  $ \left\{\begin{matrix}\frac{x}{2} + \frac{y}{3} + z = -\frac{1}{2}\\ x - \frac{y}{2} + \frac{z}{3} = -1\\  \frac{x}{3} - y - \frac{z}{2} = -\frac{1}{6}\\ \end{matrix}\right. $  \begin{solution}  $ \left[\begin{matrix}\frac{1}{2} & \frac{1}{3} & 1 & - \frac{1}{2}\\0 & - \frac{7}{6} & - \frac{5}{3} & 0\\0 & 0 & \frac{73}{126} & \frac{1}{6}\end{matrix}\right] \rightarrow  \\ \left \{ x : - \frac{95}{73}, \quad y : - \frac{30}{73}, \quad z : \frac{21}{73}\right \} $  \end{solution} \part[1]  $ \left\{\begin{matrix}\frac{x}{2} + \frac{y}{3} + \frac{z}{3} = -2\\ \frac{x}{3} - \frac{y}{2} + \frac{z}{3} = 2\\ \frac{x}{6} + \frac{y}{2} + \frac{z}{2} = 1\\ \end{matrix}\right. $  \begin{solution}  $ \left[\begin{matrix}\frac{1}{2} & \frac{1}{3} & \frac{1}{3} & -2\\0 & - \frac{13}{18} & \frac{1}{9} & \frac{10}{3}\\0 & 0 & \frac{35}{78} & \frac{45}{13}\end{matrix}\right] \rightarrow  \\ \left \{ x : - \frac{48}{7}, \quad y : - \frac{24}{7}, \quad z : \frac{54}{7}\right \} $  \end{solution}
        \end{parts}
        \end{multicols}
        
    \end{questions}
    \end{document}
    