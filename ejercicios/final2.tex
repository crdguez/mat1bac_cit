
        \documentclass[spanish, 11pt]{exam}

        %These tell TeX which packages to use.
        \usepackage{array,epsfig}
        \usepackage{amsmath, textcomp}
        \usepackage{amsfonts}
        \usepackage{amssymb}
        \usepackage{amsxtra}
        \usepackage{amsthm}
        \usepackage{mathrsfs}
        \usepackage{color}
        \usepackage{multicol, xparse}
        \usepackage{verbatim}


        \usepackage[utf8]{inputenc}
        \usepackage[spanish]{babel}
        \usepackage{eurosym}

        \usepackage{graphicx}
        \graphicspath{{../img/}}



        \printanswers
        \nopointsinmargin
        \pointformat{}

        %Pagination stuff.
        %\setlength{\topmargin}{-.3 in}
        %\setlength{\oddsidemargin}{0in}
        %\setlength{\evensidemargin}{0in}
        %\setlength{\textheight}{9.in}
        %\setlength{\textwidth}{6.5in}
        %\pagestyle{empty}

        \let\multicolmulticols\multicols
        \let\endmulticolmulticols\endmulticols
        \RenewDocumentEnvironment{multicols}{mO{}}
         {%
          \ifnum#1=1
            #2%
          \else % More than 1 column
            \multicolmulticols{#1}[#2]
          \fi
         }
         {%
          \ifnum#1=1
          \else % More than 1 column
            \endmulticolmulticols
          \fi
         }
        \renewcommand{\solutiontitle}{\noindent\textbf{Sol:}\enspace}

        \newcommand{\samedir}{\mathbin{\!/\mkern-5mu/\!}}

        \newcommand{\class}{1º Bachillerato}
        \newcommand{\examdate}{\today}

        \newcommand{\tipo}{A}


        \newcommand{\timelimit}{50 minutos}



        \pagestyle{head}
        \firstpageheader{\includegraphics[width=0.2\columnwidth]{header_left}}{\textbf{Departamento de Matemáticas\linebreak \class}\linebreak \examnum}{\includegraphics[width=0.1\columnwidth]{header_right}}
        \runningheader{\class}{\examnum}{Página \thepage\ of \numpages}
        \runningheadrule

        \newcommand{\examnum}{Final 2ª Ev.}
        \begin{document}
        \begin{questions}
        \question ex12e01 - Justifica si los siguientes pares de vectores forman base de $\mathbb{R}^2$:
        \begin{multicols}{2} 
        \begin{parts} \part[1]  $ \overrightarrow{u}=(4, 12) \, \ \overrightarrow{v}=(2, 6) $  \begin{solution}  $ False $  \end{solution} \part[1]  $ \overrightarrow{u}=(1, 2) \, \ \overrightarrow{v}=(3, 4) $  \begin{solution}  $ True $  \end{solution}
        \end{parts}
        \end{multicols}
        \question ex12e02 - Determinar el ángulo formado por las rectas:
        \begin{multicols}{2} 
        \begin{parts} \part[1]  $ r\equiv2x-y-2=0\  y \ s\equiv3x+2y-4=0 $  \begin{solution}  $ 119.74488129694222 $  \end{solution}
        \end{parts}
        \end{multicols}
        \question ex12e03 - Calcula el vértice C de un triángulo isósceles ABC, sabiendo que:
        \begin{multicols}{1} 
        \begin{parts} \part[1]  $  A(2, -3)\  , \ B=(5, 2) \ y \  C \in r\equiv -x+3y-16=0 $  \begin{solution}  $ \left [ \left \{ x : -4, \quad y : 4\right \}\right ] $  \end{solution}
        \end{parts}
        \end{multicols}
        \question ex12e04 - Dado el triángulo ABC, hallar: \\ 
\begin{itemize} \item la mediana correspondiente al vértice A \item la mediatriz correspondiente al lado AB \item y el área \end{itemize} Siendo:
        \begin{multicols}{1} 
        \begin{parts} \part[1]  $ A=(2, 1), \ B=(4, 3),\  y \ C=(6, -1) $  \begin{solution}  $ \left [ y - 1 = 0, \quad - 2 x - 2 y + 10 = 0, \quad \left [ 3 \sqrt{2}, \quad 6\right ]\right ] $  \end{solution}
        \end{parts}
        \end{multicols}
        \question ex12e05 - Resolver las siguientes ecuaciones:
        \begin{multicols}{1} 
        \begin{parts} \part[1]  $ 2\cos^2{x}-\sqrt{3}\cos{x}=0 $  \begin{solution}  $ \left [ 30, \quad 90, \quad 270, \quad 330\right ] $  \end{solution} \part[1]  $ \cos{2x}-3\cos{x}+1=0 $  \begin{solution}  $ \left [ -90, \quad 90\right ] $  \end{solution}
        \end{parts}
        \end{multicols}
        \question ex12e06 - Calcula:
        \begin{multicols}{2} 
        \begin{parts} \part[1]  $ \frac{(1+2i)i^7}{(3-2i)-(2+i)} $  \begin{solution}  $ \frac{1}{2} + \frac{i}{2} $  \end{solution}
        \end{parts}
        \end{multicols}
        \question ex12e07 - Escribe los siguientes números complejos en forma polar con el argumento en radianes:
        \begin{multicols}{2} 
        \begin{parts} \part[1]  $ 2 - 2\sqrt {3}i $  \begin{solution}  $ 4_{- \frac{\pi}{3}} $  \end{solution} \part[1]  $ 2i $  \begin{solution}  $ 2_{\frac{\pi}{2}} $  \end{solution} \part[1]  $ -4 $  \begin{solution}  $ 4_{\pi} $  \end{solution}
        \end{parts}
        \end{multicols}
        
    \end{questions}
    \end{document}
    