
        \documentclass[addpoints,spanish, 12pt,a4paper]{exam}
        %\documentclass[answers, spanish, 12pt,a4paper]{exam}
        
        \pointpoints{punto}{puntos}
        \hpword{Puntos:}
        \vpword{Puntos:}
        \htword{Total}
        \vtword{Total}
        \hsword{Resultado:}
        \hqword{Ejercicio:}
        \vqword{Ejercicio:}

        \usepackage[utf8]{inputenc}
        \usepackage[spanish]{babel}
        \usepackage{eurosym}
        %\usepackage[spanish,es-lcroman, es-tabla, es-noshorthands]{babel}


        \usepackage[margin=1in]{geometry}
        \usepackage{amsmath,amssymb}
        \usepackage{multicol, xparse}

        \usepackage{yhmath}

        \usepackage{verbatim}
        %\usepackage{pstricks}


        \usepackage{graphicx}
        \graphicspath{{../img/}}




        \let\multicolmulticols\multicols
        \let\endmulticolmulticols\endmulticols
        \RenewDocumentEnvironment{multicols}{mO{}}
         {%
          \ifnum#1=1
            #2%
          \else % More than 1 column
            \multicolmulticols{#1}[#2]
          \fi
         }
         {%
          \ifnum#1=1
          \else % More than 1 column
            \endmulticolmulticols
          \fi
         }
        \renewcommand{\solutiontitle}{\noindent\textbf{Sol:}\enspace}

        \newcommand{\samedir}{\mathbin{\!/\mkern-5mu/\!}}

        \newcommand{\class}{1º Bachillerato}
        \newcommand{\examdate}{\today}

        %\newcommand{\tipo}{A}


        \newcommand{\timelimit}{50 minutos}

        \renewcommand{\solutiontitle}{\noindent\textbf{Solución:}\enspace}


        \pagestyle{head}
        \firstpageheader{\includegraphics[width=0.2\columnwidth]{header_left}}{\textbf{Departamento de Matemáticas\linebreak \class}\linebreak \examnum}{\includegraphics[width=0.1\columnwidth]{header_right}}
        \runningheader{\class}{\examnum}{Página \thepage\ of \numpages}
        \runningheadrule
        
        \pointsinrightmargin % Para poner las puntuaciones a la derecha. Se puede cambiar. Si se comenta, sale a la izquierda.
        \extrawidth{-2.4cm} %Un poquito más de margen por si ponemos textos largos.
        \marginpointname{ \emph{\points}}

        \printanswers
            \newcommand{\tipo}{A}\newcommand{\examnum}{Recuperación 2ª evaluación}
        \begin{document}
        \noindent
        \begin{tabular*}{\textwidth}{l @{\extracolsep{\fill}} r @{\extracolsep{6pt}} }
        \textbf{Nombre:} \makebox[3.5in]{\hrulefill} & \textbf{Fecha:}\makebox[1in]{\hrulefill} \\
         & \\
        \textbf{Tiempo: \timelimit} & Tipo: \tipo 
        \end{tabular*}
        \rule[2ex]{\textwidth}{2pt}
        Esta prueba tiene \numquestions\ ejercicios. La puntuación máxima es de \numpoints. 
        La nota final de la prueba será la parte proporcional de la puntuación obtenida sobre la puntuación máxima. 

        \begin{center}


        \addpoints
             %\gradetable[h][questions]
            \pointtable[h][questions]
        \end{center}

        \noindent
        \rule[2ex]{\textwidth}{2pt}

        \begin{questions}
        
                \question Dado el triángulo ABC de coordenadas A=(2, 1), B=(6, 5),  y C=(5, 1), hallar: 
    
       \begin{multicols}{1}
      \begin{parts} 
      \part[1] El área del triángulo 
         \begin{solution} \\La base mide: $4 \sqrt{2}$ \\La altura mide: $\frac{3 \sqrt{2}}{2}$ \\El área es: $6$   
       \end{solution} 
       \part[1] El ángulo $\widehat{A}$.  \begin{solution}   
       El ángulo mide: $45$   \end{solution} \part[2] El punto simétrico de C respecto a AB  \begin{solution}   $r_{AB} \equiv - 4 x + 4 y + 4=0$, \\ $t_{C}\bot r_{AB} \equiv - 4 x - 4 y + 24=0$, \\  $r\bot t \equiv Point2D(7/2, 5/2)$, \\ El punto simétrico: $Point2D(2, 4)$   \end{solution}
        \end{parts}
        \end{multicols}


\question[1]  Escribe en forma binómica los siguientes números complejos:
        \begin{multicols}{4}
        \begin{parts} \part  $ 2_{\frac{\pi}{4}} $  \begin{solution}  $ \sqrt{2} + \sqrt{2} i $  \end{solution} \part  $ 3_{\frac{\pi}{6}} $  \begin{solution}  $ \frac{3 \sqrt{3}}{2} + \frac{3 i}{2} $  \end{solution} \part  $ \sqrt{2}_{\pi} $  \begin{solution}  $ - \sqrt{2} $  \end{solution} \part[]  $ 3_{- \frac{\pi}{2}} $  \begin{solution}  $ - 3 i $  \end{solution}
        \end{parts}
        \end{multicols}  

        \question Resuelve:
        \begin{multicols}{1}
        \begin{parts} \part[2]  $ 2\sin^2{x}+\cos{2x}=4\cos^2{x} $  \begin{solution}  $ \left [ -120, \quad 120, \quad -60, \quad 60\right ] $  \end{solution} \part[2]  $ \left(\cos^2{x}-\sin^2{x}\right)^2=\frac{1}{4}+\sin{2x} $  \begin{solution}  $ \left [ 15, \quad 75, \quad \frac{180 \left(- \frac{98174770424681}{125000000000000} - \frac{481211825059603 i}{1000000000000000}\right)}{\pi}, \quad \frac{180 \left(- \frac{98174770424681}{125000000000000} + \frac{481211825059603 i}{1000000000000000}\right)}{\pi}\right ] $  \end{solution}
        \end{parts}
        \end{multicols}
        
      \question Determina el valor que debe tener \emph{k} ($k \in \mathbb{R}$)para que la siguiente expresión sea un número imaginario puro.
        \begin{multicols}{1}
        \begin{parts} \part[1]  $ (k-2i)^2+(-5+2i) $  \begin{solution}  $ k^{2} - 4 i k - 9 + 2 i\to\left [ -3, \quad 3\right ] $  \end{solution}
       
    \end{questions}
    \end{document}
    
