
        \documentclass[spanish, 11pt]{exam}

        %These tell TeX which packages to use.
        \usepackage{array,epsfig}
        \usepackage{amsmath, textcomp}
        \usepackage{amsfonts}
        \usepackage{amssymb}
        \usepackage{amsxtra}
        \usepackage{amsthm}
        \usepackage{mathrsfs}
        \usepackage{color}
        \usepackage{multicol, xparse}
        \usepackage{verbatim}


        \usepackage[utf8]{inputenc}
        \usepackage[spanish]{babel}
        \usepackage{eurosym}

        \usepackage{graphicx}
        \graphicspath{{../img/}}
        \usepackage{pgf}



        \printanswers
        \nopointsinmargin
        \pointformat{}

        %Pagination stuff.
        %\setlength{\topmargin}{-.3 in}
        %\setlength{\oddsidemargin}{0in}
        %\setlength{\evensidemargin}{0in}
        %\setlength{\textheight}{9.in}
        %\setlength{\textwidth}{6.5in}
        %\pagestyle{empty}

        \let\multicolmulticols\multicols
        \let\endmulticolmulticols\endmulticols
        \RenewDocumentEnvironment{multicols}{mO{}}
         {%
          \ifnum#1=1
            #2%
          \else % More than 1 column
            \multicolmulticols{#1}[#2]
          \fi
         }
         {%
          \ifnum#1=1
          \else % More than 1 column
            \endmulticolmulticols
          \fi
         }
        \renewcommand{\solutiontitle}{\noindent\textbf{Sol:}\enspace}

        \newcommand{\samedir}{\mathbin{\!/\mkern-5mu/\!}}

        \newcommand{\class}{1º Bachillerato}
        \newcommand{\examdate}{\today}

        \newcommand{\tipo}{A}


        \newcommand{\timelimit}{50 minutos}



        \pagestyle{head}
        \firstpageheader{\includegraphics[width=0.2\columnwidth]{header_left}}{\textbf{Departamento de Matemáticas\linebreak \class}\linebreak \examnum}{\includegraphics[width=0.1\columnwidth]{header_right}}
        \runningheader{\class}{\examnum}{Página \thepage\ of \numpages}
        \runningheadrule

        \newcommand{\examnum}{43 - Probabilidad}
        \begin{document}
        \begin{questions}
        \question p098e05 - De los 39 alumnos de una clase, 16 escogieron francés, como idioma y 27 inglés. 9 eligieron ambos idiomas y
el resto no optó por ninguno de ellos. elegido un alumno al azar, calcula las probabilidades de que escogiera:

        \begin{multicols}{2}
        \begin{parts} \part[1] i) Francés \\ ii) Inglés \\ iii) Ambos idiomas \\ iv) Francés o Inglés \\ v) Francés, pero no inglés \\ vi) Inglés, pero no francés  \begin{solution}  $ \left [ \frac{16}{39}, \quad \frac{9}{13}, \quad \frac{3}{13}, \quad \frac{34}{39}, \quad \frac{7}{39}, \quad \frac{6}{13}\right ] $  \end{solution}
        \end{parts}
        \end{multicols}
        \question p098e06 - En la ciudad, el 53\% de sus habitantes es mayor de 30 años, el 45\% está
casado y el 60\% está casado o es mayor de 30 años. 
Calcula la probabilidad de los siguientes sucesos:
        \begin{multicols}{2}
        \begin{parts} \part[1] i) Ser mayor de 30 años y estar casado \\ ii) No estar casado   \begin{solution}  $ \left [ \frac{19}{50}, \quad \frac{11}{20}\right ] $  \end{solution}
        \end{parts}
        \end{multicols}
        \question p098e08 - Se tiene una urna con 15 bolas negras y 10 blancas, y se realizan dos extracciones 
sucesivas de una bola. 
Halla la probabilidad de que las dos bolas sean blancas en los siguientes casos:
        \begin{multicols}{2}
        \begin{parts} \part[1] i) Con devolución a la urna de la primera bola extraída ii) Sin devolución
  \begin{solution}  $ \left [ \frac{4}{25}, \quad \frac{3}{20}\right ] $  \end{solution}
        \end{parts}
        \end{multicols}
        \question p098e09 - Una urna contiene 3 bolas rojas, 2 verdes y 1 azul. Extraemos una bola, anotamos su color, la devolvemos a
la urna, sacamos otra bola y anotamos su color. Halla las siguientes probabilidades:
        \begin{multicols}{2}
        \begin{parts} \part[1] i) Que las dos bolas sean rojas
ii) Que haya alguna bola azul
iii) que no haya ninguna bola verde
iv) que las dos bolas sean del mismo color
  \begin{solution}  $ \left [ \frac{1}{4}, \quad \frac{11}{36}, \quad \frac{4}{9}, \quad \frac{4}{9}\right ] $  \end{solution}
        \end{parts}
        \end{multicols}
        
    \end{questions}
    \end{document}
    