
        \documentclass[spanish, 11pt]{exam}

        %These tell TeX which packages to use.
        \usepackage{array,epsfig}
        \usepackage{amsmath, textcomp}
        \usepackage{amsfonts}
        \usepackage{amssymb}
        \usepackage{amsxtra}
        \usepackage{amsthm}
        \usepackage{mathrsfs}
        \usepackage{color}
        \usepackage{multicol, xparse}
        \usepackage{verbatim}


        \usepackage[utf8]{inputenc}
        \usepackage[spanish]{babel}
        \usepackage{eurosym}

        \usepackage{graphicx}
        \graphicspath{{../img/}}



        \printanswers
        \nopointsinmargin
        \pointformat{}

        %Pagination stuff.
        %\setlength{\topmargin}{-.3 in}
        %\setlength{\oddsidemargin}{0in}
        %\setlength{\evensidemargin}{0in}
        %\setlength{\textheight}{9.in}
        %\setlength{\textwidth}{6.5in}
        %\pagestyle{empty}

        \let\multicolmulticols\multicols
        \let\endmulticolmulticols\endmulticols
        \RenewDocumentEnvironment{multicols}{mO{}}
         {%
          \ifnum#1=1
            #2%
          \else % More than 1 column
            \multicolmulticols{#1}[#2]
          \fi
         }
         {%
          \ifnum#1=1
          \else % More than 1 column
            \endmulticolmulticols
          \fi
         }
        \renewcommand{\solutiontitle}{\noindent\textbf{Sol:}\enspace}

        \newcommand{\samedir}{\mathbin{\!/\mkern-5mu/\!}}

        \newcommand{\class}{1º Bachillerato}
        \newcommand{\examdate}{\today}

        \newcommand{\tipo}{A}


        \newcommand{\timelimit}{50 minutos}



        \pagestyle{head}
        \firstpageheader{\includegraphics[width=0.2\columnwidth]{header_left}}{\textbf{Departamento de Matemáticas\linebreak \class}\linebreak \examnum}{\includegraphics[width=0.1\columnwidth]{header_right}}
        \runningheader{\class}{\examnum}{Página \thepage\ of \numpages}
        \runningheadrule

        \newcommand{\examnum}{Autoevaluación 2 ev2}
        \begin{document}
        \begin{questions}
        \question pa21e04 - Respecto de una base ortonormal tenemos dos vectores $\overrightarrow{u}$ y $\overrightarrow{v}$.
Calcular $\overrightarrow{u}\cdot\overrightarrow{v}$, $|\overrightarrow{u}| \ y \ |\overrightarrow{v}|$ 
y $\angle(\overrightarrow{u},\overrightarrow{v})$ siendo:
        \begin{multicols}{2} 
        \begin{parts} \part[1]  $ \overrightarrow{u}=(2, -6) \, \ \overrightarrow{v}=(5, 6) $  \begin{solution}  $ \left [ -26, \quad \left [ 2 \sqrt{10}, \quad \sqrt{61}\right ], \quad 121.759480084813\right ] $  \end{solution} \part[1]  $ \overrightarrow{u}=(2, 5) \, \ \overrightarrow{v}=(4, 6) $  \begin{solution}  $ \left [ 38, \quad \left [ \sqrt{29}, \quad 2 \sqrt{13}\right ], \quad 11.888658039628\right ] $  \end{solution}
        \end{parts}
        \end{multicols}
        \question pa21e05 - Calcula x para que los vectores
$\overrightarrow{u}$ y $\overrightarrow{v}$ formen 60º siendo: 
        \begin{multicols}{2} 
        \begin{parts} \part[1]  $ \overrightarrow{u}=(6, x) \, \ \overrightarrow{v}=(10, 2) $  \begin{solution}  $ \left [ \frac{60}{11} + \frac{78 \sqrt{3}}{11}, \quad - \frac{78 \sqrt{3}}{11} + \frac{60}{11}\right ] $  \end{solution}
        \end{parts}
        \end{multicols}
        \question pa21e06 - Resolver las siguientes ecuaciones para ángulos en el primer cuadrante:
        \begin{multicols}{1} 
        \begin{parts} \part[1]  $ \sin{2x}=\frac{\sqrt {2}}{2} $  \begin{solution}  $ \left [ \frac{\pi}{8}, \quad \frac{3 \pi}{8}\right ] $  \end{solution} \part[1]  $ \tan{\frac{x}{2}}=\sqrt{3} $  \begin{solution}  $ \left [ \frac{2 \pi}{3}\right ] $  \end{solution} \part[1]  $ \sin{(3x-\frac{\pi}{2})}=-\frac{\sqrt{2}}{2} $  \begin{solution}  $ \left [ \frac{\pi}{12}, \quad \frac{7 \pi}{12}\right ] $  \end{solution}
        \end{parts}
        \end{multicols}
        \question pa21e07 - Resolver las siguientes ecuaciones:
        \begin{multicols}{1} 
        \begin{parts} \part[1]  $ \tan{2x}=\cot{x} $  \begin{solution}  $ \left [ -90.0, \quad 90.0, \quad -150.0, \quad 150.0, \quad -30.0, \quad 30.0\right ] $  \end{solution} \part[1]  $ \sin{x}\cos{x}=-\frac{1}{2} $  \begin{solution}  $ \left [ -45.0, \quad 135.0\right ] $  \end{solution} \part[1]  $ \sqrt{3}\sin{x}+\cos{x}=-2 $  \begin{solution}  $ \left [ -120.0\right ] $  \end{solution}
        \end{parts}
        \end{multicols}
        \question pa22e08 - Dado el siguiente número $z$, calcula el valor de $\frac{z-\overline{z}}{z+\overline{z}}$
        \begin{multicols}{2} 
        \begin{parts} \part[1]  $ \sqrt{6}-2\sqrt{4}i $  \begin{solution}  $ - \frac{2 \sqrt{6} i}{3} $  \end{solution} \part[1]  $ \sqrt{4}-2\sqrt{6}i $  \begin{solution}  $ - \sqrt{6} i $  \end{solution} \part[1]  $ \sqrt{6}-2\sqrt{4}i $  \begin{solution}  $ - \frac{2 \sqrt{6} i}{3} $  \end{solution} \part[1]  $ \sqrt{4}-2\sqrt{6}i $  \begin{solution}  $ - \sqrt{6} i $  \end{solution}
        \end{parts}
        \end{multicols}
        \question pa22e08b - Calcular el módulo y el argumento (en radianes) de los siguientes números complejos:
        \begin{multicols}{2} 
        \begin{parts} \part[1]  $ 4 - 2\sqrt {3}i $  \begin{solution}  $ 2 \sqrt{7}_{- \operatorname{atan}{\left (\frac{\sqrt{3}}{2} \right )}} $  \end{solution} \part[1]  $ -1-i $  \begin{solution}  $ \sqrt{2}_{- \frac{3 \pi}{4}} $  \end{solution} \part[1]  $  -6-6i $  \begin{solution}  $ 6 \sqrt{2}_{- \frac{3 \pi}{4}} $  \end{solution} \part[1]  $ -5i $  \begin{solution}  $ 5_{- \frac{\pi}{2}} $  \end{solution} \part[1]  $ 3 $  \begin{solution}  $ 3_{0} $  \end{solution}
        \end{parts}
        \end{multicols}
        \question pa22e08c - Escribe en forma binómica los siguientes números complejos:
        \begin{multicols}{2} 
        \begin{parts} \part[1]  $ 3_{\frac{\pi}{4}} $  \begin{solution}  $ \frac{3 \sqrt{2}}{2} + \frac{3 \sqrt{2} i}{2} $  \end{solution} \part[1]  $ 2_{\frac{\pi}{6}} $  \begin{solution}  $ \sqrt{3} + i $  \end{solution} \part[1]  $ 2 \sqrt{2}_{\pi} $  \begin{solution}  $ - 2 \sqrt{2} $  \end{solution} \part[1]  $ 7_{0} $  \begin{solution}  $ 7 $  \end{solution} \part[1]  $ 1_{\frac{\pi}{2}} $  \begin{solution}  $ i $  \end{solution}
        \end{parts}
        \end{multicols}
        \question pa22e09 - Calcula el área del triángulo de vértices:
        \begin{multicols}{1} 
        \begin{parts} \part[1]  $ A=(-1, 1), \ B=(1, 6),\  y \ C=(3, -3) $  \begin{solution}  $ 14 $  \end{solution}
        \end{parts}
        \end{multicols}
        
    \end{questions}
    \end{document}
    