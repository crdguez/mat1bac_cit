
        \documentclass[addpoints,spanish, 12pt,a4paper]{exam}
        %\documentclass[answers, spanish, 12pt,a4paper]{exam}
        
        \pointpoints{punto}{puntos}
        \hpword{Puntos:}
        \vpword{Puntos:}
        \htword{Total}
        \vtword{Total}
        \hsword{Resultado:}
        \hqword{Ejercicio:}
        \vqword{Ejercicio:}

        \usepackage[utf8]{inputenc}
        \usepackage[spanish]{babel}
        \usepackage{eurosym}
        %\usepackage[spanish,es-lcroman, es-tabla, es-noshorthands]{babel}


        \usepackage[margin=1in]{geometry}
        \usepackage{amsmath,amssymb}
        \usepackage{multicol, xparse}

        \usepackage{yhmath}

        \usepackage{verbatim}
        %\usepackage{pstricks}


        \usepackage{graphicx}
        \graphicspath{{../img/}}




        \let\multicolmulticols\multicols
        \let\endmulticolmulticols\endmulticols
        \RenewDocumentEnvironment{multicols}{mO{}}
         {%
          \ifnum#1=1
            #2%
          \else % More than 1 column
            \multicolmulticols{#1}[#2]
          \fi
         }
         {%
          \ifnum#1=1
          \else % More than 1 column
            \endmulticolmulticols
          \fi
         }
        \renewcommand{\solutiontitle}{\noindent\textbf{Sol:}\enspace}

        \newcommand{\samedir}{\mathbin{\!/\mkern-5mu/\!}}

        \newcommand{\class}{1º Bachillerato}
        \newcommand{\examdate}{\today}

        %\newcommand{\tipo}{A}


        \newcommand{\timelimit}{50 minutos}

        \renewcommand{\solutiontitle}{\noindent\textbf{Solución:}\enspace}


        \pagestyle{head}
        \firstpageheader{\includegraphics[width=0.2\columnwidth]{header_left}}{\textbf{Departamento de Matemáticas\linebreak \class}\linebreak \examnum}{\includegraphics[width=0.1\columnwidth]{header_right}}
        \runningheader{\class}{\examnum}{Página \thepage\ of \numpages}
        \runningheadrule
        
        \pointsinrightmargin % Para poner las puntuaciones a la derecha. Se puede cambiar. Si se comenta, sale a la izquierda.
        \extrawidth{-2.4cm} %Un poquito más de margen por si ponemos textos largos.
        \marginpointname{ \emph{\points}}

        %\printanswers
            \newcommand{\tipo}{C}\newcommand{\examnum}{Autoevaluación12}
        \begin{document}
        \noindent
        \begin{tabular*}{\textwidth}{l @{\extracolsep{\fill}} r @{\extracolsep{6pt}} }
        \textbf{Nombre:} \makebox[3.5in]{\hrulefill} & \textbf{Fecha:}\makebox[1in]{\hrulefill} \\
         & \\
        \textbf{Tiempo: \timelimit} & Tipo: \tipo 
        \end{tabular*}
        \rule[2ex]{\textwidth}{2pt}
        Esta prueba tiene \numquestions\ ejercicios. La puntuación máxima es de \numpoints. 
        La nota final de la prueba será la parte proporcional de la puntuación obtenida sobre la puntuación máxima. 

        \begin{center}


        \addpoints
             %\gradetable[h][questions]
            \pointtable[h][questions]
        \end{center}

        \noindent
        \rule[2ex]{\textwidth}{2pt}

        \begin{questions}
        \question Resolver las siguientes inecuaciones:
        \begin{multicols}{1} 
        \begin{parts} \part[1]  $ \left|{3 x + 12}\right| - 4< 0 $  \begin{solution}  $ - \frac{16}{3} < x \wedge x < - \frac{8}{3} $  \end{solution} \part[1]  $ \left|{2 x - 4}\right| - 2> 0 $  \begin{solution}  $ \left(-\infty < x \wedge x < 1\right) \vee \left(3 < x \wedge x < \infty\right) $  \end{solution}
        \end{parts}
        \end{multicols}
        \question Efectúa simplificando el resultado si es posible:
        \begin{multicols}{1} 
        \begin{parts} \part[1]  $ \frac{{\frac{{x - 1}}{{x + 2}} - \frac{{x + 2}}{{x - 1}}}}{{1 - \frac{1}{{x - 1}}}} $  \begin{solution}  $ - \frac{6 x + 3}{x^{2} - 4} $  \end{solution} \part[1]  $ \frac{\frac{x^2+2x+1}{x - 3}}{\frac{x+1}{x^2 -9 }} $  \begin{solution}  $ x^{2} + 4 x + 3 $  \end{solution} \part[1]  $ ( {\frac{1}{x} - \frac{1}{{x + 1}}})( {x - \frac{{x + 1}}{{x - 1}}}) $  \begin{solution}  $ \frac{x^{2} - 2 x - 1}{x^{3} - x} $  \end{solution}
        \end{parts}
        \end{multicols}
        \question Resuelve los sistemas:
        \begin{multicols}{1} 
        \begin{parts} \part[1]  $ \left\{\begin{matrix}\frac{x}{2} + \frac{y}{3} + \frac{z}{3} = -2\\ \frac{x}{3} - \frac{y}{2} + \frac{z}{3} = 2\\ \frac{x}{6} + \frac{y}{2} + \frac{z}{2} = 1\\ \end{matrix}\right. $  \begin{solution}  $ \left[\begin{matrix}\frac{1}{2} & \frac{1}{3} & \frac{1}{3} & -2\\0 & - \frac{13}{18} & \frac{1}{9} & \frac{10}{3}\\0 & 0 & \frac{35}{78} & \frac{45}{13}\end{matrix}\right] \rightarrow  \\ \left \{ x : - \frac{48}{7}, \quad y : - \frac{24}{7}, \quad z : \frac{54}{7}\right \} $  \end{solution} \part[1]  $ \left\{\begin{matrix}x - y + z = 5\\ \frac{{x - 1}}{2} + \frac{y}{3} = 1\\ \frac{{2x + y}}{2} - \frac{{3z + y}}{3} = 4\\ \end{matrix}\right. $  \begin{solution}  $ \left[\begin{matrix}1 & -1 & 1 & 5\\0 & \frac{5}{6} & - \frac{1}{2} & -1\\0 & 0 & - \frac{13}{10} & \frac{2}{5}\end{matrix}\right] \rightarrow  \\ \left \{ x : \frac{51}{13}, \quad y : - \frac{18}{13}, \quad z : - \frac{4}{13}\right \} $  \end{solution} \part[1]  $ \left\{\begin{matrix}2x-y+2z=1\\ x+y-z = 3\\ 3x+2y+z =  5\\ \end{matrix}\right. $  \begin{solution}  $ \left[\begin{matrix}2 & -1 & 2 & 1\\0 & \frac{3}{2} & -2 & \frac{5}{2}\\0 & 0 & \frac{8}{3} & - \frac{7}{3}\end{matrix}\right] \rightarrow  \\ \left \{ x : \frac{13}{8}, \quad y : \frac{1}{2}, \quad z : - \frac{7}{8}\right \} $  \end{solution} \part[1]  $ \left\{\begin{matrix}x+y+z=1\\ x + 2y - z = 2\\ 2x +3y = 3\\ \end{matrix}\right. $  \begin{solution}  $ \left[\begin{matrix}1 & 1 & 1 & 1\\0 & 1 & -2 & 1\\0 & 0 & 0 & 0\end{matrix}\right] \rightarrow  \\ \left \{ x : - 3 z, \quad y : 2 z + 1\right \} $  \end{solution}
        \end{parts}
        \end{multicols}
        \question Resuelve los siguientes sistemas de inecuaciones:
        \begin{multicols}{1} 
        \begin{parts} \part[1]  $ \left\{\begin{matrix}{( {x - 1} )^2} - {( {x + 3} )^2} \leq 0\\x - 3( {x - 1}^2 ) \leq 3 \end{matrix}\right. $  \begin{solution}  $ \left[-1, \infty\right) $  \end{solution}
        \end{parts}
        \end{multicols}
        
    \end{questions}
    \end{document}
    