
        \documentclass[addpoints,spanish, 12pt,a4paper]{exam}
        %\documentclass[answers, spanish, 12pt,a4paper]{exam}
        %\printanswers
        \pointpoints{punto}{puntos}
        \hpword{Puntos:}
        \vpword{Puntos:}
        \htword{Total}
        \vtword{Total}
        \hsword{Resultado:}
        \hqword{Ejercicio:}
        \vqword{Ejercicio:}

        \usepackage[utf8]{inputenc}
        \usepackage[spanish]{babel}
        \usepackage{eurosym}
        %\usepackage[spanish,es-lcroman, es-tabla, es-noshorthands]{babel}


        \usepackage[margin=1in]{geometry}
        \usepackage{amsmath,amssymb}
        \usepackage{multicol, xparse}

        \usepackage{yhmath}

        \usepackage{verbatim}
        %\usepackage{pstricks}


        \usepackage{graphicx}
        \graphicspath{{../img/}}




        \let\multicolmulticols\multicols
        \let\endmulticolmulticols\endmulticols
        \RenewDocumentEnvironment{multicols}{mO{}}
         {%
          \ifnum#1=1
            #2%
          \else % More than 1 column
            \multicolmulticols{#1}[#2]
          \fi
         }
         {%
          \ifnum#1=1
          \else % More than 1 column
            \endmulticolmulticols
          \fi
         }
        \renewcommand{\solutiontitle}{\noindent\textbf{Sol:}\enspace}

        \newcommand{\samedir}{\mathbin{\!/\mkern-5mu/\!}}

        \newcommand{\class}{1º Bachillerato}
        \newcommand{\examdate}{\today}

        \newcommand{\tipo}{A}


        \newcommand{\timelimit}{50 minutos}

        \renewcommand{\solutiontitle}{\noindent\textbf{Solución:}\enspace}


        \pagestyle{head}
        \firstpageheader{\includegraphics[width=0.2\columnwidth]{header_left}}{\textbf{Departamento de Matemáticas\linebreak \class}\linebreak \examnum}{\includegraphics[width=0.1\columnwidth]{header_right}}
        \runningheader{\class}{\examnum}{Página \thepage\ of \numpages}
        \runningheadrule
        
        \pointsinrightmargin % Para poner las puntuaciones a la derecha. Se puede cambiar. Si se comenta, sale a la izquierda.
        \extrawidth{-2.4cm} %Un poquito más de margen por si ponemos textos largos.
        \marginpointname{ \emph{\points}}

        \newcommand{\examnum}{Autoevaluación}
        \begin{document}
        \noindent
        \begin{tabular*}{\textwidth}{l @{\extracolsep{\fill}} r @{\extracolsep{6pt}} }
        \textbf{Nombre:} \makebox[3.5in]{\hrulefill} & \textbf{Fecha:}\makebox[1in]{\hrulefill} \\
         & \\
        \textbf{Tiempo: \timelimit} & Tipo: \tipo 
        \end{tabular*}
        \rule[2ex]{\textwidth}{2pt}
        Esta prueba tiene \numquestions\ ejercicios. La puntuación máxima es de \numpoints. 
        La nota final de la prueba será la parte proporcional de la puntuación obtenida sobre la puntuación máxima. 

        \begin{center}


        \addpoints
             %\gradetable[h][questions]
            \pointtable[h][questions]
        \end{center}

        \noindent
        \rule[2ex]{\textwidth}{2pt}

        \begin{questions}
        \question Halla, para cada uno  de los siguientes polinomios, sus raíces:
        \begin{multicols}{1} 
        \begin{parts} \part[1]  $ x^2 - 1 $  \begin{solution}  $ \left\{-1, 1\right\} $  \end{solution}
        \end{parts}
        \end{multicols}
        \question Aplica el teorema del resto para calcular el resto de las siguientes divisiones:
        \begin{multicols}{1} 
        \begin{parts} \part[1]  $ ( {x^2 - 1} ):( {x + 1}) $  \begin{solution}  $ 0 $  \end{solution} \part[1]  $ ( {7{x^3} - 4x + 9} ):( {x+1} ) $  \begin{solution}  $ 6 $  \end{solution}
        \end{parts}
        \end{multicols}
        \question Descomponer en factores
        \begin{multicols}{1} 
        \begin{parts} \part[1]  $ x^3+x^2-6x $  \begin{solution}  $ x \left(x - 2\right) \left(x + 3\right) $  \end{solution} \part[1]  $ x^3-2x^2-5x+6 $  \begin{solution}  $ \left(x - 3\right) \left(x - 1\right) \left(x + 2\right) $  \end{solution}
        \end{parts}
        \end{multicols}
        \question Calcula los siguientes radicales:
        \begin{multicols}{1} 
        \begin{parts} \part[1]  $ \sqrt {1225} $  \begin{solution}  $ 35 $  \end{solution}
        \end{parts}
        \end{multicols}
        \question Resuelve las siguientes ecuaciones:
        \begin{multicols}{1} 
        \begin{parts} \part[1]  $ x^5 = -1 $  \begin{solution}  $ x^{5} = -1 $  \end{solution}
        \end{parts}
        \end{multicols}
        \question Calcula y expresa el resultado de la forma más simple:
        \begin{multicols}{1} 
        \begin{parts} \part[1]  $ \sqrt[3]{\sqrt{8}} $  \begin{solution}  $ \sqrt{2} $  \end{solution}
        \end{parts}
        \end{multicols}
        \question Calcula, descomponiendo el radicando en factores primos:
        \begin{multicols}{1} 
        \begin{parts} \part[1]  $ \sqrt[4]{50625} $  \begin{solution}  $ 15 $  \end{solution}
        \end{parts}
        \end{multicols}
        \question Efectúa:
        \begin{multicols}{1} 
        \begin{parts} \part[1]  $ \sqrt [3] {{x^2}}  \cdot \frac{{\sqrt [5] {xy} }}{{\sqrt {x{y^3}} }} $  \begin{solution}  $ \frac{\sqrt[5]{x y} \sqrt{x y^{3}} \sqrt[3]{x^{2}}}{x y^{3}} $  \end{solution} \part[1]  $ \frac{{\sqrt[3]{{x^2}{y^3}} }}{{\sqrt[3]{xy} }} $  \begin{solution}  $ \frac{\sqrt[3]{x^{2} y^{3}}}{\sqrt[3]{x y}} $  \end{solution}
        \end{parts}
        \end{multicols}
        \question Simplifica los cocientes entre factoriales:
        \begin{multicols}{1} 
        \begin{parts} \part[1]  $ \frac{{( {m + 1} )!}}{{( {m - 1} )!}} $  \begin{solution}  $ m \left(m + 1\right) $  \end{solution}
        \end{parts}
        \end{multicols}
        \question Resolver las siguientes inecuaciones:
        \begin{multicols}{1} 
        \begin{parts} \part[1]  $ \left|{3 x - 2}\right| - 0.5\leq 0 $  \begin{solution}  $ 0.5 \leq x \wedge x \leq 0.833333333333333 $  \end{solution}
        \end{parts}
        \end{multicols}
        \question Realiza los desarrollos de los siguientes binomios:
        \begin{multicols}{1} 
        \begin{parts} \part[1]  $ ( {\frac{x}{2} + \frac{2}{{{x^2}}}} )^5 $  \begin{solution}  $ \frac{x^{5}}{32} + \frac{5 x^{2}}{8} + \frac{5}{x} + \frac{20}{x^{4}} + \frac{40}{x^{7}} + \frac{32}{x^{10}} $  \end{solution}
        \end{parts}
        \end{multicols}
        \question Realiza los desarrollos de los siguientes binomios para identificar determinados términos y coeficientes:
        \begin{multicols}{1} 
        \begin{parts} \part[1]  $ ( {\frac{2}{5} + \frac{3}{x}} )^8 $  \begin{solution}  $ \frac{256}{390625} + \frac{3072}{78125 x} + \frac{16128}{15625 x^{2}} + \frac{48384}{3125 x^{3}} + \frac{18144}{125 x^{4}} + \frac{108864}{125 x^{5}} + \frac{81648}{25 x^{6}} + \frac{34992}{5 x^{7}} + \frac{6561}{x^{8}} $  \end{solution}
        \end{parts}
        \end{multicols}
        
    \end{questions}
    \end{document}
    