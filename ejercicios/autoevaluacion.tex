
        \documentclass[addpoints,spanish, 12pt,a4paper]{exam}
        %\documentclass[answers, spanish, 12pt,a4paper]{exam}
        %\printanswers
        \pointpoints{punto}{puntos}
        \hpword{Puntos:}
        \vpword{Puntos:}
        \htword{Total}
        \vtword{Total}
        \hsword{Resultado:}
        \hqword{Ejercicio:}
        \vqword{Ejercicio:}

        \usepackage[utf8]{inputenc}
        \usepackage[spanish]{babel}
        \usepackage{eurosym}
        %\usepackage[spanish,es-lcroman, es-tabla, es-noshorthands]{babel}


        \usepackage[margin=1in]{geometry}
        \usepackage{amsmath,amssymb}
        \usepackage{multicol, xparse}

        \usepackage{yhmath}

        \usepackage{verbatim}
        %\usepackage{pstricks}


        \usepackage{graphicx}
        \graphicspath{{../img/}}




        \let\multicolmulticols\multicols
        \let\endmulticolmulticols\endmulticols
        \RenewDocumentEnvironment{multicols}{mO{}}
         {%
          \ifnum#1=1
            #2%
          \else % More than 1 column
            \multicolmulticols{#1}[#2]
          \fi
         }
         {%
          \ifnum#1=1
          \else % More than 1 column
            \endmulticolmulticols
          \fi
         }
        \renewcommand{\solutiontitle}{\noindent\textbf{Sol:}\enspace}

        \newcommand{\samedir}{\mathbin{\!/\mkern-5mu/\!}}

        \newcommand{\class}{1º Bachillerato}
        \newcommand{\examdate}{\today}

        \newcommand{\tipo}{A}


        \newcommand{\timelimit}{50 minutos}

        \renewcommand{\solutiontitle}{\noindent\textbf{Solución:}\enspace}


        \pagestyle{head}
        \firstpageheader{\includegraphics[width=0.2\columnwidth]{header_left}}{\textbf{Departamento de Matemáticas\linebreak \class}\linebreak \examnum}{\includegraphics[width=0.1\columnwidth]{header_right}}
        \runningheader{\class}{\examnum}{Página \thepage\ of \numpages}
        \runningheadrule
        
        \pointsinrightmargin % Para poner las puntuaciones a la derecha. Se puede cambiar. Si se comenta, sale a la izquierda.
        \extrawidth{-2.4cm} %Un poquito más de margen por si ponemos textos largos.
        \marginpointname{ \emph{\points}}

        \newcommand{\examnum}{AutoevaluaciónC}
        \begin{document}
        \noindent
        \begin{tabular*}{\textwidth}{l @{\extracolsep{\fill}} r @{\extracolsep{6pt}} }
        \textbf{Nombre:} \makebox[3.5in]{\hrulefill} & \textbf{Fecha:}\makebox[1in]{\hrulefill} \\
         & \\
        \textbf{Tiempo: \timelimit} & Tipo: \tipo 
        \end{tabular*}
        \rule[2ex]{\textwidth}{2pt}
        Esta prueba tiene \numquestions\ ejercicios. La puntuación máxima es de \numpoints. 
        La nota final de la prueba será la parte proporcional de la puntuación obtenida sobre la puntuación máxima. 

        \begin{center}


        \addpoints
             %\gradetable[h][questions]
            \pointtable[h][questions]
        \end{center}

        \noindent
        \rule[2ex]{\textwidth}{2pt}

        \begin{questions}
        \question Descomponer en factores
        \begin{multicols}{1} 
        \begin{parts} \part[1]  $ x^2-81 $  \begin{solution}  $ \left(x - 9\right) \left(x + 9\right) $  \end{solution}
        \end{parts}
        \end{multicols}
        \question Calcula :
        \begin{multicols}{1} 
        \begin{parts} \part[1]  $ {( {\frac{2}{3}} )^{ - 2}} \cdot {( {\frac{3}{2}} )^4} $  \begin{solution}  $ \frac{729}{64} $  \end{solution} \part[1]  $ \frac{{{2^{ - 2}} \cdot {{( {{2^2}} )}^3}}}{{{2^{ - 3}}}} $  \begin{solution}  $ 128 $  \end{solution}
        \end{parts}
        \end{multicols}
        \question Calcula y expresa el resultado como potencia de exponente racional:
        \begin{multicols}{1} 
        \begin{parts} \part[1]  $ \frac{{\sqrt[3]{a} 3}}{{\sqrt{a}}} $  \begin{solution}  $ \frac{3}{\sqrt[6]{a}} $  \end{solution}
        \end{parts}
        \end{multicols}
        \question Racionaliza:
        \begin{multicols}{1} 
        \begin{parts} \part[1]  $ \frac{5}{{\sqrt{2} }} $  \begin{solution}  $ \frac{5 \sqrt{2}}{2} $  \end{solution}
        \end{parts}
        \end{multicols}
        \question Calcula y simplifica:
        \begin{multicols}{1} 
        \begin{parts} \part[1]  $ \frac{{\sqrt[3]{5} \cdot \sqrt{3}}}{{\sqrt {15}  \cdot \sqrt {6} }} $  \begin{solution}  $ \frac{5^{\frac{5}{6}} \sqrt{6}}{30} $  \end{solution}
        \end{parts}
        \end{multicols}
        \question Efectúa:
        \begin{multicols}{1} 
        \begin{parts} \part[1]  $ \sqrt[3] {5} \cdot \sqrt[3]{5^2} $  \begin{solution}  $ 5 $  \end{solution} \part[1]  $ \sqrt {n\sqrt [5] {n\sqrt [6]{n} } } $  \begin{solution}  $ \sqrt{n \sqrt[5]{n^{\frac{7}{6}}}} $  \end{solution}
        \end{parts}
        \end{multicols}
        \question Racionaliza:
        \begin{multicols}{1} 
        \begin{parts} \part[1]  $ \frac{5}{{2 + \sqrt {3} }} $  \begin{solution}  $ - 5 \sqrt{3} + 10 $  \end{solution}
        \end{parts}
        \end{multicols}
        \question Resolver las siguientes inecuaciones:
        \begin{multicols}{1} 
        \begin{parts} \part[1]  $ \left|{3 x - 2}\right| - 0.5\leq 0 $  \begin{solution}  $ 0.5 \leq x \wedge x \leq 0.833333333333333 $  \end{solution} \part[1]  $ \left|{2 x + 3}\right| - 4< 0 $  \begin{solution}  $ - \frac{7}{2} < x \wedge x < \frac{1}{2} $  \end{solution} \part[1]  $ \left|{x - 2}\right| - 1< 0 $  \begin{solution}  $ 1 < x \wedge x < 3 $  \end{solution}
        \end{parts}
        \end{multicols}
        \question Realiza los desarrollos de los siguientes binomios:
        \begin{multicols}{1} 
        \begin{parts} \part[1]  $ ( {\frac{x}{2} + \frac{2}{{{x^2}}}} )^5 $  \begin{solution}  $ \frac{x^{5}}{32} + \frac{5 x^{2}}{8} + \frac{5}{x} + \frac{20}{x^{4}} + \frac{40}{x^{7}} + \frac{32}{x^{10}} $  \end{solution} \part[1]  $ ( {1 + 2\sqrt{ 2} })^3 $  \begin{solution}  $ 25 + 22 \sqrt{2} $  \end{solution} \part[1]  $ ( {\frac{2}{\sqrt {2} } + \sqrt {2} } )^4 $  \begin{solution}  $ 16 \frac{\sqrt{2}}{2} \sqrt{2} + 48 $  \end{solution}
        \end{parts}
        \end{multicols}
        
    \end{questions}
    \end{document}
    