
        \documentclass[addpoints,spanish, 12pt,a4paper]{exam}
        %\documentclass[answers, spanish, 12pt,a4paper]{exam}
        %\printanswers
        \pointpoints{punto}{puntos}
        \hpword{Puntos:}
        \vpword{Puntos:}
        \htword{Total}
        \vtword{Total}
        \hsword{Resultado:}
        \hqword{Ejercicio:}
        \vqword{Ejercicio:}

        \usepackage[utf8]{inputenc}
        \usepackage[spanish]{babel}
        \usepackage{eurosym}
        %\usepackage[spanish,es-lcroman, es-tabla, es-noshorthands]{babel}


        \usepackage[margin=1in]{geometry}
        \usepackage{amsmath,amssymb}
        \usepackage{multicol, xparse}

        \usepackage{yhmath}

        \usepackage{verbatim}
        %\usepackage{pstricks}


        \usepackage{graphicx}
        \graphicspath{{../img/}}




        \let\multicolmulticols\multicols
        \let\endmulticolmulticols\endmulticols
        \RenewDocumentEnvironment{multicols}{mO{}}
         {%
          \ifnum#1=1
            #2%
          \else % More than 1 column
            \multicolmulticols{#1}[#2]
          \fi
         }
         {%
          \ifnum#1=1
          \else % More than 1 column
            \endmulticolmulticols
          \fi
         }
        \renewcommand{\solutiontitle}{\noindent\textbf{Sol:}\enspace}

        \newcommand{\samedir}{\mathbin{\!/\mkern-5mu/\!}}

        \newcommand{\class}{1º Bachillerato}
        \newcommand{\examdate}{\today}

        \newcommand{\tipo}{A}


        \newcommand{\timelimit}{50 minutos}

        \renewcommand{\solutiontitle}{\noindent\textbf{Solución:}\enspace}


        \pagestyle{head}
        \firstpageheader{\includegraphics[width=0.2\columnwidth]{header_left}}{\textbf{Departamento de Matemáticas\linebreak \class}\linebreak \examnum}{\includegraphics[width=0.1\columnwidth]{header_right}}
        \runningheader{\class}{\examnum}{Página \thepage\ of \numpages}
        \runningheadrule

        \newcommand{\examnum}{Autoevaluación}
        \begin{document}
        \noindent
        \begin{tabular*}{\textwidth}{l @{\extracolsep{\fill}} r @{\extracolsep{6pt}} }
        \textbf{Nombre:} \makebox[3.5in]{\hrulefill} & \textbf{Fecha:}\makebox[1in]{\hrulefill} \\
         & \\
        \textbf{Tiempo: \timelimit} & Tipo: \tipo 
        \end{tabular*}
        \rule[2ex]{\textwidth}{2pt}
        Esta prueba tiene \numquestions\ ejercicios. La puntuación máxima es de \numpoints. 
        La nota final de la prueba será la parte proporcional de la puntuación obtenida sobre la puntuación máxima. Para la recuperación de pendientes de 3º se tendrán en cuenta los apartados: 1.a y 4.a

        \begin{center}


        \addpoints
             %\gradetable[h][questions]
            \pointtable[h][questions]
        \end{center}

        \noindent
        \rule[2ex]{\textwidth}{2pt}

        \begin{questions}
        \question p012e09 - Halla, para cada uno  de los siguientes polinomios, sus raíces:
        \begin{multicols}{3.0} 
        \begin{parts} \part[]  $ 3x^2 -12 $  \begin{solution}  $ \left\{-2, 2\right\} $  \end{solution}
        \end{parts}
        \end{multicols}
        \question p4e1-2 - Calcula :
        \begin{multicols}{nan} 
        \begin{parts} \part[]  $ \frac{{{9^{\frac{1}{2}}} \cdot {3^{ - 1}} \cdot {2^{\frac{3}{2}}}}}{{\sqrt {2} }} $  \begin{solution}  $ 2 $  \end{solution} \part[]  $ {( {\frac{2}{3}} )^{ - 2}} \cdot {( {\frac{3}{2}} )^4} $  \begin{solution}  $ \frac{729}{64} $  \end{solution}
        \end{parts}
        \end{multicols}
        \question p6e13 - Calcula los siguientes radicales:
        \begin{multicols}{nan} 
        \begin{parts} \part[]  $ \sqrt[5]{-1} $  \begin{solution}  $ \sqrt[5]{-1} $  \end{solution}
        \end{parts}
        \end{multicols}
        \question p6e15 - Calcula y expresa el resultado de la forma más simple:
        \begin{multicols}{nan} 
        \begin{parts} \part[]  $ \frac{{\sqrt[3] {625} }}{{\sqrt[3] {5}  }} $  \begin{solution}  $ 5 $  \end{solution}
        \end{parts}
        \end{multicols}
        \question p6e19 - Racionaliza:
        \begin{multicols}{nan} 
        \begin{parts} \part[]  $ \frac{5}{{2 - \sqrt {6} }} $  \begin{solution}  $ \frac{- 5 \sqrt{6} - 10}{2} $  \end{solution}
        \end{parts}
        \end{multicols}
        \question p6e23 - Calcula, descomponiendo el radicando en factores primos:
        \begin{multicols}{nan} 
        \begin{parts} \part[]  $ 51    \sqrt[5]{59049}
51    \sqrt[5]{59049}
Name: enunciado_latex, dtype: object $  \begin{solution}  $ 51    9
51    9
Name: solucion, dtype: object $  \end{solution} \part[]  $ 51    \sqrt[5]{59049}
51    \sqrt[5]{59049}
Name: enunciado_latex, dtype: object $  \begin{solution}  $ 51    9
51    9
Name: solucion, dtype: object $  \end{solution}
        \end{parts}
        \end{multicols}
        \question p6e27 - Efectúa:
        \begin{multicols}{nan} 
        \begin{parts} \part[]  $ \frac{{\sqrt[4]{{x^3}{y^3}} }}{{\sqrt[3]{xy} }} $  \begin{solution}  $ \frac{\sqrt[4]{x^{3} y^{3}}}{\sqrt[3]{x y}} $  \end{solution}
        \end{parts}
        \end{multicols}
        \question p9e3 - Resolver las siguientes inecuaciones:
        \begin{multicols}{nan} 
        \begin{parts} \part[]  $ \left|{2 x + 1}\right| - 0.5\geq 0 $  \begin{solution}  $ \left(-0.25 \leq x \wedge x < \infty\right) \vee \left(x \leq -0.75 \wedge - \infty < x\right) $  \end{solution} \part[]  $ \left|{2 x + 3}\right| - 4< 0 $  \begin{solution}  $ - \frac{7}{2} < x \wedge x < \frac{1}{2} $  \end{solution} \part[]  $ 94    \binom{4}{2} + \binom{4}{3} + \binom{5}{4}+ \b...
94    \binom{4}{2} + \binom{4}{3} + \binom{5}{4}+ \b...
Name: enunciado_latex, dtype: object $  \begin{solution}  $ 94    36
94    36
Name: solucion, dtype: object $  \end{solution} \part[]  $ 94    \binom{4}{2} + \binom{4}{3} + \binom{5}{4}+ \b...
94    \binom{4}{2} + \binom{4}{3} + \binom{5}{4}+ \b...
Name: enunciado_latex, dtype: object $  \begin{solution}  $ 94    36
94    36
Name: solucion, dtype: object $  \end{solution} \part[]  $ \binom{25}{3} + \binom{25}{4} $  \begin{solution}  $ 14950 $  \end{solution}
        \end{parts}
        \end{multicols}
        \question p9e7-13 - Realiza los desarrollos de los siguientes binomios para identificar determinados términos y coeficientes:
        \begin{multicols}{nan} 
        \begin{parts} \part[]  $ ( {{x^2} + \frac{1}{x}} )^8 $  \begin{solution}  $ x^{16} + 8 x^{13} + 28 x^{10} + 56 x^{7} + 70 x^{4} + 56 x + \frac{28}{x^{2}} + \frac{8}{x^{5}} + \frac{1}{x^{8}} $  \end{solution}
        \end{parts}
        \end{multicols}
        
    \end{questions}
    \end{document}
    