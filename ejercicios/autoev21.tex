
        \documentclass[spanish, 11pt]{exam}

        %These tell TeX which packages to use.
        \usepackage{array,epsfig}
        \usepackage{amsmath, textcomp}
        \usepackage{amsfonts}
        \usepackage{amssymb}
        \usepackage{amsxtra}
        \usepackage{amsthm}
        \usepackage{mathrsfs}
        \usepackage{color}
        \usepackage{multicol, xparse}
        \usepackage{verbatim}


        \usepackage[utf8]{inputenc}
        \usepackage[spanish]{babel}
        \usepackage{eurosym}

        \usepackage{graphicx}
        \graphicspath{{../img/}}



        \printanswers
        \nopointsinmargin
        \pointformat{}

        %Pagination stuff.
        %\setlength{\topmargin}{-.3 in}
        %\setlength{\oddsidemargin}{0in}
        %\setlength{\evensidemargin}{0in}
        %\setlength{\textheight}{9.in}
        %\setlength{\textwidth}{6.5in}
        %\pagestyle{empty}

        \let\multicolmulticols\multicols
        \let\endmulticolmulticols\endmulticols
        \RenewDocumentEnvironment{multicols}{mO{}}
         {%
          \ifnum#1=1
            #2%
          \else % More than 1 column
            \multicolmulticols{#1}[#2]
          \fi
         }
         {%
          \ifnum#1=1
          \else % More than 1 column
            \endmulticolmulticols
          \fi
         }
        \renewcommand{\solutiontitle}{\noindent\textbf{Sol:}\enspace}

        \newcommand{\samedir}{\mathbin{\!/\mkern-5mu/\!}}

        \newcommand{\class}{1º Bachillerato}
        \newcommand{\examdate}{\today}

        \newcommand{\tipo}{A}


        \newcommand{\timelimit}{50 minutos}



        \pagestyle{head}
        \firstpageheader{\includegraphics[width=0.2\columnwidth]{header_left}}{\textbf{Departamento de Matemáticas\linebreak \class}\linebreak \examnum}{\includegraphics[width=0.1\columnwidth]{header_right}}
        \runningheader{\class}{\examnum}{Página \thepage\ of \numpages}
        \runningheadrule

        \newcommand{\examnum}{Autoevaluación 1 ev2}
        \begin{document}
        \begin{questions}
        \question p21e01 - ¿Son equipolentes los vectores  $\overrightarrow {AB} $  y  $\overrightarrow {CD} $ siendo A, B, C y D los puntos de coordenadas:?
        \begin{multicols}{2} 
        \begin{parts} \part[1] A(2, 4), B(7, 3), C(-2, 0) y D(3, -1)  \begin{solution}  $ Point2D(5, -1), Point2D(5, -1): True $  \end{solution}
        \end{parts}
        \end{multicols}
        \question pa21e02 - Sea 
$\left\lbrace\overrightarrow{ i},\overrightarrow{ j}\right\rbrace$ la base canónica de $V_2$, y los vectores:
$\overrightarrow u=  - 3\overrightarrow i + \overrightarrow j$, 
$\overrightarrow v = 2\overrightarrow i - 2\overrightarrow j$, 
$\overrightarrow w = \overrightarrow i - \overrightarrow j$, 
$\overrightarrow z =  - \overrightarrow i - 4\overrightarrow j$
	Calcular:
        \begin{multicols}{1} 
        \begin{parts} \part[1] Las coordenadas de cada uno de ellos respecto de la base canónica.
	Las coordenadas de los vectores:$\overrightarrow u + 2\overrightarrow v$, 
    $5\overrightarrow u - \overrightarrow w$, 
    $ - 3\overrightarrow v + 4\overrightarrow w$, 
    $\overrightarrow w - 2\overrightarrow z$  \begin{solution}  $ [[(-3, 1), (2, -2), (1, -1), (-1, -4)], [(1, -3), (6, -16), (2, -2), (3, 7)]] $  \end{solution}
        \end{parts}
        \end{multicols}
        \question pa21e03 - Estudia la dependencia lineal de los siguientes conjuntos de vectores:
        \begin{multicols}{2} 
        \begin{parts} \part[1]  $ \overrightarrow{u}=(8, 12) \, \ \overrightarrow{v}=(2, 3) $  \begin{solution}  $ True $  \end{solution} \part[1]  $ \overrightarrow{u}=(2, 6) \, \ \overrightarrow{v}=(4, 7) $  \begin{solution}  $ False $  \end{solution}
        \end{parts}
        \end{multicols}
        \question pa21e04 - Respecto de una base ortonormal tenemos dos vectores $\overrightarrow{u}$ y $\overrightarrow{v}$.
Calcular $\overrightarrow{u}\cdot\overrightarrow{v}$, $|\overrightarrow{u}| \ y \ |\overrightarrow{v}|$ 
y $\angle(\overrightarrow{u},\overrightarrow{v})$ siendo:
        \begin{multicols}{2} 
        \begin{parts} \part[1]  $ \overrightarrow{u}=(2, -5) \, \ \overrightarrow{v}=(6, 2) $  \begin{solution}  $ \left [ 2, \quad \left [ \sqrt{29}, \quad 2 \sqrt{10}\right ], \quad 86.6335393365702\right ] $  \end{solution} \part[1]  $ \overrightarrow{u}=(1, 4) \, \ \overrightarrow{v}=(3, 8) $  \begin{solution}  $ \left [ 35, \quad \left [ \sqrt{17}, \quad \sqrt{73}\right ], \quad 6.51980175165697\right ] $  \end{solution}
        \end{parts}
        \end{multicols}
        \question pa21e05 - Calcula x para que los vectores
$\overrightarrow{u}$ y $\overrightarrow{v}$ formen 60º siendo: 
        \begin{multicols}{2} 
        \begin{parts} \part[1]  $ \overrightarrow{u}=(6, x) \, \ \overrightarrow{v}=(10, 2) $  \begin{solution}  $ \left [ \frac{60}{11} + \frac{78 \sqrt{3}}{11}, \quad - \frac{78 \sqrt{3}}{11} + \frac{60}{11}\right ] $  \end{solution}
        \end{parts}
        \end{multicols}
        \question pa21e06 - Resolver las siguientes ecuaciones para ángulos en el primer cuadrante:
        \begin{multicols}{1} 
        \begin{parts} \part[1]  $ \sin{2x}=\frac{\sqrt {3}}{2} $  \begin{solution}  $ \left [ \frac{\pi}{6}, \quad \frac{\pi}{3}\right ] $  \end{solution} \part[1]  $ \tan{\frac{x}{2}}=1 $  \begin{solution}  $ \left [ \frac{\pi}{2}\right ] $  \end{solution} \part[1]  $ \sin{(3x-\frac{\pi}{2})}=-\frac{\sqrt{2}}{2} $  \begin{solution}  $ \left [ \frac{\pi}{12}, \quad \frac{7 \pi}{12}\right ] $  \end{solution}
        \end{parts}
        \end{multicols}
        \question pa21e07 - Resolver las siguientes ecuaciones:
        \begin{multicols}{1} 
        \begin{parts} \part[1]  $ \tan{2x}=\cot{x} $  \begin{solution}  $ \left [ -90, \quad 90, \quad -150, \quad 150, \quad -30, \quad 30\right ] $  \end{solution} \part[1]  $ \sin{x}\cos{x}=\frac{1}{2} $  \begin{solution}  $ \left [ -135, \quad 45\right ] $  \end{solution} \part[1]  $ 3\sin{x}+\cos{x}=1 $  \begin{solution}  $ \left [ 0, \quad \frac{360 \operatorname{atan}{\left (3 \right )}}{\pi}\right ] $  \end{solution}
        \end{parts}
        \end{multicols}
        \question pa21e08 - Dado el siguiente número $z$, calcula el valor de $\frac{z-\overline{z}}{z+\overline{z}}$
        \begin{multicols}{2} 
        \begin{parts} \part[1]  $ \sqrt{3}-2\sqrt{2}i $  \begin{solution}  $ - \frac{2 \sqrt{6} i}{3} $  \end{solution} \part[1]  $ \sqrt{2}-2\sqrt{5}i $  \begin{solution}  $ - \sqrt{10} i $  \end{solution}
        \end{parts}
        \end{multicols}
        
    \end{questions}
    \end{document}
    