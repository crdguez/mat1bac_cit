
        \documentclass[spanish, 11pt]{exam}

        %These tell TeX which packages to use.
        \usepackage{array,epsfig}
        \usepackage{amsmath, textcomp}
        \usepackage{amsfonts}
        \usepackage{amssymb}
        \usepackage{amsxtra}
        \usepackage{amsthm}
        \usepackage{mathrsfs}
        \usepackage{color}
        \usepackage{multicol, xparse}
        \usepackage{verbatim}


        \usepackage[utf8]{inputenc}
        \usepackage[spanish]{babel}
        \usepackage{eurosym}

        \usepackage{graphicx}
        \graphicspath{{../img/}}
        \usepackage{pgf}



        \printanswers
        \nopointsinmargin
        \pointformat{}

        %Pagination stuff.
        %\setlength{\topmargin}{-.3 in}
        %\setlength{\oddsidemargin}{0in}
        %\setlength{\evensidemargin}{0in}
        %\setlength{\textheight}{9.in}
        %\setlength{\textwidth}{6.5in}
        %\pagestyle{empty}

        \let\multicolmulticols\multicols
        \let\endmulticolmulticols\endmulticols
        \RenewDocumentEnvironment{multicols}{mO{}}
         {%
          \ifnum#1=1
            #2%
          \else % More than 1 column
            \multicolmulticols{#1}[#2]
          \fi
         }
         {%
          \ifnum#1=1
          \else % More than 1 column
            \endmulticolmulticols
          \fi
         }
        \renewcommand{\solutiontitle}{\noindent\textbf{Sol:}\enspace}

        \newcommand{\samedir}{\mathbin{\!/\mkern-5mu/\!}}

        \newcommand{\class}{1º Bachillerato}
        \newcommand{\examdate}{\today}

        \newcommand{\tipo}{A}


        \newcommand{\timelimit}{50 minutos}



        \pagestyle{head}
        \firstpageheader{\includegraphics[width=0.2\columnwidth]{header_left}}{\textbf{Departamento de Matemáticas\linebreak \class}\linebreak \examnum}{\includegraphics[width=0.1\columnwidth]{header_right}}
        \runningheader{\class}{\examnum}{Página \thepage\ of \numpages}
        \runningheadrule

        \newcommand{\examnum}{Recuperación 2ª Ev.}
        \begin{document}
        \begin{questions}
        \question ex03e01b - Resuelve:
        \begin{multicols}{1}
        \begin{parts} \part[1]  $ 2\sin^2{x}+\cos{2x}=4\cos^2{x} $  \begin{solution}  $ \left [ -120, \quad 120, \quad -60, \quad 60\right ] $  \end{solution} \part[1]  $ \cos^2{2x}=\frac{1}{4}+\sin{2x} $  \begin{solution}  $ \left [ 15, \quad 75, \quad \frac{180 \left(- \frac{98174770424681}{125000000000000} - \frac{481211825059603 i}{1000000000000000}\right)}{\pi}, \quad \frac{180 \left(- \frac{98174770424681}{125000000000000} + \frac{481211825059603 i}{1000000000000000}\right)}{\pi}\right ] $  \end{solution}
        \end{parts}
        \end{multicols}
        \question ex13e01-0 - Dado el triángulo ABC de coordenadas A=(2, 1), B=(6, 5),  y C=(5, 1), hallar: 
    
        \begin{multicols}{1}
        \begin{parts} \part[1] El área del triángulo  \begin{solution}   \\La base mide: $4 \sqrt{2}$ \\La altura mide: $\frac{3 \sqrt{2}}{2}$ \\El área es: $6$   \end{solution} \part[1] El ángulo $\widehat{A}$.  \begin{solution}   El ángulo mide: $45$   \end{solution} \part[1] El punto simétrico de C respecto a AB  \begin{solution}   $r_{AB} \equiv - 4 x + 4 y + 4=0$, \\ $t_{C}\bot r_{AB} \equiv - 4 x - 4 y + 24=0$, \\  $r\bot t \equiv Point2D(7/2, 5/2)$, \\ El punto simétrico: $Point2D(2, 4)$   \end{solution}
        \end{parts}
        \end{multicols}
        \question ex13e01-1 - Dado el triángulo ABC de coordenadas A=(2, 1), B=(6, 5),  y C=(2, 4), hallar: 
    
        \begin{multicols}{1}
        \begin{parts} \part[1] El área del triángulo  \begin{solution}   \\La base mide: $4 \sqrt{2}$ \\La altura mide: $\frac{3 \sqrt{2}}{2}$ \\El área es: $6$   \end{solution} \part[1] El ángulo $\widehat{A}$.  \begin{solution}   El ángulo mide: $45$   \end{solution} \part[1] El punto simétrico de C respecto a AB  \begin{solution}   $r_{AB} \equiv - 4 x + 4 y + 4=0$, \\ $t_{C}\bot r_{AB} \equiv - 4 x - 4 y + 24=0$, \\  $r\bot t \equiv Point2D(7/2, 5/2)$, \\ El punto simétrico: $Point2D(5, 1)$   \end{solution}
        \end{parts}
        \end{multicols}
        \question Determina el valor que debe tener k ($k \in \mathbb{R}$)para que la siguiente expresión sea un número real.
        \begin{multicols}{1}
        \begin{parts} \part[1]  $ (k+2i)^2+(5+2i) $  \begin{solution}  $ k^{2} + 4 i k + 1 + 2 i\to\left [ - \frac{1}{2}\right ] $  \end{solution}
        \end{parts}
        \end{multicols}
        \question ex13e02-1 - Determina el valor que debe tener k ($k \in \mathbb{R}$)para que la siguiente expresión sea un número imaginario puro.
        \begin{multicols}{2}
        \begin{parts} \part[1]  $ (k-2i)^2+(-5+2i) $  \begin{solution}  $ k^{2} - 4 i k - 9 + 2 i\to\left [ -3, \quad 3\right ] $  \end{solution}
        \end{parts}
        \end{multicols}
        \question ex13e03 - Halla el punto de la recta que equidista de los puntos A y B, siendo:
        \begin{multicols}{1}
        \begin{parts} \part[1]  $  A(2, 2)\  , \ B=(-2, 4) \ y \  C \in r\equiv 2x+y+1=0 $  \begin{solution}  $ \left [ \left \{ x : - \frac{y}{2} - \frac{1}{2}, \quad x : \frac{y}{2} - \frac{3}{2}\right \}\right ] $  \end{solution}
        \end{parts}
        \end{multicols}
        \question ex13e04 - Escribe en forma binómica los siguientes números complejos:
        \begin{multicols}{2}
        \begin{parts} \part[1]  $ 2_{\frac{\pi}{4}} $  \begin{solution}  $ \sqrt{2} + \sqrt{2} i $  \end{solution} \part[1]  $ 3_{\frac{\pi}{6}} $  \begin{solution}  $ \frac{3 \sqrt{3}}{2} + \frac{3 i}{2} $  \end{solution} \part[1]  $ \sqrt{2}_{\pi} $  \begin{solution}  $ - \sqrt{2} $  \end{solution} \part[1]  $ 3_{- \frac{\pi}{2}} $  \begin{solution}  $ - 3 i $  \end{solution}
        \end{parts}
        \end{multicols}
        
    \end{questions}
    \end{document}
    