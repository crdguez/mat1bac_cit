
        \documentclass[spanish, 11pt]{exam}

        %These tell TeX which packages to use.
        \usepackage{array,epsfig}
        \usepackage{amsmath, textcomp}
        \usepackage{amsfonts}
        \usepackage{amssymb}
        \usepackage{amsxtra}
        \usepackage{amsthm}
        \usepackage{mathrsfs}
        \usepackage{color}
        \usepackage{multicol, xparse}
        \usepackage{verbatim}


        \usepackage[utf8]{inputenc}
        \usepackage[spanish]{babel}
        \usepackage{eurosym}

        \usepackage{graphicx}
        \graphicspath{{../img/}}
        \usepackage{pgf}



        \printanswers
        \nopointsinmargin
        \pointformat{}

        %Pagination stuff.
        %\setlength{\topmargin}{-.3 in}
        %\setlength{\oddsidemargin}{0in}
        %\setlength{\evensidemargin}{0in}
        %\setlength{\textheight}{9.in}
        %\setlength{\textwidth}{6.5in}
        %\pagestyle{empty}

        \let\multicolmulticols\multicols
        \let\endmulticolmulticols\endmulticols
        \RenewDocumentEnvironment{multicols}{mO{}}
         {%
          \ifnum#1=1
            #2%
          \else % More than 1 column
            \multicolmulticols{#1}[#2]
          \fi
         }
         {%
          \ifnum#1=1
          \else % More than 1 column
            \endmulticolmulticols
          \fi
         }
        \renewcommand{\solutiontitle}{\noindent\textbf{Sol:}\enspace}

        \newcommand{\samedir}{\mathbin{\!/\mkern-5mu/\!}}

        \newcommand{\class}{1º Bachillerato}
        \newcommand{\examdate}{\today}

        \newcommand{\tipo}{A}


        \newcommand{\timelimit}{50 minutos}



        \pagestyle{head}
        \firstpageheader{\includegraphics[width=0.2\columnwidth]{header_left}}{\textbf{Departamento de Matemáticas\linebreak \class}\linebreak \examnum}{\includegraphics[width=0.1\columnwidth]{header_right}}
        \runningheader{\class}{\examnum}{Página \thepage\ of \numpages}
        \runningheadrule

        \newcommand{\examnum}{44 - Variables aleatorias}
        \begin{document}
        \begin{questions}
        \question p105e02-0 - Sea X una variable aleatoria discreta cuya función de probabilidad es ($x_i:p_i$):\\{0: 0.1, 1: 0.2, 2: 0.1, 3: 0.4, 4: 0.1, 5: 0.1} 
    
        \begin{multicols}{1}
        \begin{parts} \part[1] Calcula sus parámetros  \begin{solution}   \\La media es: $2.5$ \\La varianza: $2.05$   \end{solution} \part[1] Calcula $P\left(X < 4.5\right)$: , $P\left( X \geqslant 3 \right)$, $P\left(3\leqslant X < 4.5 \right)$   \begin{solution}    $\left [ 0.9, \quad 0.6, \quad 0.5\right ]$   \end{solution}
        \end{parts}
        \end{multicols}
        \question p105e08-0 - En una distribución binomial B(9 , 0.2) calcula:\\ 
    
        \begin{multicols}{2}
        \begin{parts} \part[1] $P\left(X > 3\right)$  \begin{solution}    $0.085641728$   \end{solution} \part[1] $P\left(X \geq 7\right)$  \begin{solution}    $0.000313856$   \end{solution} \part[1] $P\left(X > 0\right)$  \begin{solution}    $0.865782272$   \end{solution} \part[1] $P\left(X \leq 9\right)$  \begin{solution}    $1.0$   \end{solution}
        \end{parts}
        \end{multicols}
        \question p105e09-0 - La última novela de cierto afamado autor ha tenido un importante éxito, hasta el punto de que el
80\% de los lectores ya la han leído. Un grupo de 4 amigos son aficionados a la lectura\\ 
    
        \begin{multicols}{1}
        \begin{parts} \part[1] Describe la variable que indica el número de individuos del grupo que han leído la novela  \begin{solution}   $\left \{ 0 : 0.0016, \quad 1 : 0.0256, \quad 2 : 0.1536, \quad 3 : 0.4096, \quad 4 : 0.4096\right \}$   \end{solution} \part[1] ¿Cuál es la probabilidad de que en el grupo hayan leído la obra 2 personas? ¿Y al menos 2?  \begin{solution}   $0.1536$ y $0.9728$   \end{solution}
        \end{parts}
        \end{multicols}
        \question p105e10-0 - La probabilidad de que un jugador de baloncesto enceste una canasta de 3 puntos es 0.6. Si tira 6
veces:\\ 
    
        \begin{multicols}{1}
        \begin{parts} \part[1] Describe la variable del ejercicio  \begin{solution}   $\left \{ 0 : 0.004096, \quad 1 : 0.036864, \quad 2 : 0.13824, \quad 3 : 0.27648, \quad 4 : 0.31104, \quad 5 : 0.186624, \quad 6 : 0.046656\right \}$   \end{solution} \part[1] Calcula la probabilidad de que enceste 3  \begin{solution}   $P{\left (X = 3 \right )}=0.27648$   \end{solution} \part[1] Calcula la probabilidad de que enceste al menos 1  \begin{solution}   $P{\left (X \geq 1 \right )}=0.995904$   \end{solution} \part[1] Calcula la probabilidad de que enceste más de 3  \begin{solution}   $P{\left (X > 3 \right )}=0.54432$   \end{solution}
        \end{parts}
        \end{multicols}
        \question p105e18-0 - En una estación de ferrocarril se sabe que la probabilidad de que un tren llegue a la hora es del
95\%. Un determinado día en el que llegan 20 trenes a la estación:
    
        \begin{multicols}{1}
        \begin{parts} \part[1] Calcula la probabilidad de que al menos 18 lleguen a la hora  \begin{solution}   $P{\left (X \geq 18 \right )}=0.924516326211503$   \end{solution} \part[1] ¿Y la de que como máximo 1 no llegue a la hora?  \begin{solution}   $P{\left (X \geq 19 \right )}=0.735839524943849$   \end{solution}
        \end{parts}
        \end{multicols}
        \question p106e19-0 - En una distribución Normal Z(0 , 1) calcula:\\ 
    
        \begin{multicols}{2}
        \begin{parts} \part[1] $P\left(Z \leq 1.83\right)$  \begin{solution}    $0.966375030580372$   \end{solution} \part[1] $P\left(Z \geq 0.27\right)$  \begin{solution}    $0.39358012680196$   \end{solution} \part[1] $P\left(Z \leq 0.78\right)$  \begin{solution}    $0.782304562414267$   \end{solution} \part[1] $P\left(Z \geq -2.4\right)$  \begin{solution}    $0.991802464075404$   \end{solution} \part[1] $P\left(Z = 1.6\right)$  \begin{solution}    $0$   \end{solution} \part[1] $P\left(Z \geq -2.71 \wedge Z \leq -1.83\right)$  \begin{solution}    $0.0302608090129591$   \end{solution} \part[1] $P\left(Z \geq 1.5 \wedge Z \leq 2.5\right)$  \begin{solution}    $0.0605975359430819$   \end{solution} \part[1] $P\left(Z \geq -1.87 \wedge Z \leq 1.25\right)$  \begin{solution}    $0.863608317403679$   \end{solution} \part[1] $P\left(Z \geq 1.32\right)$  \begin{solution}    $0.0934175089934718$   \end{solution} \part[1] $P\left(Z \geq -1.32\right)$  \begin{solution}    $0.906582491006528$   \end{solution} \part[1] $P\left(Z \leq -2.17\right)$  \begin{solution}    $0.0150034229737322$   \end{solution} \part[1] $P\left(Z \geq 1.52 \wedge Z \leq 2.05\right)$  \begin{solution}    $0.0440732724132314$   \end{solution} \part[1] $P\left(Z \geq -2.03 \wedge Z \leq -1.52\right)$  \begin{solution}    $0.0430772181762636$   \end{solution} \part[1] $P\left(Z \leq 0\right)$  \begin{solution}    $0.5$   \end{solution}
        \end{parts}
        \end{multicols}
        \question p106e20-0 - Calcula el valor de k en cada uno de los siguientes casos:\\ 
    
        \begin{multicols}{2}
        \begin{parts} \part[1] $P\left(Z < k\right)=0.8635$  \begin{solution}    $1.1$   \end{solution} \part[1] $P\left(Z < k\right)=0.1894$  \begin{solution}    $-0.88$   \end{solution} \part[1] $P\left(Z > k\right)=0.8635$  \begin{solution}    $-1.1$   \end{solution} \part[1] $P\left(Z > - k \wedge Z < k\right)=0.95$  \begin{solution}    $-1.96$   \end{solution}
        \end{parts}
        \end{multicols}
        \question p106e22-0 - En una distribución Normal N(5 , 2) calcula:\\ 
    
        \begin{multicols}{2}
        \begin{parts} \part[1] $P\left(X \leq 6\right)$  \begin{solution}    $0.691462461274013$   \end{solution} \part[1] $P\left(X \geq 4.5\right)$  \begin{solution}    $0.598706325682924$   \end{solution} \part[1] $P\left(X \leq 7.2\right)$  \begin{solution}    $0.864333939053617$   \end{solution} \part[1] $P\left(X \geq 3 \wedge X \leq 6\right)$  \begin{solution}    $0.532807207342556$   \end{solution}
        \end{parts}
        \end{multicols}
        \question p106e23-0 - Calcula el valor de k en cada uno de los siguientes casos:\\ 
    
        \begin{multicols}{2}
        \begin{parts} \part[1] $P\left(X \geq k\right)=0.8106$  \begin{solution}    $3.24$   \end{solution} \part[1] $P\left(X \geq k\right)=0.4801$  \begin{solution}    $5.1$   \end{solution} \part[1] $P\left(X > - k + 5 \wedge X < k + 5\right)=0.5934$  \begin{solution}    $1.66$   \end{solution}
        \end{parts}
        \end{multicols}
        \question p106e24-0 - La duración media de un lavavajillas es de 15 años, con una desviación típica igual a 0.5 años. Si la
vida útil del electrodoméstico se distribuye normalmente: 
    
        \begin{multicols}{1}
        \begin{parts} \part[1] Halla la probabilidad de que al comprar
un lavavajillas, este dure más de 16 años  \begin{solution}   $P{\left (X \geq 16 \right )}=0.0227501319481792$   \end{solution}
        \end{parts}
        \end{multicols}
        \question p106e26-0 - Las tallas de 800 recién nacidos se distribuyen normalmente con una media de 50 cm y una
desviación típica de 5: 
    
        \begin{multicols}{1}
        \begin{parts} \part[1] Calcula cuántos recién nacidos cabe esperar con tallas comprendidas entre 47
y 52 cm  \begin{solution}   $P{\left (X \geq 47 \wedge X \leq 52 \right )}=0.381168623860251$, luego $305$ recién nacidos   \end{solution}
        \end{parts}
        \end{multicols}
        \question p107e39-0 - En un examen tipo test de 200 preguntas de elección múltiple, cada pregunta tiene una
respuesta correcta y una incorrecta. Se aprueba si se contestan más de 110 respuestas correctas: 
    
        \begin{multicols}{1}
        \begin{parts} \part[1] Suponiendo que se contesta al azar, calcula la probabilidad de aprobar el examen  \begin{solution}   La media: $100.0$, la desviación: $7.07106781186548$, $P{\left (X > 110.5 \right )}=0.0687819469549518$   \end{solution}
        \end{parts}
        \end{multicols}
        \question p107e41-0 - La probabilidad de que determinadas piezas de una máquina sean defectuosas es del 6\%. En un
almacén se han recibido 2000 piezas.: 
    
        \begin{multicols}{1}
        \begin{parts} \part[1] ¿Cuántas habrá defectuosas por término medio?, ¿Cuál será la desviación típica?  \begin{solution}   La media: $120.0$, la desviación: $10.6207344378814$   \end{solution} \part[1] La probabilidad de que haya más de 5 personas que han leído más de 3 libros  \begin{solution}   La media: $9.0$, la desviación: $2.76586333718787$, $P{\left (X > 5.5 \right )}=0.897140969367886$   \end{solution} \part[1] La probabilidad de que como máximo haya 6 personas que han leído más de tres libros  \begin{solution}   $P{\left (X < 6.5 \right )}=0.183030337628131$   \end{solution}
        \end{parts}
        \end{multicols}
        
    \end{questions}
    \end{document}
    