
        \documentclass[spanish, 11pt]{exam}

        %These tell TeX which packages to use.
        \usepackage{array,epsfig}
        \usepackage{amsmath, textcomp}
        \usepackage{amsfonts}
        \usepackage{amssymb}
        \usepackage{amsxtra}
        \usepackage{amsthm}
        \usepackage{mathrsfs}
        \usepackage{color}
        \usepackage{multicol, xparse}
        \usepackage{verbatim}


        \usepackage[utf8]{inputenc}
        \usepackage[spanish]{babel}
        \usepackage{eurosym}

        \usepackage{graphicx}
        \graphicspath{{../img/}}



        \printanswers
        \nopointsinmargin
        \pointformat{}

        %Pagination stuff.
        %\setlength{\topmargin}{-.3 in}
        %\setlength{\oddsidemargin}{0in}
        %\setlength{\evensidemargin}{0in}
        %\setlength{\textheight}{9.in}
        %\setlength{\textwidth}{6.5in}
        %\pagestyle{empty}

        \let\multicolmulticols\multicols
        \let\endmulticolmulticols\endmulticols
        \RenewDocumentEnvironment{multicols}{mO{}}
         {%
          \ifnum#1=1
            #2%
          \else % More than 1 column
            \multicolmulticols{#1}[#2]
          \fi
         }
         {%
          \ifnum#1=1
          \else % More than 1 column
            \endmulticolmulticols
          \fi
         }
        \renewcommand{\solutiontitle}{\noindent\textbf{Sol:}\enspace}

        \newcommand{\samedir}{\mathbin{\!/\mkern-5mu/\!}}

        \newcommand{\class}{1º Bachillerato}
        \newcommand{\examdate}{\today}

        \newcommand{\tipo}{A}


        \newcommand{\timelimit}{50 minutos}



        \pagestyle{head}
        \firstpageheader{\includegraphics[width=0.2\columnwidth]{header_left}}{\textbf{Departamento de Matemáticas\linebreak \class}\linebreak \examnum}{\includegraphics[width=0.1\columnwidth]{header_right}}
        \runningheader{\class}{\examnum}{Página \thepage\ of \numpages}
        \runningheadrule

        \newcommand{\examnum}{Ejercicios de Estadística Unidimensional Clases}
        \begin{document}
        \begin{questions}
        \question p090e06 - La medida del tórax de una muestra de varones se distribuye: \\\begin{tabular}{rlr}
\hline
    & Duración                    &   Cantidad \\
\hline
  0 & $\left[79.5, 85.5\right)$   &          4 \\
  1 & $\left[85.5, 91.5\right)$   &          8 \\
  2 & $\left[91.5, 97.5\right)$   &         12 \\
  3 & $\left[97.5, 103.5\right)$  &         20 \\
  4 & $\left[103.5, 109.5\right)$ &          9 \\
  5 & $\left[109.5, 115.5\right)$ &          5 \\
  6 & $\left[115.5, 121.5\right)$ &          2 \\
\hline
\end{tabular}
        \begin{multicols}{1} 
        \begin{parts} \part[1] Haz una tabla de frecuencias  \begin{solution}   \begin{tabular}{rrrrrrrrrr}
\hline
    &   lim\_inf &   lim\_sup &   x\_i &   f\_i &   F\_i &       h\_i &         H\_i &   x\_if\_i &   x\^{}2\_if\_i \\
\hline
  0 &      79.5 &      85.5 &  82.5 &     4 &     4 & 0.0666667 &   0.0666667 &    330   &    27225   \\
  1 &      85.5 &      91.5 &  88.5 &     8 &    12 & 0.133333  &   0.2       &    708   &    62658   \\
  2 &      91.5 &      97.5 &  94.5 &    12 &    24 & 0.2       &   0.4       &   1134   &   107163   \\
  3 &      97.5 &     103.5 & 100.5 &    20 &    44 & 0.333333  &   0.733333  &   2010   &   202005   \\
  4 &     103.5 &     109.5 & 106.5 &     9 &    53 & 0.15      &   0.883333  &    958.5 &   102080   \\
  5 &     109.5 &     115.5 & 112.5 &     5 &    58 & 0.0833333 &   0.966667  &    562.5 &    63281.2 \\
  6 &     115.5 &     121.5 & 118.5 &     2 &    60 & 0.0333333 &   1         &    237   &    28084.5 \\
  7 &     nan   &     nan   & nan   &    60 &   nan & 1         & nan         &   5940   &   592497   \\
\hline
\end{tabular}   \end{solution} \part[1] Calcula media, la varianza, la desviación típica y el coeficiente de variación  \begin{solution}   {'media': 99.0, 'varianza': 73.95000000000073, 'desviación típica': 8.59941858499752, 'coeficiente de variación': 0.0868628139898739}   \end{solution}
        \end{parts}
        \end{multicols}
        
\question Una oficina bancaria ha tabulado las cantidades de dinero que retiran de sus cuentas 100 clientes jóvenes en
un determinado día:



\begin{tabular}{rlr}
\hline
    & Euros                  &   Clientes \\
\hline
  0 & $\left[0, 40\right)$   &         40 \\
  1 & $\left[40, 80\right)$  &         35 \\
  2 & $\left[80, 120\right)$ &         25 \\
\hline
\end{tabular}

\begin{parts}
\part[1] Realizar una tabla de frecuencias con los datos que vayas a necesitar para resolver el ejercicio

\begin{solution}


\begin{tabular}{rrrrrrrrrr}
\hline
    &   lim\_inf &   lim\_sup &   x\_i &   f\_i &   F\_i &   h\_i &    H\_i &   x\_if\_i &   x\^{}2\_if\_i \\
\hline
  0 &         0 &        40 &    20 &    40 &    40 &  0.4  &   0.4  &      800 &      16000 \\
  1 &        40 &        80 &    60 &    35 &    75 &  0.35 &   0.75 &     2100 &     126000 \\
  2 &        80 &       120 &   100 &    25 &   100 &  0.25 &   1    &     2500 &     250000 \\
  3 &       nan &       nan &   nan &   100 &   nan &  1    & nan    &     5400 &     392000 \\
\hline
\end{tabular}

\end{solution}
\part[1] Calcula la media y la varianza.
\begin{solution}
\{'media': 54.0, 'varianza': 1004.0, 'desviación típica': 31.6859590355097\}
\end{solution}
\part[1] Calcula la mediana. Ayuda:
$$P_k=L_i + \frac{k\frac{N}{100}-F_{i-1}}{f_i}\cdot C_i$$
\begin{solution}
 ${'k': 50, 'N': 100.0, 'L_i': 40.0, 'f_i': 35.0, 'F_{i-1}': 40.0, 'C_i': 40.0}$
 \\
 51.42857142857143
\end{solution}
\part[1] ¿Qué porcentaje de clientes ha retirado menos de 60\euro?
\begin{solution}
${'valor': 60, 'N': 100.0, 'L_i': 40.0, 'f_i': 35.0, 'F_{i-1}': 40.0, 'C_i': 40.0}$\\
57.5
\end{solution}

\end{parts}        
        
        \question p090e07 - En una consulta médica la distribución de pacientes por su edad ha sido, en la última semana, la siguiente:\\\begin{tabular}{rlr}
\hline
    & Duración              &   Cantidad \\
\hline
  0 & $\left[15, 23\right)$ &          3 \\
  1 & $\left[23, 31\right)$ &          4 \\
  2 & $\left[31, 39\right)$ &          5 \\
  3 & $\left[39, 47\right)$ &          8 \\
  4 & $\left[47, 55\right)$ &         10 \\
  5 & $\left[55, 63\right)$ &         12 \\
  6 & $\left[63, 71\right)$ &         15 \\
  7 & $\left[71, 79\right)$ &         12 \\
  8 & $\left[79, 87\right)$ &          6 \\
\hline
\end{tabular}
        \begin{multicols}{1} 
        \begin{parts} \part[1] Haz una tabla de frecuencias  \begin{solution}   \begin{tabular}{rrrrrrrrrr}
\hline
    &   lim\_inf &   lim\_sup &   x\_i &   f\_i &   F\_i &       h\_i &         H\_i &   x\_if\_i &   x\^{}2\_if\_i \\
\hline
  0 &        15 &        23 &    19 &     3 &     3 & 0.04      &   0.04      &       57 &       1083 \\
  1 &        23 &        31 &    27 &     4 &     7 & 0.0533333 &   0.0933333 &      108 &       2916 \\
  2 &        31 &        39 &    35 &     5 &    12 & 0.0666667 &   0.16      &      175 &       6125 \\
  3 &        39 &        47 &    43 &     8 &    20 & 0.106667  &   0.266667  &      344 &      14792 \\
  4 &        47 &        55 &    51 &    10 &    30 & 0.133333  &   0.4       &      510 &      26010 \\
  5 &        55 &        63 &    59 &    12 &    42 & 0.16      &   0.56      &      708 &      41772 \\
  6 &        63 &        71 &    67 &    15 &    57 & 0.2       &   0.76      &     1005 &      67335 \\
  7 &        71 &        79 &    75 &    12 &    69 & 0.16      &   0.92      &      900 &      67500 \\
  8 &        79 &        87 &    83 &     6 &    75 & 0.08      &   1         &      498 &      41334 \\
  9 &       nan &       nan &   nan &    75 &   nan & 1         & nan         &     4305 &     268867 \\
\hline
\end{tabular}   \end{solution} \part[1] Calcula media, la varianza, la desviación típica y el coeficiente de variación  \begin{solution}   {'media': 57.4, 'varianza': 290.13333333333367, 'desviación típica': 17.0333007175161, 'coeficiente de variación': 0.296747399259862}   \end{solution} \part[1] La edad mas frecuente de los pacientes  \begin{solution}   {'Intervalo modal': '$\\left[63.0, 71.0\\right)$', 'moda': 67.0}   \end{solution} \part[1] El percentil 47  \begin{solution}   {'k': 47, 'N': 75.0, '$L_i$': 55.0, '$f_i$': 12.0, '$F_{i-1}$': 30.0, '$C_i$': 8.0, 'percentil': 58.5}   \end{solution} \part[1] ¿Qué porcentaje de pacientes tenían una edad superior a 60 años?  \begin{solution}   {'valor': 60, 'N': 75.0, '$L_i$': 55.0, '$f_i$': 12.0, '$F_{i-1}$': 30.0, '$C_i$': 8.0, 'Porcentaje': 50.0000000000000}   \end{solution}
        \end{parts}
        \end{multicols}
        
    \end{questions}
    \end{document}
    