
        \documentclass[spanish, 11pt]{exam}

        %These tell TeX which packages to use.
        \usepackage{array,epsfig}
        \usepackage{amsmath, textcomp}
        \usepackage{amsfonts}
        \usepackage{amssymb}
        \usepackage{amsxtra}
        \usepackage{amsthm}
        \usepackage{mathrsfs}
        \usepackage{color}
        \usepackage{multicol, xparse}
        \usepackage{verbatim}


        \usepackage[utf8]{inputenc}
        \usepackage[spanish]{babel}
        \usepackage{eurosym}

        \usepackage{graphicx}
        \graphicspath{{../img/}}



        \printanswers
        \nopointsinmargin
        \pointformat{}

        %Pagination stuff.
        %\setlength{\topmargin}{-.3 in}
        %\setlength{\oddsidemargin}{0in}
        %\setlength{\evensidemargin}{0in}
        %\setlength{\textheight}{9.in}
        %\setlength{\textwidth}{6.5in}
        %\pagestyle{empty}

        \let\multicolmulticols\multicols
        \let\endmulticolmulticols\endmulticols
        \RenewDocumentEnvironment{multicols}{mO{}}
         {%
          \ifnum#1=1
            #2%
          \else % More than 1 column
            \multicolmulticols{#1}[#2]
          \fi
         }
         {%
          \ifnum#1=1
          \else % More than 1 column
            \endmulticolmulticols
          \fi
         }
        \renewcommand{\solutiontitle}{\noindent\textbf{Sol:}\enspace}

        \newcommand{\samedir}{\mathbin{\!/\mkern-5mu/\!}}

        \newcommand{\class}{1º Bachillerato}
        \newcommand{\examdate}{\today}

        \newcommand{\tipo}{A}


        \newcommand{\timelimit}{50 minutos}



        \pagestyle{head}
        \firstpageheader{\includegraphics[width=0.2\columnwidth]{header_left}}{\textbf{Departamento de Matemáticas\linebreak \class}\linebreak \examnum}{\includegraphics[width=0.1\columnwidth]{header_right}}
        \runningheader{\class}{\examnum}{Página \thepage\ of \numpages}
        \runningheadrule

        \newcommand{\examnum}{22 - Producto Escalar}
        \begin{document}
        \begin{questions}
        \question p035e01 - Sea 
$\left\lbrace\overrightarrow{ i},\overrightarrow{ j}\right\rbrace$ la base canónica de $V_2$, y los vectores:
$\overrightarrow u=  - 2\overrightarrow i + \overrightarrow j$, 
$\overrightarrow v = 2\overrightarrow i - 3\overrightarrow j$, 
$\overrightarrow w = \overrightarrow i + \overrightarrow j$, 
$\overrightarrow z =  - \overrightarrow i - 3\overrightarrow j$
	Calcular:
        \begin{multicols}{1} 
        \begin{parts} \part[1] Las coordenadas de cada uno de ellos respecto de la base canónica.
	Las coordenadas de los vectores:$\overrightarrow u + 2\overrightarrow v$, 
    $5\overrightarrow u - \overrightarrow w$, 
    $ - 3\overrightarrow v + 4\overrightarrow w$, 
    $\overrightarrow w - 2\overrightarrow z$  \begin{solution}  $ [[(-2, 1), (2, -3), (1, 1), (-1, -3)], [(2, -5), (4, -11), (13, -2), (3, 7)]] $  \end{solution}
        \end{parts}
        \end{multicols}
        \question p035e02 - Estudia la dependencia lineal de los siguientes conjuntos de vectores:
        \begin{multicols}{2} 
        \begin{parts} \part[1]  $ \overrightarrow{u}=(4, 12) \, \ \overrightarrow{v}=(2, 6) $  \begin{solution}  $ True $  \end{solution} \part[1]  $ \overrightarrow{u}=(1, 2) \, \ \overrightarrow{v}=(3, 4) $  \begin{solution}  $ False $  \end{solution} \part[1]  $ \overrightarrow{u}=(1, 1) \, \ \overrightarrow{v}=(-2, -3) $  \begin{solution}  $ False $  \end{solution}
        \end{parts}
        \end{multicols}
        \question p036e09 - Respecto de una base ortonormal tenemos dos vectores $\overrightarrow{u}$ y $\overrightarrow{v}$.
Calcular $\overrightarrow{u}\cdot\overrightarrow{v}$, $|\overrightarrow{u}| \ y \ |\overrightarrow{v}|$ 
y $\angle(\overrightarrow{u},\overrightarrow{v})$ siendo:
        \begin{multicols}{2} 
        \begin{parts} \part[1]  $ \overrightarrow{u}=(2, -3) \, \ \overrightarrow{v}=(5, 4) $  \begin{solution}  $ \left [ -2, \quad \left [ \sqrt{13}, \quad \sqrt{41}\right ], \quad 94.9697407281103\right ] $  \end{solution} \part[1]  $ \overrightarrow{u}=(1, 2) \, \ \overrightarrow{v}=(3, 4) $  \begin{solution}  $ \left [ 11, \quad \left [ \sqrt{5}, \quad 5\right ], \quad 10.304846468766\right ] $  \end{solution} \part[1]  $ \overrightarrow{u}=(1, 1) \, \ \overrightarrow{v}=(-2, -3) $  \begin{solution}  $ \left [ -5, \quad \left [ \sqrt{2}, \quad \sqrt{13}\right ], \quad 168.69006752598\right ] $  \end{solution} \part[1]  $ \overrightarrow{u}=(2, -3) \, \ \overrightarrow{v}=(5, 4) $  \begin{solution}  $ \left [ -2, \quad \left [ \sqrt{13}, \quad \sqrt{41}\right ], \quad 94.9697407281103\right ] $  \end{solution} \part[1]  $ \overrightarrow{u}=(1, 2) \, \ \overrightarrow{v}=(3, 4) $  \begin{solution}  $ \left [ 11, \quad \left [ \sqrt{5}, \quad 5\right ], \quad 10.304846468766\right ] $  \end{solution} \part[1]  $ \overrightarrow{u}=(1, 1) \, \ \overrightarrow{v}=(-2, -3) $  \begin{solution}  $ \left [ -5, \quad \left [ \sqrt{2}, \quad \sqrt{13}\right ], \quad 168.69006752598\right ] $  \end{solution}
        \end{parts}
        \end{multicols}
        \question p036e12 - Calcula x, de modo que el producto escalar
de  $\overrightarrow{u}$ y $\overrightarrow{v}$ sea igual a 7, siendo: 
        \begin{multicols}{2} 
        \begin{parts} \part[1]  $ \overrightarrow{u}=(3, -5) \, \ \overrightarrow{v}=(x, 2) $  \begin{solution}  $ \left [ \frac{17}{3}\right ] $  \end{solution} \part[1]  $ \overrightarrow{u}=(3, 1) \, \ \overrightarrow{v}=(2, x) $  \begin{solution}  $ \left [ 1\right ] $  \end{solution}
        \end{parts}
        \end{multicols}
        \question p036e13 - Dado el vector
$\overrightarrow{u}$, calcula x de modo que sea ortogonal a $\overrightarrow{v}$ siendo: 
        \begin{multicols}{2} 
        \begin{parts} \part[1]  $ \overrightarrow{u}=(-5, x) \, \ \overrightarrow{v}=(4, -2) $  \begin{solution}  $ \left [ -10\right ] $  \end{solution} \part[1]  $ \overrightarrow{u}=(2, x) \, \ \overrightarrow{v}=(3, 1) $  \begin{solution}  $ \left [ -6\right ] $  \end{solution} \part[1]  $ \overrightarrow{u}=(3, x) \, \ \overrightarrow{v}=(5, 2) $  \begin{solution}  $ \left\{\frac{120}{13} + \frac{87 \sqrt{3}}{13}, - \frac{87 \sqrt{3}}{13} + \frac{120}{13}\right\} $  \end{solution} \part[1]  $ \overrightarrow{u}=(2, x) \, \ \overrightarrow{v}=(3, 1) $  \begin{solution}  $ \left\{4 + \frac{10 \sqrt{3}}{3}, - \frac{10 \sqrt{3}}{3} + 4\right\} $  \end{solution} \part[1]  $ \overrightarrow{u}=(1, 0) \, \ \overrightarrow{v}=(1, x) $  \begin{solution}  $ \left\{- \sqrt{3}, \sqrt{3}\right\} $  \end{solution}
        \end{parts}
        \end{multicols}
        \question p036e13b - Dado el vector
$\overrightarrow{u}$, calcula x de modo que $|\overrightarrow{u}|=\sqrt{34}$ siendo: 
        \begin{multicols}{2} 
        \begin{parts} \part[1]  $ \overrightarrow{u}=(-5, x) $  \begin{solution}  $ \left [ -3, \quad 3\right ] $  \end{solution} \part[1]  $ \overrightarrow{u}=(2, x) $  \begin{solution}  $ \left [ - \sqrt{30}, \quad \sqrt{30}\right ] $  \end{solution}
        \end{parts}
        \end{multicols}
        \question p036e14 - Respecto de una base ortonormal tenemos dos vectores $\overrightarrow{u}$ y $\overrightarrow{v}$.
Calcular $\overrightarrow{u}\cdot\overrightarrow{v}$, $|\overrightarrow{u}| \ y \ |\overrightarrow{v}|$ 
y $\angle(\overrightarrow{u},\overrightarrow{v})$ siendo:
        \begin{multicols}{2} 
        \begin{parts} \part[1]  $ \overrightarrow{u}=\left ( 3, \quad 2\right ) \, \ \overrightarrow{v}=\left ( 1, \quad -5\right ) $  \begin{solution}  $ \left [ -7, \quad \left [ \sqrt{13}, \quad \sqrt{26}\right ], \quad 112.38013505196\right ] $  \end{solution} \part[1]  $ \overrightarrow{u}=\left ( 1, \quad 6\right ) \, \ \overrightarrow{v}=\left ( -0.5, \quad -3\right ) $  \begin{solution}  $ \left [ - \frac{37}{2}, \quad \left [ \sqrt{37}, \quad \frac{\sqrt{37}}{2}\right ], \quad 180.0\right ] $  \end{solution}
        \end{parts}
        \end{multicols}
        
    \end{questions}
    \end{document}
    