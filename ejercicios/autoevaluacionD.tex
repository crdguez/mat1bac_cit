
        \documentclass[addpoints,spanish, 12pt,a4paper]{exam}
        %\documentclass[answers, spanish, 12pt,a4paper]{exam}
        %\printanswers
        \pointpoints{punto}{puntos}
        \hpword{Puntos:}
        \vpword{Puntos:}
        \htword{Total}
        \vtword{Total}
        \hsword{Resultado:}
        \hqword{Ejercicio:}
        \vqword{Ejercicio:}

        \usepackage[utf8]{inputenc}
        \usepackage[spanish]{babel}
        \usepackage{eurosym}
        %\usepackage[spanish,es-lcroman, es-tabla, es-noshorthands]{babel}


        \usepackage[margin=1in]{geometry}
        \usepackage{amsmath,amssymb}
        \usepackage{multicol, xparse}

        \usepackage{yhmath}

        \usepackage{verbatim}
        %\usepackage{pstricks}


        \usepackage{graphicx}
        \graphicspath{{../img/}}




        \let\multicolmulticols\multicols
        \let\endmulticolmulticols\endmulticols
        \RenewDocumentEnvironment{multicols}{mO{}}
         {%
          \ifnum#1=1
            #2%
          \else % More than 1 column
            \multicolmulticols{#1}[#2]
          \fi
         }
         {%
          \ifnum#1=1
          \else % More than 1 column
            \endmulticolmulticols
          \fi
         }
        \renewcommand{\solutiontitle}{\noindent\textbf{Sol:}\enspace}

        \newcommand{\samedir}{\mathbin{\!/\mkern-5mu/\!}}

        \newcommand{\class}{1º Bachillerato}
        \newcommand{\examdate}{\today}

        %\newcommand{\tipo}{A}


        \newcommand{\timelimit}{50 minutos}

        \renewcommand{\solutiontitle}{\noindent\textbf{Solución:}\enspace}


        \pagestyle{head}
        \firstpageheader{\includegraphics[width=0.2\columnwidth]{header_left}}{\textbf{Departamento de Matemáticas\linebreak \class}\linebreak \examnum}{\includegraphics[width=0.1\columnwidth]{header_right}}
        \runningheader{\class}{\examnum}{Página \thepage\ of \numpages}
        \runningheadrule
        
        \pointsinrightmargin % Para poner las puntuaciones a la derecha. Se puede cambiar. Si se comenta, sale a la izquierda.
        \extrawidth{-2.4cm} %Un poquito más de margen por si ponemos textos largos.
        \marginpointname{ \emph{\points}}

        \newcommand{\tipo}{D}\newcommand{\examnum}{Autoevaluación}
        \begin{document}
        \noindent
        \begin{tabular*}{\textwidth}{l @{\extracolsep{\fill}} r @{\extracolsep{6pt}} }
        \textbf{Nombre:} \makebox[3.5in]{\hrulefill} & \textbf{Fecha:}\makebox[1in]{\hrulefill} \\
         & \\
        \textbf{Tiempo: \timelimit} & Tipo: \tipo 
        \end{tabular*}
        \rule[2ex]{\textwidth}{2pt}
        Esta prueba tiene \numquestions\ ejercicios. La puntuación máxima es de \numpoints. 
        La nota final de la prueba será la parte proporcional de la puntuación obtenida sobre la puntuación máxima. 

        \begin{center}


        \addpoints
             %\gradetable[h][questions]
            \pointtable[h][questions]
        \end{center}

        \noindent
        \rule[2ex]{\textwidth}{2pt}

        \begin{questions}
        \question Dados los polinomios $ A(x)=2{x^3} - 5{x^2} + 6, \  B(x)=- \frac{1}{2}{x^5} - {x^4} + 6x$ halla:
        \begin{multicols}{1} 
        \begin{parts} \part[1]  $ A(x) - B(x) $  \begin{solution}  $ \frac{x^{5}}{2} + x^{4} + 2 x^{3} - 5 x^{2} - 6 x + 6 $  \end{solution}
        \end{parts}
        \end{multicols}
        \question Halla el cociente y el resto de:
        \begin{multicols}{1} 
        \begin{parts} \part[1]  $ ( {x^9} - 7{x} + 1 ):( {{x^3} + x} ) $  \begin{solution}  $ \left ( x^{6} - x^{4} + x^{2} - 1, \quad - 6 x + 1\right ) $  \end{solution}
        \end{parts}
        \end{multicols}
        \question Aplica la regla de Ruffini para hallar el cociente y el resto de las siguientes divisiones:
        \begin{multicols}{1} 
        \begin{parts} \part[1]  $ ( {x^2 - 1 } ):( {x} + 1) $  \begin{solution}  $ \left ( x - 1, \quad 0\right ) $  \end{solution}
        \end{parts}
        \end{multicols}
        \question Descomponer en factores
        \begin{multicols}{1} 
        \begin{parts} \part[1]  $ x^4-2x^3+2x^2-2x+1 $  \begin{solution}  $ \left(x - 1\right)^{2} \left(x^{2} + 1\right) $  \end{solution} \part[1]  $ x^2-3x $  \begin{solution}  $ x \left(x - 3\right) $  \end{solution}
        \end{parts}
        \end{multicols}
        \question Resuelve las siguientes ecuaciones:
        \begin{multicols}{1} 
        \begin{parts} \part[1]  $ x^4 = 81 $  \begin{solution}  $ x^{4} = 81 $  \end{solution}
        \end{parts}
        \end{multicols}
        \question Calcula y expresa el resultado como potencia de exponente racional:
        \begin{multicols}{1} 
        \begin{parts} \part[1]  $ \frac{{\sqrt[3]{a} 3}}{{\sqrt{a}}} $  \begin{solution}  $ \frac{3}{\sqrt[6]{a}} $  \end{solution} \part[1]  $ \sqrt {a\sqrt[3]{a} } $  \begin{solution}  $ \sqrt{a^{\frac{4}{3}}} $  \end{solution}
        \end{parts}
        \end{multicols}
        \question Calcula y simplifica:
        \begin{multicols}{1} 
        \begin{parts} \part[1]  $ \frac{{\sqrt[3]{5} \cdot \sqrt{3}}}{{\sqrt {15}  \cdot \sqrt {6} }} $  \begin{solution}  $ \frac{5^{\frac{5}{6}} \sqrt{6}}{30} $  \end{solution}
        \end{parts}
        \end{multicols}
        \question Efectúa:
        \begin{multicols}{1} 
        \begin{parts} \part[1]  $ \frac{{6\sqrt[3] { 5} }}{{2\sqrt {10} }} $  \begin{solution}  $ \frac{3 \sqrt{2} \cdot 5^{\frac{5}{6}}}{10} $  \end{solution}
        \end{parts}
        \end{multicols}
        \question Racionaliza:
        \begin{multicols}{1} 
        \begin{parts} \part[1]  $ \frac{{abc}}{{\sqrt {ab{c^3}} }} $  \begin{solution}  $ \frac{\sqrt{a b c^{3}}}{c^{2}} $  \end{solution}
        \end{parts}
        \end{multicols}
        \question Calcula las siguientes operaciones:
        \begin{multicols}{1} 
        \begin{parts} \part[1]  $ \binom{4}{0} + \binom{4}{1} + \binom{4}{2}+\binom{4}{3} $  \begin{solution}  $ 15 $  \end{solution} \part[1]  $ \left|{2 x + 3}\right| - 4< 0 $  \begin{solution}  $ - \frac{7}{2} < x \wedge x < \frac{1}{2} $  \end{solution} \part[1]  $ \left|{2 x - 3}\right| - 1\geq 0 $  \begin{solution}  $ \left(2 \leq x \wedge x < \infty\right) \vee \left(x \leq 1 \wedge -\infty < x\right) $  \end{solution}
        \end{parts}
        \end{multicols}
        \question Realiza los desarrollos de los siguientes binomios para identificar determinados términos y coeficientes:
        \begin{multicols}{1} 
        \begin{parts} \part[1]  $ (2a^2b - 3 a^3)^7 $  \begin{solution}  $ - 2187 a^{21} + 10206 a^{20} b - 20412 a^{19} b^{2} + 22680 a^{18} b^{3} - 15120 a^{17} b^{4} + 6048 a^{16} b^{5} - 1344 a^{15} b^{6} + 128 a^{14} b^{7} $  \end{solution}
        \end{parts}
        \end{multicols}
        
    \end{questions}
    \end{document}
    