
        \documentclass[addpoints,spanish, 12pt,a4paper]{exam}
        %\documentclass[answers, spanish, 12pt,a4paper]{exam}

        \pointpoints{punto}{puntos}
        \hpword{Puntos:}
        \vpword{Puntos:}
        \htword{Total}
        \vtword{Total}
        \hsword{Resultado:}
        \hqword{Ejercicio:}
        \vqword{Ejercicio:}

        \usepackage[utf8]{inputenc}
        \usepackage[spanish]{babel}
        \usepackage{eurosym}
        %\usepackage[spanish,es-lcroman, es-tabla, es-noshorthands]{babel}


        \usepackage[margin=1in]{geometry}
        \usepackage{amsmath,amssymb}
        \usepackage{multicol, xparse}

        \usepackage{yhmath}

        \usepackage{verbatim}
        %\usepackage{pstricks}


        \usepackage{graphicx}
        \graphicspath{{../img/}}
        \usepackage{pgf}




        \let\multicolmulticols\multicols
        \let\endmulticolmulticols\endmulticols
        \RenewDocumentEnvironment{multicols}{mO{}}
         {%
          \ifnum#1=1
            #2%
          \else % More than 1 column
            \multicolmulticols{#1}[#2]
          \fi
         }
         {%
          \ifnum#1=1
          \else % More than 1 column
            \endmulticolmulticols
          \fi
         }
        \renewcommand{\solutiontitle}{\noindent\textbf{Sol:}\enspace}

        \newcommand{\samedir}{\mathbin{\!/\mkern-5mu/\!}}

        \newcommand{\class}{1º Bachillerato}
        \newcommand{\examdate}{\today}

        %\newcommand{\tipo}{A}


        \newcommand{\timelimit}{50 minutos}

        \renewcommand{\solutiontitle}{\noindent\textbf{Solución:}\enspace}


        \pagestyle{head}
        \firstpageheader{\includegraphics[width=0.2\columnwidth]{header_left}}{\textbf{Departamento de Matemáticas\linebreak \class}\linebreak \examnum}{\includegraphics[width=0.1\columnwidth]{header_right}}
        \runningheader{\class}{\examnum}{Página \thepage\ of \numpages}
        \runningheadrule

        \pointsinrightmargin % Para poner las puntuaciones a la derecha. Se puede cambiar. Si se comenta, sale a la izquierda.
        \extrawidth{-2.4cm} %Un poquito más de margen por si ponemos textos largos.
        \marginpointname{ \emph{\points}}

        \printanswers
            \newcommand{\tipo}{l}\newcommand{\examnum}{Final 1ª evaluación}
        \begin{document}
        \noindent
        \begin{tabular*}{\textwidth}{l @{\extracolsep{\fill}} r @{\extracolsep{6pt}} }
        \textbf{Nombre:} \makebox[3.5in]{\hrulefill} & \textbf{Fecha:}\makebox[1in]{\hrulefill} \\
         & \\
        \textbf{Tiempo: \timelimit} & Tipo: \tipo
        \end{tabular*}
        \rule[2ex]{\textwidth}{2pt}
        Esta prueba tiene \numquestions\ ejercicios. La puntuación máxima es de \numpoints.
        La nota final de la prueba será la parte proporcional de la puntuación obtenida sobre la puntuación máxima.

        \begin{center}


        \addpoints
             %\gradetable[h][questions]
            \pointtable[h][questions]
        \end{center}

        \noindent
        \rule[2ex]{\textwidth}{2pt}

        \begin{questions}
        \question Dados los siguientes conjuntos A, B y C, represéntalos en la recta real. A continuación, calcula $A \cup  B$ , $A \cap B$ y $(A \cup B) \cap C$ , 
y expresa los resultados en forma de Intervalos. 
Indica además, si existe, el máximo y el mínimo de cada uno de los conjuntos resultado.
        \begin{multicols}{1}
        \begin{parts} \part[1]  $ A=\left\{ x \in \mathbb{R}| 6 \leq x \wedge x < 8 \right\}, \\ B=\left(-\infty, -3\right) \cup \left(3, \infty\right)  y \\  C=\left\{ x \in \mathbb{R}| \left|{x - 3}\right|\leq12 \right\} \\ $  \begin{solution}  $ C=\left[-9, 15\right] \ \ A \cup  B = \left(-\infty, -3\right) \cup \left(3, \infty\right)  \\  A \cap B= \left[6, 8\right)   \\  (A \cup B) \cap C= \left[-9, -3\right) \cup \left(3, 15\right] $  \end{solution}
        \end{parts}
        \end{multicols}
        \question Calcular:
        \begin{multicols}{1}
        \begin{parts} \part[1]  $ {( {\sqrt {7}  + \sqrt {3} } )^2}\cdot( {5 - \sqrt {21} }) $  \begin{solution}  $ 8 $  \end{solution} \part[1]  $ \frac{{3\sqrt{ 2 } - 2\sqrt {3} }}{{3\sqrt {2}  + 2\sqrt {3} }} - \frac{3}{{2\sqrt {6 }}} $  \begin{solution}  $ \frac{- 7 \sqrt{3} + 3 \sqrt{2}}{2 \left(2 \sqrt{3} + 3 \sqrt{2}\right)} $  \end{solution}
        \end{parts}
        \end{multicols}
        \question Resuelve mediante expresiones algebraicas:
        \begin{multicols}{1}
        \begin{parts} \part[1] Halla tres números naturales e impares consecutivos sabiendo que su producto menos su suma vale 6.  \begin{solution}  $ 6 x - \left(2 x - 1\right) \left(2 x + 1\right) \left(2 x + 3\right) + 9 = 0\to \left\{1\right\} $  \end{solution}
        \end{parts}
        \end{multicols}
        \question Resuelve:
        \begin{multicols}{1}
        \begin{parts} \part[1]  $ \sqrt {x + 5}  - \sqrt {x - 1}  = 2 $  \begin{solution}  $ \left [ \frac{5}{4}\right ] $  \end{solution} \part[1]  $ \frac{{7 - x}}{{x + 4}} - \frac{3}{{x - 5}} = \frac{{26x - 25}}{{{x^2} - x - 20}} + \frac{1}{3} $  \begin{solution}  $ \left [ - \frac{23}{2}, \quad -1\right ] $  \end{solution}
        \end{parts}
        \end{multicols}
        \question Resolver :
        \begin{multicols}{1}
        \begin{parts} \part[1]  $ \left\{\begin{matrix}{2^x} + {2^y} = 24\\{2^x} \cdot {2^y} = 128\end{matrix}\right. $  \begin{solution}  $ \left [ \left \{ x : 3, \quad y : 4\right \}, \quad \left \{ x : 4, \quad y : 3\right \}\right ] $  \end{solution}
        \end{parts}
        \end{multicols}
        \question Resolver :
        \begin{multicols}{1}
        \begin{parts} \part[1]  $ 2 \log x - \log (x + 6) = 3 \log 2 $  \begin{solution}  $ \left [ 12\right ] $  \end{solution}
        \end{parts}
        \end{multicols}
        \question Discute el tipo de sistema y resuelve si es posible:
        \begin{multicols}{1}
        \begin{parts} \part[1]  $ \left\{\begin{matrix}2x - y + z = 6\\ 2x + 2y - 4z = 2\\ x - 2y + 3z = 0\\ \end{matrix}\right. $  \begin{solution}  $ \left[\begin{matrix}-1 & 2 & 1 & 6\\0 & 6 & -2 & 14\\0 & 0 & 0 & -5\end{matrix}\right] \rightarrow  \\ \left [ \right ] $  \end{solution} \part[1]  $ \left\{\begin{matrix}x + 2y - 3z = 9\\ 4x - 2y = 12\\ 4x + 3y - 6z = 24\\ \end{matrix}\right. $  \begin{solution}  $ \left[\begin{matrix}2 & 1 & -3 & 9\\0 & 5 & -3 & 21\\0 & 0 & 0 & 0\end{matrix}\right] \rightarrow  \\ \left \{ x : \frac{3 z}{5} + \frac{21}{5}, \quad y : \frac{6 z}{5} + \frac{12}{5}\right \} $  \end{solution}
        \end{parts}
        \end{multicols}
        \question Usando la definición y las propiedades de los números combinatorios, resolver las ecuaciones:
        \begin{multicols}{1}
        \begin{parts} \part[1]  $ {\binom{17}{x}} = \binom{17}{x+1} $  \begin{solution}  $ \left\{8\right\} $  \end{solution}
        \end{parts}
        \end{multicols}
        \question Calcula el valor de $m$ para que:
        \begin{multicols}{1}
        \begin{parts} \part[1]  $ P(x)=9x^2-mx+\frac{1}{4} \ no \ tenga\ ninguna \ ra \acute \imath z \ real $  \begin{solution}  $ \left [ 9, \quad - m, \quad \frac{1}{4}\right ]\to -3 < m \wedge m < 3 $  \end{solution}
        \end{parts}
        \end{multicols}
        \question Resuelve:
        \begin{multicols}{1}
        \begin{parts} \part[1]  $ \frac{{3x - 2}}{{x - 1}} - \frac{{3x + 2}}{{x + 1}} \geq \frac{{2x - 1}}{{x^2 - 1}} $  \begin{solution}  $ \left(-\infty, -1\right) \cup \left(1, \infty\right) $  \end{solution} \part[1]  $ \frac{{{x^3} - 5{x^2} + 2x + 8}}{x^2+1} < 0 $  \begin{solution}  $ \left(-\infty, -1\right) \cup \left(2, 4\right) $  \end{solution}
        \end{parts}
        \end{multicols}
        \question Calcula expresando el resultado en forma de fracción algebraica irreducible:
        \begin{multicols}{1}
        \begin{parts} \part[1]  $ \frac{2+{ \frac{1}{x}}}{{2 + \frac{1}{{1 + \frac{1}{x}}}}} $  \begin{solution}  $ \frac{2 x^{2} + 3 x + 1}{3 x^{2} + 2 x} $  \end{solution}
        \end{parts}
        \end{multicols}
        
    \end{questions}
    \end{document}
    