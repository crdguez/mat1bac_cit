
        \documentclass[addpoints,spanish, 12pt,a4paper]{exam}
        %\documentclass[answers, spanish, 12pt,a4paper]{exam}
        
        \pointpoints{punto}{puntos}
        \hpword{Puntos:}
        \vpword{Puntos:}
        \htword{Total}
        \vtword{Total}
        \hsword{Resultado:}
        \hqword{Ejercicio:}
        \vqword{Ejercicio:}

        \usepackage[utf8]{inputenc}
        \usepackage[spanish]{babel}
        \usepackage{eurosym}
        %\usepackage[spanish,es-lcroman, es-tabla, es-noshorthands]{babel}


        \usepackage[margin=1in]{geometry}
        \usepackage{amsmath,amssymb}
        \usepackage{multicol, xparse}

        \usepackage{yhmath}

        \usepackage{verbatim}
        %\usepackage{pstricks}


        \usepackage{graphicx}
        \graphicspath{{../img/}}




        \let\multicolmulticols\multicols
        \let\endmulticolmulticols\endmulticols
        \RenewDocumentEnvironment{multicols}{mO{}}
         {%
          \ifnum#1=1
            #2%
          \else % More than 1 column
            \multicolmulticols{#1}[#2]
          \fi
         }
         {%
          \ifnum#1=1
          \else % More than 1 column
            \endmulticolmulticols
          \fi
         }
        \renewcommand{\solutiontitle}{\noindent\textbf{Sol:}\enspace}

        \newcommand{\samedir}{\mathbin{\!/\mkern-5mu/\!}}

        \newcommand{\class}{1º Bachillerato}
        \newcommand{\examdate}{\today}

        %\newcommand{\tipo}{A}


        \newcommand{\timelimit}{50 minutos}

        \renewcommand{\solutiontitle}{\noindent\textbf{Solución:}\enspace}


        \pagestyle{head}
        \firstpageheader{\includegraphics[width=0.2\columnwidth]{header_left}}{\textbf{Departamento de Matemáticas\linebreak \class}\linebreak \examnum}{\includegraphics[width=0.1\columnwidth]{header_right}}
        \runningheader{\class}{\examnum}{Página \thepage\ of \numpages}
        \runningheadrule
        
        \pointsinrightmargin % Para poner las puntuaciones a la derecha. Se puede cambiar. Si se comenta, sale a la izquierda.
        \extrawidth{-2.4cm} %Un poquito más de margen por si ponemos textos largos.
        \marginpointname{ \emph{\points}}

        %\printanswers
            \newcommand{\tipo}{A}\newcommand{\examnum}{Parcial}
        \begin{document}
        \noindent
        \begin{tabular*}{\textwidth}{l @{\extracolsep{\fill}} r @{\extracolsep{6pt}} }
        \textbf{Nombre:} \makebox[3.5in]{\hrulefill} & \textbf{Fecha:}\makebox[1in]{\hrulefill} \\
         & \\
        \textbf{Tiempo: \timelimit} & Tipo: \tipo 
        \end{tabular*}
        \rule[2ex]{\textwidth}{2pt}
        Esta prueba tiene \numquestions\ ejercicios. La puntuación máxima es de \numpoints. 
        La nota final de la prueba será la parte proporcional de la puntuación obtenida sobre la puntuación máxima. 

        \begin{center}


        \addpoints
             %\gradetable[h][questions]
            \pointtable[h][questions]
        \end{center}

        \noindent
        \rule[2ex]{\textwidth}{2pt}

        \begin{questions}
        \question Dados los siguientes conjuntos A, B y C, represéntalos en la recta real. A continuación, calcula $A \cup  B$ , $A \cap B$ y $A \cap B \cap C$ , 
y expresa los resultados en forma de Intervalos. 
Encuentra, si existe, el supremo y el ínfimo de cada uno de los
conjuntos anteriores
        \begin{multicols}{1} 
        \begin{parts} \part[1]  $ A=\left\{ x \in \mathbb{R}| 3 \leq x \wedge x < 8 \right\}, \\ B=\left(-\infty, 1\right) \cup \left(3, \infty\right)  y \\  C=\left\{ x \in \mathbb{R}| \left|{x + 2}\right|\geq8 \right\} \\ $  \begin{solution}  $ A \cup  B = \left(-\infty, 1\right) \cup \left[3, \infty\right)  \\  A \cap B= \left(3, 8\right)   \\  A \cap B  \cap C= \left[6, 8\right) $  \end{solution} \part[1]  $ A=\left\{ x \in \mathbb{R}| -2 \leq x \wedge x \leq 5 \right\}, \\ B=\left(-\infty, -1\right) \cup \left(1, \infty\right)  y \\  C=\left\{ x \in \mathbb{R}| \left|{x - 2}\right|\leq3 \right\} \\ $  \begin{solution}  $ A \cup  B = \left(-\infty, \infty\right)  \\  A \cap B= \left[-2, -1\right) \cup \left(1, 5\right]   \\  A \cap B  \cap C= \left(1, 5\right] $  \end{solution}
        \end{parts}
        \end{multicols}
        \question Resolver las ecuaciones:
        \begin{multicols}{1} 
        \begin{parts} \part[1]  $ {\binom{x}{2}} = 21 $  \begin{solution}  $ \left\{7\right\} $  \end{solution} \part[1]  $ {\binom{x}{x - 2}} = 10 $  \begin{solution}  $ \left\{5\right\} $  \end{solution}
        \end{parts}
        \end{multicols}
        
    \end{questions}
    \end{document}
    