
        \documentclass[addpoints,spanish, 12pt,a4paper]{exam}
        %\documentclass[answers, spanish, 12pt,a4paper]{exam}
        %\printanswers
        \pointpoints{punto}{puntos}
        \hpword{Puntos:}
        \vpword{Puntos:}
        \htword{Total}
        \vtword{Total}
        \hsword{Resultado:}
        \hqword{Ejercicio:}
        \vqword{Ejercicio:}

        \usepackage[utf8]{inputenc}
        \usepackage[spanish]{babel}
        \usepackage{eurosym}
        %\usepackage[spanish,es-lcroman, es-tabla, es-noshorthands]{babel}


        \usepackage[margin=1in]{geometry}
        \usepackage{amsmath,amssymb}
        \usepackage{multicol, xparse}

        \usepackage{yhmath}

        \usepackage{verbatim}
        %\usepackage{pstricks}


        \usepackage{graphicx}
        \graphicspath{{../img/}}




        \let\multicolmulticols\multicols
        \let\endmulticolmulticols\endmulticols
        \RenewDocumentEnvironment{multicols}{mO{}}
         {%
          \ifnum#1=1
            #2%
          \else % More than 1 column
            \multicolmulticols{#1}[#2]
          \fi
         }
         {%
          \ifnum#1=1
          \else % More than 1 column
            \endmulticolmulticols
          \fi
         }
        \renewcommand{\solutiontitle}{\noindent\textbf{Sol:}\enspace}

        \newcommand{\samedir}{\mathbin{\!/\mkern-5mu/\!}}

        \newcommand{\class}{1º Bachillerato}
        \newcommand{\examdate}{\today}

        %\newcommand{\tipo}{A}


        \newcommand{\timelimit}{50 minutos}

        \renewcommand{\solutiontitle}{\noindent\textbf{Solución:}\enspace}


        \pagestyle{head}
        \firstpageheader{\includegraphics[width=0.2\columnwidth]{header_left}}{\textbf{Departamento de Matemáticas\linebreak \class}\linebreak \examnum}{\includegraphics[width=0.1\columnwidth]{header_right}}
        \runningheader{\class}{\examnum}{Página \thepage\ of \numpages}
        \runningheadrule
        
        \pointsinrightmargin % Para poner las puntuaciones a la derecha. Se puede cambiar. Si se comenta, sale a la izquierda.
        \extrawidth{-2.4cm} %Un poquito más de margen por si ponemos textos largos.
        \marginpointname{ \emph{\points}}

        \newcommand{\tipo}{A}\newcommand{\examnum}{Autoevaluación}
        \begin{document}
        \noindent
        \begin{tabular*}{\textwidth}{l @{\extracolsep{\fill}} r @{\extracolsep{6pt}} }
        \textbf{Nombre:} \makebox[3.5in]{\hrulefill} & \textbf{Fecha:}\makebox[1in]{\hrulefill} \\
         & \\
        \textbf{Tiempo: \timelimit} & Tipo: \tipo 
        \end{tabular*}
        \rule[2ex]{\textwidth}{2pt}
        Esta prueba tiene \numquestions\ ejercicios. La puntuación máxima es de \numpoints. 
        La nota final de la prueba será la parte proporcional de la puntuación obtenida sobre la puntuación máxima. 

        \begin{center}


        \addpoints
             %\gradetable[h][questions]
            \pointtable[h][questions]
        \end{center}

        \noindent
        \rule[2ex]{\textwidth}{2pt}

        \begin{questions}
        \question Descomponer en factores
        \begin{multicols}{1} 
        \begin{parts} \part[1]  $ x^2-2x+1 $  \begin{solution}  $ \left(x - 1\right)^{2} $  \end{solution} \part[1]  $ x^3-x^2-x+1 $  \begin{solution}  $ \left(x - 1\right)^{2} \left(x + 1\right) $  \end{solution}
        \end{parts}
        \end{multicols}
        \question Halla el valor numérico del polinomio $x^4-2x^3-x^2+3$,  para los valores:
        \begin{multicols}{1} 
        \begin{parts} \part[1]  $ x=1 $  \begin{solution}  $ 1 $  \end{solution}
        \end{parts}
        \end{multicols}
        \question Calcula los siguientes radicales:
        \begin{multicols}{1} 
        \begin{parts} \part[1]  $ \sqrt {16} $  \begin{solution}  $ 4 $  \end{solution} \part[1]  $ \sqrt[4]{ - 16} $  \begin{solution}  $ 2 \sqrt[4]{-1} $  \end{solution}
        \end{parts}
        \end{multicols}
        \question Calcula y expresa el resultado como potencia de exponente racional:
        \begin{multicols}{1} 
        \begin{parts} \part[1]  $ \frac{{\sqrt[5]{a}  \cdot \sqrt {a} }}{{{a^{\frac{1}{3}}}}} $  \begin{solution}  $ a^{\frac{11}{30}} $  \end{solution}
        \end{parts}
        \end{multicols}
        \question Racionaliza:
        \begin{multicols}{1} 
        \begin{parts} \part[1]  $ \frac{6}{{\sqrt {5}  + \sqrt {2} }} $  \begin{solution}  $ - 2 \sqrt{2} + 2 \sqrt{5} $  \end{solution}
        \end{parts}
        \end{multicols}
        \question Efectúa:
        \begin{multicols}{1} 
        \begin{parts} \part[1]  $ \sqrt [3] {\sqrt[4] {a} } $  \begin{solution}  $ \sqrt[12]{a} $  \end{solution}
        \end{parts}
        \end{multicols}
        \question Racionaliza:
        \begin{multicols}{1} 
        \begin{parts} \part[1]  $ \frac{3}{{\sqrt {5} }} $  \begin{solution}  $ \frac{3 \sqrt{5}}{5} $  \end{solution}
        \end{parts}
        \end{multicols}
        \question Simplifica los cocientes entre factoriales:
        \begin{multicols}{1} 
        \begin{parts} \part[1]  $ \frac{{( {m + 1} )!}}{{( {m - 1} )!}} $  \begin{solution}  $ m \left(m + 1\right) $  \end{solution}
        \end{parts}
        \end{multicols}
        \question Resolver las siguientes inecuaciones:
        \begin{multicols}{1} 
        \begin{parts} \part[1]  $ \left|{x + 2}\right| - 5> 0 $  \begin{solution}  $ \left(-\infty < x \wedge x < -7\right) \vee \left(3 < x \wedge x < \infty\right) $  \end{solution}
        \end{parts}
        \end{multicols}
        \question Simplifica:
        \begin{multicols}{1} 
        \begin{parts} \part[1]  $ \frac{{6!}}{{5!}} + \frac{{8!}}{{6!}} $  \begin{solution}  $ 62 $  \end{solution}
        \end{parts}
        \end{multicols}
        \question Realiza los desarrollos de los siguientes binomios:
        \begin{multicols}{1} 
        \begin{parts} \part[1]  $ (2 + x)^4 $  \begin{solution}  $ x^{4} + 8 x^{3} + 24 x^{2} + 32 x + 16 $  \end{solution}
        \end{parts}
        \end{multicols}
        \question Realiza los desarrollos de los siguientes binomios para identificar determinados términos y coeficientes:
        \begin{multicols}{1} 
        \begin{parts} \part[1]  $ (2 + x)^8 $  \begin{solution}  $ x^{8} + 16 x^{7} + 112 x^{6} + 448 x^{5} + 1120 x^{4} + 1792 x^{3} + 1792 x^{2} + 1024 x + 256 $  \end{solution} \part[1]  $ ( {3x - \frac{1}{x}})^7 $  \begin{solution}  $ 2187 x^{7} - 5103 x^{5} + 5103 x^{3} - 2835 x + \frac{945}{x} - \frac{189}{x^{3}} + \frac{21}{x^{5}} - \frac{1}{x^{7}} $  \end{solution}
        \end{parts}
        \end{multicols}
        
    \end{questions}
    \end{document}
    