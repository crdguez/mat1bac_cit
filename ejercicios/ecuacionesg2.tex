
        \documentclass[spanish, 11pt]{exam}

        %These tell TeX which packages to use.
        \usepackage{array,epsfig}
        \usepackage{amsmath, textcomp}
        \usepackage{amsfonts}
        \usepackage{amssymb}
        \usepackage{amsxtra}
        \usepackage{amsthm}
        \usepackage{mathrsfs}
        \usepackage{color}
        \usepackage{multicol, xparse}
        \usepackage{verbatim}


        \usepackage[utf8]{inputenc}
        \usepackage[spanish]{babel}
        \usepackage{eurosym}

        \usepackage{graphicx}
        \graphicspath{{../img/}}



        \printanswers
        \nopointsinmargin
        \pointformat{}

        %Pagination stuff.
        %\setlength{\topmargin}{-.3 in}
        %\setlength{\oddsidemargin}{0in}
        %\setlength{\evensidemargin}{0in}
        %\setlength{\textheight}{9.in}
        %\setlength{\textwidth}{6.5in}
        %\pagestyle{empty}

        \let\multicolmulticols\multicols
        \let\endmulticolmulticols\endmulticols
        \RenewDocumentEnvironment{multicols}{mO{}}
         {%
          \ifnum#1=1
            #2%
          \else % More than 1 column
            \multicolmulticols{#1}[#2]
          \fi
         }
         {%
          \ifnum#1=1
          \else % More than 1 column
            \endmulticolmulticols
          \fi
         }
        \renewcommand{\solutiontitle}{\noindent\textbf{Sol:}\enspace}

        \newcommand{\samedir}{\mathbin{\!/\mkern-5mu/\!}}

        \newcommand{\class}{1º Bachillerato}
        \newcommand{\examdate}{\today}

        \newcommand{\tipo}{A}


        \newcommand{\timelimit}{50 minutos}



        \pagestyle{head}
        \firstpageheader{\includegraphics[width=0.2\columnwidth]{header_left}}{\textbf{Departamento de Matemáticas\linebreak \class}\linebreak \examnum}{\includegraphics[width=0.1\columnwidth]{header_right}}
        \runningheader{\class}{\examnum}{Página \thepage\ of \numpages}
        \runningheadrule

        \newcommand{\examnum}{6 - Ecuaciones de segundo grado}
        \begin{document}
        \begin{questions}
        \question p016e01 - Resuelve las ecuaciones:
        \begin{multicols}{2} 
        \begin{parts} \part[1]  $ x^{2} + 6 = 0 $  \begin{solution}  $ \emptyset $  \end{solution} \part[1]  $ x^{2} - 9 = 0 $  \begin{solution}  $ \left\{-3, 3\right\} $  \end{solution} \part[1]  $ x^{2} + 3 x = 0 $  \begin{solution}  $ \left\{-3, 0\right\} $  \end{solution} \part[1]  $ 3 x^{2} - 11 x = 0 $  \begin{solution}  $ \left\{0, \frac{11}{3}\right\} $  \end{solution} \part[1]  $ 4 x^{2} - 32 x = 0 $  \begin{solution}  $ \left\{0, 8\right\} $  \end{solution} \part[1]  $ 5 x^{2} = 0 $  \begin{solution}  $ \left\{0\right\} $  \end{solution} \part[1]  $ 12 x^{2} - 18 = 0 $  \begin{solution}  $ \left\{- \frac{\sqrt{6}}{2}, \frac{\sqrt{6}}{2}\right\} $  \end{solution} \part[1]  $ 3 \left(- x + 1\right) \left(x + 1\right) = 3 $  \begin{solution}  $ \left\{0\right\} $  \end{solution} \part[1]  $ 3 \left(x^{2} - 2\right) = 21 $  \begin{solution}  $ \left\{-3, 3\right\} $  \end{solution}
        \end{parts}
        \end{multicols}
        \question p016e02 - Resuelve las ecuaciones:
        \begin{multicols}{2} 
        \begin{parts} \part[1]  $ \left(2 x^{2} + 11 x\right) - 6 = 0 $  \begin{solution}  $ \left\{-6, \frac{1}{2}\right\} $  \end{solution} \part[1]  $ \left(x^{2} - 10 x\right) + 25 = 0 $  \begin{solution}  $ \left\{5\right\} $  \end{solution} \part[1]  $ \left(x^{2} + x\right) + 1 = 0 $  \begin{solution}  $ \emptyset $  \end{solution} \part[1]  $ \left(x^{2} - 2 x\right) - 1 = 0 $  \begin{solution}  $ \left\{1 + \sqrt{2}, - \sqrt{2} + 1\right\} $  \end{solution} \part[1]  $ \left(3 x^{2} + 5 x\right) - 2 = 0 $  \begin{solution}  $ \left\{-2, \frac{1}{3}\right\} $  \end{solution} \part[1]  $ \left(4 x^{2} - 4 x\right) + 1 = 0 $  \begin{solution}  $ \left\{\frac{1}{2}\right\} $  \end{solution} \part[1]  $ \left(2 x^{2} - 9 x\right) + 11 = 0 $  \begin{solution}  $ \emptyset $  \end{solution}
        \end{parts}
        \end{multicols}
        \question p016e03 - Resuelve las ecuaciones:
        \begin{multicols}{2} 
        \begin{parts} \part[1]  $ - x \left(x - 2\right) + 9 = 4 x + 6 $  \begin{solution}  $ \left\{-3, 1\right\} $  \end{solution} \part[1]  $ - \left(x - 1\right) \left(x + 4\right) + 2 \left(x^{2} - 3\right) = x - 2 $  \begin{solution}  $ \left\{0, 4\right\} $  \end{solution} \part[1]  $ x \left(x - 1\right) - 2 \left(x - 3\right) \left(x - 2\right) = 2 $  \begin{solution}  $ \left\{2, 7\right\} $  \end{solution} \part[1]  $ \left(2 x^{2} - 11 x\right) + 12 = 0 $  \begin{solution}  $ \left\{\frac{3}{2}, 4\right\} $  \end{solution} \part[1]  $ 3 \left(x - 1\right) \left(x + 2\right) = 0 $  \begin{solution}  $ \left\{-2, 1\right\} $  \end{solution} \part[1]  $ \left(x - 2\right)^{2} = 3 $  \begin{solution}  $ \left\{- \sqrt{3} + 2, \sqrt{3} + 2\right\} $  \end{solution} \part[1]  $ 21 x - 100 = - x + \left(x^{2} + 21\right) $  \begin{solution}  $ \left\{11\right\} $  \end{solution} \part[1]  $ \frac{x}{3} \left(x - \frac{1}{6}\right) = x - 1 $  \begin{solution}  $ \emptyset $  \end{solution} \part[1]  $ \left(- \frac{x}{3} + \frac{1}{3}\right) + \frac{2 x^{2} + 1}{2} = \frac{- x + 1}{6} $  \begin{solution}  $ \emptyset $  \end{solution}
        \end{parts}
        \end{multicols}
        \question p016e04 - Resuelve las siguientes ecuaciones de segundo grado, pasándolas previamente a la forma general:
        \begin{multicols}{2} 
        \begin{parts} \part[1]  $ \frac{x + 1}{x - 1} - \frac{1}{x} = \frac{5}{2} $  \begin{solution}  $ \left\{- \frac{1}{3}, 2\right\} $  \end{solution} \part[1]  $ - \frac{2}{x + 1} + \frac{3 x + 2}{x - 1} = 5 $  \begin{solution}  $ \left\{- \frac{3}{2}, 3\right\} $  \end{solution} \part[1]  $ - \frac{x}{x + 4} + 1 = \frac{1}{x - 5} $  \begin{solution}  $ \left\{8\right\} $  \end{solution} \part[1]  $ \frac{x}{x + 1} + \frac{x + 1}{x} = \frac{13}{6} $  \begin{solution}  $ \left\{-3, 2\right\} $  \end{solution} \part[1]  $ \left(-3 + \frac{2 \left(2 x + 1\right)}{2 x - 1}\right) + 5 = 0 $  \begin{solution}  $ \left\{0\right\} $  \end{solution} \part[1]  $ \frac{x - 3}{x + 3} + \frac{x + 3}{x - 3} = \frac{x - 2}{x + 3} $  \begin{solution}  $ \emptyset $  \end{solution} \part[1]  $ - \frac{x - 7}{x - 1} + \frac{2 x - 1}{x + 1} = 4 - \frac{3 x - 1}{x + 2} $  \begin{solution}  $ \left\{- \frac{5}{4}, 5\right\} $  \end{solution} \part[1]  $ \frac{3 x - 4}{5 x - 16} = \frac{4 x + 1}{6 x - 11} $  \begin{solution}  $ \left\{-5, 6\right\} $  \end{solution} \part[1]  $ \frac{- x + 3}{5} = - \frac{4}{5} + \frac{2}{x} $  \begin{solution}  $ \left\{2, 5\right\} $  \end{solution} \part[1]  $ \frac{x^{2}}{x + 1} = \left(\frac{x^{3}}{x^{2} - 1} - \frac{1}{x - 1}\right) + \frac{1}{- x^{2} + 1} $  \begin{solution}  $ \left\{2\right\} $  \end{solution} \part[1]  $ \frac{5}{2 x + 6} = \left(\left(\frac{1}{x{\left (x - 3 \right )}} + \frac{1}{\left(x - 3\right) \left(x + 3\right)}\right) + \frac{1}{x + 3}\right) - \frac{1}{x - 3} $  \begin{solution}  $ \left\{x \mid x \in \mathbb{R} \wedge - 2 \left(x - 3\right) \left(x + 3\right) + 3 \left(x - 3\right) x{\left (x - 3 \right )} + 2 \left(x + 3\right) x{\left (x - 3 \right )} - 2 x{\left (x - 3 \right )} = 0 \right\} \setminus \left\{x \mid x \in \mathbb{R} \wedge \left(x - 3\right) \left(x + 3\right) x{\left (x - 3 \right )} = 0 \right\} $  \end{solution} \part[1]  $ \frac{x}{2} + \frac{4}{x} = \frac{12}{x} $  \begin{solution}  $ \left\{-4, 4\right\} $  \end{solution} \part[1]  $ \frac{5}{4 x^{2}} - \frac{1}{2 x^{2}} = \frac{1}{3} $  \begin{solution}  $ \left\{- \frac{3}{2}, \frac{3}{2}\right\} $  \end{solution} \part[1]  $ \frac{- x + 6}{3} - \frac{3 \left(x - 4\right)}{x + 6} = \frac{x - 2}{2} $  \begin{solution}  $ \left\{-3 + 3 \sqrt{5}, - 3 \sqrt{5} - 3\right\} $  \end{solution}
        \end{parts}
        \end{multicols}
        \question p017e05 - Discute, sin resolver, las ecuaciones:
        \begin{multicols}{2} 
        \begin{parts} \part[1]  $ \left(x^{2} - 9 x\right) + 1 = 0 $  \begin{solution}  $ \left\{- \frac{\sqrt{77}}{2} + \frac{9}{2}, \frac{\sqrt{77}}{2} + \frac{9}{2}\right\} $  \end{solution} \part[1]  $ \left(2 x^{2} + 6 x\right) - 5 = 0 $  \begin{solution}  $ \left\{- \frac{3}{2} + \frac{\sqrt{19}}{2}, - \frac{\sqrt{19}}{2} - \frac{3}{2}\right\} $  \end{solution} \part[1]  $ \left(3 x^{2} - x\right) + 1 = 0 $  \begin{solution}  $ \emptyset $  \end{solution} \part[1]  $ \left(x^{2} - 12 x\right) + 36 = 0 $  \begin{solution}  $ \left\{6\right\} $  \end{solution}
        \end{parts}
        \end{multicols}
        \question p017e19 - Resuelve:
        \begin{multicols}{2} 
        \begin{parts} \part[1]  $ x^{4} - 16 = 0 $  \begin{solution}  $ \left\{-2, 2\right\} $  \end{solution} \part[1]  $ x^{4} - 225 x^{2} = 0 $  \begin{solution}  $ \left\{-15, 0, 15\right\} $  \end{solution} \part[1]  $ \left(x^{4} - 10 x^{2}\right) + 9 = 0 $  \begin{solution}  $ \left\{-3, -1, 1, 3\right\} $  \end{solution} \part[1]  $ \left(2 x^{4} + 11 x^{2}\right) - 6 = 0 $  \begin{solution}  $ \left\{- \frac{\sqrt{2}}{2}, \frac{\sqrt{2}}{2}\right\} $  \end{solution} \part[1]  $ \left(x^{4} - 6 x^{2}\right) + 8 = 0 $  \begin{solution}  $ \left\{-2, 2, - \sqrt{2}, \sqrt{2}\right\} $  \end{solution} \part[1]  $ x^{4} + 2 x^{2} = - 3 $  \begin{solution}  $ \emptyset $  \end{solution} \part[1]  $ \left(x^{4} - 8 x^{2}\right) - 9 = 0 $  \begin{solution}  $ \left\{-3, 3\right\} $  \end{solution} \part[1]  $ \left(x^{4} - 10 x^{2}\right) + 25 = 0 $  \begin{solution}  $ \left\{- \sqrt{5}, \sqrt{5}\right\} $  \end{solution} \part[1]  $ \left(x^{4} - 29 x^{2}\right) + 100 = 0 $  \begin{solution}  $ \left\{-5, -2, 2, 5\right\} $  \end{solution} \part[1]  $ \left(x^{4} + 21 x^{2}\right) - 100 = 0 $  \begin{solution}  $ \left\{-2, 2\right\} $  \end{solution} \part[1]  $ 9 x^{4} + 16 = 40 x^{2} $  \begin{solution}  $ \left\{-2, - \frac{2}{3}, \frac{2}{3}, 2\right\} $  \end{solution} \part[1]  $ \left(x^{4} - \frac{5 x^{2}}{4}\right) + \frac{1}{4} = 0 $  \begin{solution}  $ \left\{-1, - \frac{1}{2}, \frac{1}{2}, 1\right\} $  \end{solution} \part[1]  $ - x^{2} + 34 = \frac{225}{x^{2}} $  \begin{solution}  $ \left\{-5, -3, 3, 5\right\} $  \end{solution} \part[1]  $ x^{2} = \frac{12}{x^{2} - 1} $  \begin{solution}  $ \left\{-2, 2\right\} $  \end{solution} \part[1]  $ \left(x^{4} + 4 x^{2}\right) + 8 = 0 $  \begin{solution}  $ \emptyset $  \end{solution} \part[1]  $ -2 + \frac{8}{x^{2} - 5} = \frac{\left(x - 3\right) \left(x + 3\right)}{x^{2} - 1} $  \begin{solution}  $ \left\{-3, 3, - \frac{\sqrt{21}}{3}, \frac{\sqrt{21}}{3}\right\} $  \end{solution} \part[1]  $ \frac{x^{2} \left(2 x + 5\right)}{x + 1} = \frac{9 \left(- x + 1\right)}{2 x - 5} $  \begin{solution}  $ \left\{- \frac{3 \sqrt{2}}{2}, \frac{3 \sqrt{2}}{2}\right\} $  \end{solution}
        \end{parts}
        \end{multicols}
        \question p018e20 - Resuelve:
        \begin{multicols}{2} 
        \begin{parts} \part[1]  $ \sqrt{2 x - 1} + 5 = 2 x + 4 $  \begin{solution}  $ \left\{\frac{1}{2}, 1\right\} $  \end{solution} \part[1]  $ 2 \sqrt{x - 3} + \sqrt{6 x - 8} = 6 $  \begin{solution}  $ \left\{4\right\} $  \end{solution} \part[1]  $ \sqrt{2 x + 2} = x - 3 $  \begin{solution}  $ \left\{7\right\} $  \end{solution} \part[1]  $ \sqrt{2 x - 1} + 5 = 2 \sqrt{x + 3} + 2 $  \begin{solution}  $ \left\{1, 13\right\} $  \end{solution} \part[1]  $ - \sqrt{x - 2} + \sqrt{x - 1} = 1 $  \begin{solution}  $ \left\{2\right\} $  \end{solution} \part[1]  $ \sqrt{x - 1} + 2 = x - 5 $  \begin{solution}  $ \left\{10\right\} $  \end{solution} \part[1]  $ \sqrt{x} + x = 6 $  \begin{solution}  $ \left\{4\right\} $  \end{solution} \part[1]  $ \sqrt{x} + \sqrt{x + 4} = 4 $  \begin{solution}  $ \left\{\frac{9}{4}\right\} $  \end{solution} \part[1]  $ \sqrt{3 x - 2} - 4 = 0 $  \begin{solution}  $ \left\{6\right\} $  \end{solution} \part[1]  $ \sqrt{2 x + 1} = x - 1 $  \begin{solution}  $ \left\{4\right\} $  \end{solution} \part[1]  $ - x + \sqrt{- 3 x + 7} = 7 $  \begin{solution}  $ \left\{-3\right\} $  \end{solution} \part[1]  $ 3 \sqrt{6 x + 1} - 5 = 2 x $  \begin{solution}  $ \left\{\frac{1}{2}, 8\right\} $  \end{solution} \part[1]  $ \sqrt{3 x + 1} + 1 = 3 x $  \begin{solution}  $ \left\{1\right\} $  \end{solution} \part[1]  $ \sqrt{9 x^{2} - 11} + 1 = 3 x $  \begin{solution}  $ \left\{2\right\} $  \end{solution} \part[1]  $ \sqrt{\left(x^{2} + x\right) - 1} = - x + 2 $  \begin{solution}  $ \left\{1\right\} $  \end{solution} \part[1]  $ \sqrt{\frac{- x + 2}{x + 2}} = \frac{1}{2} $  \begin{solution}  $ \left\{\frac{6}{5}\right\} $  \end{solution} \part[1]  $ \sqrt{x + 4} = - \sqrt{x - 1} + 3 $  \begin{solution}  $ \left\{\frac{13}{9}\right\} $  \end{solution} \part[1]  $ \sqrt{x + 4} + \sqrt{2 x - 1} = 6 $  \begin{solution}  $ \left\{5\right\} $  \end{solution} \part[1]  $ 2 \sqrt{x + 4} = \sqrt{5 x + 4} $  \begin{solution}  $ \left\{12\right\} $  \end{solution} \part[1]  $ 2 \sqrt{2 x - 1} = \sqrt{2 x - 9} + \sqrt{6 x - 5} $  \begin{solution}  $ \left\{5\right\} $  \end{solution} \part[1]  $ \sqrt{x} + \frac{2}{\sqrt{x}} = \sqrt{x - 5} $  \begin{solution}  $ \emptyset $  \end{solution}
        \end{parts}
        \end{multicols}
        \question p018e21 - Resuelve las siguientes ecuaciones de grado superior a dos:
        \begin{multicols}{2} 
        \begin{parts} \part[1]  $ \left(7 x + \left(x^{3} - 7 x^{2}\right)\right) + 15 = 0 $  \begin{solution}  $ \left\{-1, 3, 5\right\} $  \end{solution} \part[1]  $ \left(- x + \left(x^{3} - 2 x^{2}\right)\right) + 2 = 0 $  \begin{solution}  $ \left\{-1, 1, 2\right\} $  \end{solution} \part[1]  $ \left(- 4 x + \left(- 16 x^{2} + \left(x^{4} + x^{3}\right)\right)\right) + 48 = 0 $  \begin{solution}  $ \left\{-4, -2, 2, 3\right\} $  \end{solution} \part[1]  $ \left(- x + \left(- 7 x^{2} + \left(x^{4} + x^{3}\right)\right)\right) + 6 = 0 $  \begin{solution}  $ \left\{-3, -1, 1, 2\right\} $  \end{solution} \part[1]  $ x^{4} - 81 = 0 $  \begin{solution}  $ \left\{-3, 3\right\} $  \end{solution}
        \end{parts}
        \end{multicols}
        \question p018e22e25e28e34 - Resuelve mediante expresiones algebraicas:
        \begin{multicols}{1} 
        \begin{parts} \part[1] Calcula un número que sumado con el doble de su raíz cuadrada nos de 24.  \begin{solution}  $ 2 \sqrt{x} + x - 24 = 0\to \left\{16\right\} $  \end{solution} \part[1] Tres segmentos miden, respectivamente, 8, 22 y 24 cm. 
Si a los tres les añadimos una misma longitud, el triángulo 
construido con ellos es rectángulo. Halla dicha longitud.  \begin{solution}  $ \left(x + 8\right)^{2} + \left(x + 22\right)^{2} - \left(x + 24\right)^{2} = 0\to \left\{2\right\} $  \end{solution} \part[1] Un terreno rectangular tiene 100 m de largo y 80 m de ancho y está rodeado 
por calles de anchura uniforme.
Si el área de las calles es de 4000 m² , ¿cuál es su anchura?  \begin{solution}  $ \left(x + 80\right) \left(x + 100\right) - 12000 = 0\to \left\{20\right\} $  \end{solution} \part[1] En cada esquina de una hoja de papel cuadrada se recorta un cuadrado de 5 cm de lado, y entonces,
doblando y pegando, se puede formar una caja de 1.280 cm 3 de capacidad. Halla el lado de la hoja inicial.  \begin{solution}  $ 5 \left(x - 10\right)^{2} - 1280 = 0\to \left\{26\right\} $  \end{solution}
        \end{parts}
        \end{multicols}
        \question p018e23 - Halla tres números impares consecutivos, tales que sus cuadrados sumen 5051
        \begin{multicols}{1} 
        \begin{parts} \part[1]  $ \left(2 x + 1\right)^{2} + \left(\left(2 x - 3\right)^{2} + \left(2 x - 1\right)^{2}\right) = 5051 $  \begin{solution}  $ \left [ -20, \quad 21\right ] \to \left [ -43, \quad -41, \quad -39\right ] \lor \left [ 39, \quad 41, \quad 43\right ] $  \end{solution}
        \end{parts}
        \end{multicols}
        \question p018e24e26e29e30e31e32e33 - Resuelve mediante expresiones algebraicas:
        \begin{multicols}{1} 
        \begin{parts} \part[1] Las dos cifras de un número suman 11 y el producto de dicho número por el que 
    se obtiene de invertir sus cifras es 3154. Halla dicho número.  \begin{solution}  $ x + y - 11 = 0\land \left(x + 10 y\right) \left(10 x + y\right) - 3154 = 0\to \left [ \left ( 3, \quad 8\right ), \quad \left ( 8, \quad 3\right )\right ] $  \end{solution} \part[1] El perímetro de un triángulo rectángulo es 90 m y el cateto mayor tiene 3 m menos que 
la hipotenusa. Halla los tres lados del triángulo.  \begin{solution}  $ x + 2 y - 87 = 0\land x^{2} + y^{2} - \left(y + 3\right)^{2} = 0\to \left [ \left ( -18, \quad \frac{105}{2}\right ), \quad \left ( 15, \quad 36\right )\right ] $  \end{solution} \part[1] Halla dos números cuya suma es 14 y la de sus cuadrados 100.  \begin{solution}  $ x + y - 14 = 0\land x^{2} + y^{2} - 100 = 0\to \left [ \left ( 6, \quad 8\right ), \quad \left ( 8, \quad 6\right )\right ] $  \end{solution} \part[1] Halla dos números positivos cuya diferencia sea 7 y la suma de sus cuadrados 3809.  \begin{solution}  $ x - y - 7 = 0\land x^{2} + y^{2} - 3809 = 0\to \left [ \left ( -40, \quad -47\right ), \quad \left ( 47, \quad 40\right )\right ] $  \end{solution} \part[1] Una habitación rectangular tiene una superficie de 28 m² y 
    su zócalo tiene una longitud de 22 m. Halla las
    dimensiones de la habitación.  \begin{solution}  $ x y - 28 = 0\land 2 x + 2 y - 22 = 0\to \left [ \left ( 4, \quad 7\right ), \quad \left ( 7, \quad 4\right )\right ] $  \end{solution} \part[1] Para vallar una finca rectangular de 750 m² se han utilizado 110 m de cerca. 
    Calcula las dimensiones de la cerca.  \begin{solution}  $ x y - 750 = 0\land 2 x + 2 y - 110 = 0\to \left [ \left ( 25, \quad 30\right ), \quad \left ( 30, \quad 25\right )\right ] $  \end{solution} \part[1] Uno de los lados de un rectángulo mide 6 cm más que el otro. 
    ¿Cuáles son sus dimensiones si su área es 91 cm²?  \begin{solution}  $ - x + y - 6 = 0\land x y - 91 = 0\to \left [ \left ( -13, \quad -7\right ), \quad \left ( 7, \quad 13\right )\right ] $  \end{solution}
        \end{parts}
        \end{multicols}
        \question p018e27 - Halla tres números pares consecutivos sabiendo que su 
producto es igual a cuatro veces su suma.
        \begin{multicols}{1} 
        \begin{parts} \part[1]  $ \left(2 x + 2\right) + \left(2 x + \left(2 x - 2\right)\right) = \frac{2 x \left(2 x + 2\right) \left(2 x - 2\right)}{4} $  \begin{solution}  $ \left [ -2, \quad 0, \quad 2\right ] \to \left [ -6, \quad -4, \quad -2\right ] \lor \left [ -2, \quad 0, \quad 2\right ] \lor \left [ 2, \quad 4, \quad 6\right ] $  \end{solution}
        \end{parts}
        \end{multicols}
        
    \end{questions}
    \end{document}
    