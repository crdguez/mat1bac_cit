
	\documentclass[spanish, 11pt]{exam}
	
	%These tell TeX which packages to use.
	\usepackage{array,epsfig}
	\usepackage{amsmath, textcomp}
	\usepackage{amsfonts}
	\usepackage{amssymb}
	\usepackage{amsxtra}
	\usepackage{amsthm}
	\usepackage{mathrsfs}
	\usepackage{color}
	\usepackage{multicol}
	\usepackage{verbatim}
	
	
	\usepackage[utf8]{inputenc}
	\usepackage[spanish]{babel}
	\usepackage{eurosym}
	
	\usepackage{graphicx}
	\graphicspath{{../img/}}
	
	
	
	\printanswers
	\nopointsinmargin
	\pointformat{}
	
	%Pagination stuff.
	%\setlength{\topmargin}{-.3 in}
	%\setlength{\oddsidemargin}{0in}
	%\setlength{\evensidemargin}{0in}
	%\setlength{\textheight}{9.in}
	%\setlength{\textwidth}{6.5in}
	%\pagestyle{empty}
	
	\renewcommand{\solutiontitle}{\noindent\textbf{Sol:}\enspace}
	
	\newcommand{\samedir}{\mathbin{\!/\mkern-5mu/\!}}
	
	\newcommand{\class}{1º Bachillerato}
	\newcommand{\examdate}{\today}
	
	\newcommand{\tipo}{A}
	
	
	\newcommand{\timelimit}{50 minutos}
	
	
	
	\pagestyle{head}
	\firstpageheader{\includegraphics[width=0.2\columnwidth]{header_left}}{\textbf{Departamento de Matemáticas\linebreak \class}\linebreak \examnum}{\includegraphics[width=0.1\columnwidth]{header_right}}
	\runningheader{\class}{\examnum}{Página \thepage\ of \numpages}
	\runningheadrule
	
	\newcommand{\examnum}{Potencias y radicales}
    \begin{document}
    \begin{questions}
    \question Calcula :
        \begin{multicols}{3} 
        \begin{parts} \part[]  $ \frac{{{3^{ - 2}} \cdot {3^5} \cdot {2^3}}}{{{{( {3 \cdot 2} )}^4}}} $  \begin{solution}  $ \frac{1}{6} $  \end{solution} \part[]  $ {3^{ - 5}} \cdot {( {\frac{1}{3}} )^{ - 2}} \cdot 81 $  \begin{solution}  $ 3 $  \end{solution} \part[]  $ {( {\frac{5}{4}} )^5} \cdot \frac{{{2^6}}}{{{5^2}}} $  \begin{solution}  $ \frac{125}{16} $  \end{solution} \part[]  $ \frac{{{2^{ - 2}} \cdot {{( {{2^2}} )}^3}}}{{{2^{ - 3}}}} $  \begin{solution}  $ 128 $  \end{solution} \part[]  $ \frac{{{5^{ - 3}} \cdot {5^{ - 1}} \cdot {5^2}}}{{{5^0} + {5^6}}} $  \begin{solution}  $ \frac{1}{390650} $  \end{solution} \part[]  $ {( {\frac{2}{3}} )^{ - 2}} \cdot {( {\frac{3}{2}} )^4} $  \begin{solution}  $ \frac{729}{64} $  \end{solution} \part[]  $ \frac{{\sqrt {2}  \cdot {{( {\sqrt {2} } )}^3} \cdot {{( {\sqrt {5} } )}^3}}}{{{{( {5\sqrt{2} } )}^2}}} $  \begin{solution}  $ \frac{2 \sqrt{5}}{5} $  \end{solution} \part[]  $ \frac{{{9^{\frac{1}{2}}} \cdot {3^{ - 1}} \cdot {2^{\frac{3}{2}}}}}{{\sqrt {2} }} $  \begin{solution}  $ 2 $  \end{solution}
        \end{parts}
        \end{multicols}
        \question Calcula los siguientes radicales:
        \begin{multicols}{3} 
        \begin{parts} \part[]  $ \sqrt {16} $  \begin{solution}  $ 4 $  \end{solution} \part[]  $ \sqrt[4]{ - 16} $  \begin{solution}  $ 2 \sqrt[4]{-1} $  \end{solution} \part[]  $ \sqrt[3]{27} $  \begin{solution}  $ 3 $  \end{solution} \part[]  $ \sqrt[5]{-1} $  \begin{solution}  $ \sqrt[5]{-1} $  \end{solution} \part[]  $ \sqrt {1225} $  \begin{solution}  $ 35 $  \end{solution} \part[]  $ \sqrt[7]{1} $  \begin{solution}  $ 1 $  \end{solution}
        \end{parts}
        \end{multicols}
        \question Resuelve las siguientes ecuaciones:
        \begin{multicols}{3} 
        \begin{parts} \part[]  $ x^4 = 81 $  \begin{solution}  $ x^{4} = 81 $  \end{solution} \part[]  $ x^3 = 125 $  \begin{solution}  $ x^{3} = 125 $  \end{solution} \part[]  $ x^2 = -6 $  \begin{solution}  $ x^{2} = -6 $  \end{solution} \part[]  $ x^5 = -1 $  \begin{solution}  $ x^{5} = -1 $  \end{solution}
        \end{parts}
        \end{multicols}
        \question Calcula y expresa el resultado de la forma más simple:
        \begin{multicols}{3} 
        \begin{parts} \part[]  $ \sqrt {27}  \cdot \sqrt {243}  \cdot \sqrt {81} $  \begin{solution}  $ 729 $  \end{solution} \part[]  $ \frac{{\sqrt[3] {625} }}{{\sqrt[3] {5}  }} $  \begin{solution}  $ 5 $  \end{solution} \part[]  $ ({\sqrt[3]{5}})^7 $  \begin{solution}  $ 25 \sqrt[3]{5} $  \end{solution} \part[]  $ \sqrt[3]{\sqrt{8}} $  \begin{solution}  $ \sqrt{2} $  \end{solution}
        \end{parts}
        \end{multicols}
        \question Extrae factores fuera del signo radical en:
        \begin{multicols}{3} 
        \begin{parts} \part[]  $ \sqrt {512} $  \begin{solution}  $ 16 \sqrt{2} $  \end{solution} \part[]  $ \sqrt[3]{11664} $  \begin{solution}  $ 18 \sqrt[3]{2} $  \end{solution} \part[]  $ \sqrt[4]{48} $  \begin{solution}  $ 2 \sqrt[4]{3} $  \end{solution} \part[]  $ \sqrt{a^5\cdot b^3} $  \begin{solution}  $ \sqrt{a^{5} b^{3}} $  \end{solution}
        \end{parts}
        \end{multicols}
        \question Calcula y expresa el resultado como potencia de exponente racional:
        \begin{multicols}{3} 
        \begin{parts} \part[]  $ \sqrt{a}  \cdot \sqrt[5]{a} \cdot \sqrt[6]{a} $  \begin{solution}  $ a^{\frac{13}{15}} $  \end{solution} \part[]  $ \sqrt[3]{\sqrt{a} } $  \begin{solution}  $ \sqrt[6]{a} $  \end{solution} \part[]  $ \frac{{\sqrt[3]{a} 3}}{{\sqrt{a}}} $  \begin{solution}  $ \frac{3}{\sqrt[6]{a}} $  \end{solution} \part[]  $ \sqrt {\sqrt {\sqrt {2} } } $  \begin{solution}  $ \sqrt[8]{2} $  \end{solution} \part[]  $ \sqrt {a\sqrt[3]{a} } $  \begin{solution}  $ \sqrt{a^{\frac{4}{3}}} $  \end{solution} \part[]  $ \frac{{\sqrt[5]{a}  \cdot \sqrt {a} }}{{{a^{\frac{1}{3}}}}} $  \begin{solution}  $ a^{\frac{11}{30}} $  \end{solution}
        \end{parts}
        \end{multicols}
        \question Calcula:
        \begin{multicols}{3} 
        \begin{parts} \part[]  $ 4\sqrt {3125}  + 2\sqrt {20}  - 30\sqrt {45} $  \begin{solution}  $ 14 \sqrt{5} $  \end{solution} \part[]  $ \frac{1}{4}\sqrt {3125}  - 2\sqrt {20}  - \frac{3}{2}\sqrt {45} $  \begin{solution}  $ - \frac{9 \sqrt{5}}{4} $  \end{solution}
        \end{parts}
        \end{multicols}
        \question Racionaliza:
        \begin{multicols}{3} 
        \begin{parts} \part[]  $ \frac{3}{{2\sqrt {6} }} $  \begin{solution}  $ \frac{\sqrt{6}}{4} $  \end{solution} \part[]  $ \frac{5}{{\sqrt{2} }} $  \begin{solution}  $ \frac{5 \sqrt{2}}{2} $  \end{solution} \part[]  $ \frac{2}{{3\sqrt[3]{ 2} }} $  \begin{solution}  $ \frac{2^{\frac{2}{3}}}{3} $  \end{solution} \part[]  $ \frac{2}{{5\sqrt[5]{ 2} }} $  \begin{solution}  $ \frac{2^{\frac{4}{5}}}{5} $  \end{solution} \part[]  $ \frac{6}{{\sqrt {5}  + \sqrt {2} }} $  \begin{solution}  $ - 2 \sqrt{2} + 2 \sqrt{5} $  \end{solution} \part[]  $ \frac{5}{{2 - \sqrt {6} }} $  \begin{solution}  $ \frac{- 5 \sqrt{6} - 10}{2} $  \end{solution}
        \end{parts}
        \end{multicols}
        \question Calcula, descomponiendo el radicando en factores primos:
        \begin{multicols}{3} 
        \begin{parts} \part[]  $ \sqrt {729} $  \begin{solution}  $ 27 $  \end{solution} \part[]  $ \sqrt[3]{64000} $  \begin{solution}  $ 40 $  \end{solution} \part[]  $ \sqrt[4]{50625} $  \begin{solution}  $ 15 $  \end{solution} \part[]  $ \sqrt[5]{59049} $  \begin{solution}  $ 9 $  \end{solution}
        \end{parts}
        \end{multicols}
        \question Calcula:
        \begin{multicols}{3} 
        \begin{parts} \part[]  $ 5\sqrt {8}  - 2\sqrt {50}  + \sqrt {32}  - \sqrt {2} $  \begin{solution}  $ 3 \sqrt{2} $  \end{solution} \part[]  $ \sqrt {27}  - \frac{1}{4}\sqrt {12}  + \frac{2}{5}\sqrt {75} $  \begin{solution}  $ \frac{9 \sqrt{3}}{2} $  \end{solution} \part[]  $ \sqrt {\frac{2}{9}}  + \sqrt 8  - \sqrt {\frac{1}{8}} $  \begin{solution}  $ \frac{\sqrt{2}}{3} $  \end{solution}
        \end{parts}
        \end{multicols}
        \question Calcula y simplifica:
        \begin{multicols}{3} 
        \begin{parts} \part[]  $ \sqrt [3] {5}  \cdot \sqrt [4] {3}   \cdot \sqrt {2} $  \begin{solution}  $ \sqrt{2} \sqrt[4]{3} \sqrt[3]{5} $  \end{solution} \part[]  $ \frac{{\sqrt[3]{5} \cdot \sqrt{3}}}{{\sqrt {15}  \cdot \sqrt {6} }} $  \begin{solution}  $ \frac{5^{\frac{5}{6}} \sqrt{6}}{30} $  \end{solution} \part[]  $ \frac{{\sqrt[6]{5}}}{{\sqrt[3]{5}}} $  \begin{solution}  $ \frac{5^{\frac{5}{6}}}{5} $  \end{solution} \part[]  $ \sqrt[5]{{{27}^{\frac{5}{3}}}} $  \begin{solution}  $ 3 $  \end{solution} \part[]  $ \sqrt[3] {4}  \cdot \sqrt[4] {8}  \cdot \sqrt {2} $  \begin{solution}  $ 2 \cdot 2^{\frac{11}{12}} $  \end{solution}
        \end{parts}
        \end{multicols}
        \question Efectúa:
        \begin{multicols}{3} 
        \begin{parts} \part[]  $ \sqrt[3] {5} \cdot \sqrt[3]{5^2} $  \begin{solution}  $ 5 $  \end{solution} \part[]  $ \frac{{\sqrt[3]{{x^2}{y^3}} }}{{\sqrt[3]{xy} }} $  \begin{solution}  $ \frac{\sqrt[3]{x^{2} y^{3}}}{\sqrt[3]{x y}} $  \end{solution} \part[]  $ ( {\sqrt[5]{{3^2}} } )^4 $  \begin{solution}  $ 3 \cdot 3^{\frac{3}{5}} $  \end{solution} \part[]  $ \sqrt[3] {5}  \cdot \sqrt[4] {{5^2}} $  \begin{solution}  $ 5^{\frac{5}{6}} $  \end{solution} \part[]  $ 3\sqrt{ 5}  \cdot 2\sqrt[3]{25} $  \begin{solution}  $ 30 \sqrt[6]{5} $  \end{solution} \part[]  $ \sqrt[3]{{a^3} b}  \cdot \sqrt[6]{a{b^4}} $  \begin{solution}  $ \sqrt[6]{a b^{4}} \sqrt[3]{a^{3} b} $  \end{solution} \part[]  $ 3\sqrt[4]{2}  \cdot \sqrt {8}  $  \begin{solution}  $ 6 \cdot 2^{\frac{3}{4}} $  \end{solution} \part[]  $ \frac{{\sqrt[4]{{x^3}{y^3}} }}{{\sqrt[3]{xy} }} $  \begin{solution}  $ \frac{\sqrt[4]{x^{3} y^{3}}}{\sqrt[3]{x y}} $  \end{solution} \part[]  $ \frac{{4\sqrt [4]{6} }}{{2\sqrt {3} }} $  \begin{solution}  $ \frac{2 \sqrt[4]{2} \cdot 3^{\frac{3}{4}}}{3} $  \end{solution} \part[]  $ \frac{{6\sqrt[3] { 5} }}{{2\sqrt {10} }} $  \begin{solution}  $ \frac{3 \sqrt{2} \cdot 5^{\frac{5}{6}}}{10} $  \end{solution} \part[]  $ \frac{{\sqrt[5]{{{( {a + b})}^3}} }}{{\sqrt {a + b} }} $  \begin{solution}  $ \frac{\sqrt{a + b} \sqrt[5]{\left(a + b\right)^{3}}}{a + b} $  \end{solution} \part[]  $ \sqrt [3] {{x^2}}  \cdot \frac{{\sqrt [5] {xy} }}{{\sqrt {x{y^3}} }} $  \begin{solution}  $ \frac{\sqrt[5]{x y} \sqrt{x y^{3}} \sqrt[3]{x^{2}}}{x y^{3}} $  \end{solution} \part[]  $ \sqrt [3] {\sqrt[4] {a} } $  \begin{solution}  $ \sqrt[12]{a} $  \end{solution} \part[]  $ \sqrt {\sqrt [3]  {{x^2}\sqrt[5] {{x^3}} } } $  \begin{solution}  $ \sqrt[6]{x^{2} \sqrt[5]{x^{3}}} $  \end{solution} \part[]  $ \sqrt {n\sqrt [5] {n\sqrt [6]{n} } } $  \begin{solution}  $ \sqrt{n \sqrt[5]{n^{\frac{7}{6}}}} $  \end{solution}
        \end{parts}
        \end{multicols}
        \question Racionaliza:
        \begin{multicols}{3} 
        \begin{parts} \part[]  $ \frac{3}{{\sqrt {5} }} $  \begin{solution}  $ \frac{3 \sqrt{5}}{5} $  \end{solution} \part[]  $ \frac{{12}}{{\sqrt {8} }} $  \begin{solution}  $ 3 \sqrt{2} $  \end{solution} \part[]  $ \frac{5}{{\sqrt {5} }} $  \begin{solution}  $ \sqrt{5} $  \end{solution} \part[]  $ \frac{a}{{\sqrt[3]{{a^2}} }} $  \begin{solution}  $ \frac{a}{\sqrt[3]{a^{2}}} $  \end{solution} \part[]  $ \frac{{{x^2}}}{{\sqrt [4] {x} }} $  \begin{solution}  $ x^{\frac{7}{4}} $  \end{solution} \part[]  $ \frac{{abc}}{{\sqrt {ab{c^3}} }} $  \begin{solution}  $ \frac{\sqrt{a b c^{3}}}{c^{2}} $  \end{solution} \part[]  $ \frac{5}{{2 + \sqrt {3} }} $  \begin{solution}  $ - 5 \sqrt{3} + 10 $  \end{solution} \part[]  $ \frac{{\sqrt {2}  - \sqrt {3} }}{{\sqrt{ 2}  + \sqrt {3} }} $  \begin{solution}  $ - \left(- \sqrt{3} + \sqrt{2}\right)^{2} $  \end{solution} \part[]  $ \frac{{2 - \sqrt {2} }}{{2 + \sqrt {2} }} $  \begin{solution}  $ \frac{\left(- \sqrt{2} + 2\right)^{2}}{2} $  \end{solution} \part[]  $ \frac{1}{{\sqrt {\sqrt {2 }} }} $  \begin{solution}  $ \frac{2^{\frac{3}{4}}}{2} $  \end{solution} \part[]  $ \frac{a}{{\sqrt {a}  + \sqrt {b} }} $  \begin{solution}  $ \frac{a^{\frac{3}{2}} - a \sqrt{b}}{a - b} $  \end{solution}
        \end{parts}
        \end{multicols}
        
    \end{questions}
    \end{document}
    