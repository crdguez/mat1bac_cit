
        \documentclass[spanish, 11pt]{exam}

        %These tell TeX which packages to use.
        \usepackage{array,epsfig}
        \usepackage{amsmath, textcomp}
        \usepackage{amsfonts}
        \usepackage{amssymb}
        \usepackage{amsxtra}
        \usepackage{amsthm}
        \usepackage{mathrsfs}
        \usepackage{color}
        \usepackage{multicol, xparse}
        \usepackage{verbatim}


        \usepackage[utf8]{inputenc}
        \usepackage[spanish]{babel}
        \usepackage{eurosym}

        \usepackage{graphicx}
        \graphicspath{{../img/}}



        \printanswers
        \nopointsinmargin
        \pointformat{}

        %Pagination stuff.
        %\setlength{\topmargin}{-.3 in}
        %\setlength{\oddsidemargin}{0in}
        %\setlength{\evensidemargin}{0in}
        %\setlength{\textheight}{9.in}
        %\setlength{\textwidth}{6.5in}
        %\pagestyle{empty}

        \let\multicolmulticols\multicols
        \let\endmulticolmulticols\endmulticols
        \RenewDocumentEnvironment{multicols}{mO{}}
         {%
          \ifnum#1=1
            #2%
          \else % More than 1 column
            \multicolmulticols{#1}[#2]
          \fi
         }
         {%
          \ifnum#1=1
          \else % More than 1 column
            \endmulticolmulticols
          \fi
         }
        \renewcommand{\solutiontitle}{\noindent\textbf{Sol:}\enspace}

        \newcommand{\samedir}{\mathbin{\!/\mkern-5mu/\!}}

        \newcommand{\class}{1º Bachillerato}
        \newcommand{\examdate}{\today}

        \newcommand{\tipo}{A}


        \newcommand{\timelimit}{50 minutos}



        \pagestyle{head}
        \firstpageheader{\includegraphics[width=0.2\columnwidth]{header_left}}{\textbf{Departamento de Matemáticas\linebreak \class}\linebreak \examnum}{\includegraphics[width=0.1\columnwidth]{header_right}}
        \runningheader{\class}{\examnum}{Página \thepage\ of \numpages}
        \runningheadrule

        \newcommand{\examnum}{4 - Polinomios}
    \begin{document}
    \begin{questions}
    \question p012e03 - Dados los polinomios $ A(x)=2{x^3} - 5{x^2} + 6, \  B(x)=- \frac{1}{2}{x^5} - {x^4} + 6x$ halla:
        \begin{multicols}{3} 
        \begin{parts} \part[]  $ A(x) + B(x) $  \begin{solution}  $ - \frac{x^{5}}{2} - x^{4} + 2 x^{3} - 5 x^{2} + 6 x + 6 $  \end{solution} \part[]  $ A(x) - B(x) $  \begin{solution}  $ \frac{x^{5}}{2} + x^{4} + 2 x^{3} - 5 x^{2} - 6 x + 6 $  \end{solution}
        \end{parts}
        \end{multicols}
        \question p012e04 - Dados los polinomios $ A(x)=3{x^3} - 6{x^2} + 2x - 1, \  B(x)=- {x^4} + {x^3} + x - 6, \  C(x)={x^4} - {x^2} + x + \frac{1}{2}$ halla:
        \begin{multicols}{3} 
        \begin{parts} \part[]  $ A(x) \cdot B(x) $  \begin{solution}  $ - 3 x^{7} + 9 x^{6} - 8 x^{5} + 6 x^{4} - 25 x^{3} + 38 x^{2} - 13 x + 6 $  \end{solution} \part[]  $ A(x) - 3B(x) + 5C(x) $  \begin{solution}  $ 8 x^{4} - 11 x^{2} + 4 x + \frac{39}{2} $  \end{solution} \part[]  $ x^2\cdot A(x) + 3x\cdot B(x) $  \begin{solution}  $ - 3 x^{4} + 2 x^{3} + 2 x^{2} - 18 x $  \end{solution}
        \end{parts}
        \end{multicols}
        \question p012e07 - Halla el cociente y el resto de:
        \begin{multicols}{1} 
        \begin{parts} \part[]  $ ( {5{x^4} - 7{x^2} + 6x + 1} ):( {3{x^2}} ) $  \begin{solution}  $ \left ( \frac{5 x^{2}}{3} - \frac{7}{3}, \quad 6 x + 1\right ) $  \end{solution} \part[]  $ ( {7{x^4} - 3{x^2} + 6x - 1} ):( {{x^2}-x+3} ) $  \begin{solution}  $ \left ( 7 x^{2} + 7 x - 17, \quad - 32 x + 50\right ) $  \end{solution} \part[]  $ ( {{x^6} - 5} ):( {{x^2} - x}) $  \begin{solution}  $ \left ( x^{4} + x^{3} + x^{2} + x + 1, \quad x - 5\right ) $  \end{solution} \part[]  $ ( {8{x^6} - 5x^4 + 6 } ):( {2x^2} - 1) $  \begin{solution}  $ \left ( 4 x^{4} - \frac{x^{2}}{2} - \frac{1}{4}, \quad \frac{23}{4}\right ) $  \end{solution} \part[]  $ ( {3{x^5} - 6{x^2} + 9} ):( {{x^2} + 1} ) $  \begin{solution}  $ \left ( 3 x^{3} - 3 x - 6, \quad 3 x + 15\right ) $  \end{solution} \part[]  $ ( {x^9} - 7{x} + 1 ):( {{x^3} + x} ) $  \begin{solution}  $ \left ( x^{6} - x^{4} + x^{2} - 1, \quad - 6 x + 1\right ) $  \end{solution}
        \end{parts}
        \end{multicols}
        \question p012e08 - Dados $ A(x)= - {x^3} + 2{x^2} + 5, \  B(x)= 2{x^4} + 3x + 6$ halla el valor numérico de ambos polinomios en:
        \begin{multicols}{3} 
        \begin{parts} \part[]  $ x=1 $  \begin{solution}  $ 6\ y \ 11 $  \end{solution} \part[]  $ x=-1 $  \begin{solution}  $ 8\ y \ 5 $  \end{solution} \part[]  $ x=2 $  \begin{solution}  $ 5\ y \ 44 $  \end{solution} \part[]  $ x=-2 $  \begin{solution}  $ 21\ y \ 32 $  \end{solution} \part[]  $ x=\frac{1}{2} $  \begin{solution}  $ \frac{43}{8}\ y \ \frac{61}{8} $  \end{solution} \part[]  $ x=-\frac{1}{2} $  \begin{solution}  $ \frac{45}{8}\ y \ \frac{37}{8} $  \end{solution}
        \end{parts}
        \end{multicols}
        \question p012e09 - Halla, para cada uno  de los siguientes polinomios, sus raíces:
        \begin{multicols}{3} 
        \begin{parts} \part[]  $ x^2 - 1 $  \begin{solution}  $ \left\{-1, 1\right\} $  \end{solution} \part[]  $ x^2 - 7 $  \begin{solution}  $ \left\{- \sqrt{7}, \sqrt{7}\right\} $  \end{solution} \part[]  $ 3x^2 -12 $  \begin{solution}  $ \left\{-2, 2\right\} $  \end{solution} \part[]  $ 5x^2 -25 $  \begin{solution}  $ \left\{- \sqrt{5}, \sqrt{5}\right\} $  \end{solution}
        \end{parts}
        \end{multicols}
        \question p012e10 - ¿Tiene el polinomio A(x)= $x^4 +3$ alguna raíz real?
        \begin{multicols}{1} 
        \begin{parts} \part[]  $ x^4 +3 $  \begin{solution}  $ \left\{- \frac{\sqrt{2} \sqrt[4]{3}}{2} - \frac{\sqrt{2} \sqrt[4]{3} i}{2}, - \frac{\sqrt{2} \sqrt[4]{3}}{2} + \frac{\sqrt{2} \sqrt[4]{3} i}{2}, \frac{\sqrt{2} \sqrt[4]{3}}{2} - \frac{\sqrt{2} \sqrt[4]{3} i}{2}, \frac{\sqrt{2} \sqrt[4]{3}}{2} + \frac{\sqrt{2} \sqrt[4]{3} i}{2}\right\} $  \end{solution}
        \end{parts}
        \end{multicols}
        \question p013e11 - Aplica la regla de Ruffini para hallar el cociente y el resto de las siguientes divisiones:
        \begin{multicols}{1} 
        \begin{parts} \part[]  $ ( {{x^2} - 3x + 6} ):( {x+2} ) $  \begin{solution}  $ \left ( x - 5, \quad 16\right ) $  \end{solution} \part[]  $ ( {2{x^6} - 7{x^4} + 6x - 9} ):( {x+3} ) $  \begin{solution}  $ \left ( 2 x^{5} - 6 x^{4} + 11 x^{3} - 33 x^{2} + 99 x - 291, \quad 864\right ) $  \end{solution} \part[]  $ ( {7{x^3} - 4x - 3} ):( {{x} - 1}) $  \begin{solution}  $ \left ( 7 x^{2} + 7 x + 3, \quad 0\right ) $  \end{solution} \part[]  $ ( {x^2 - 1 } ):( {x} + 1) $  \begin{solution}  $ \left ( x - 1, \quad 0\right ) $  \end{solution}
        \end{parts}
        \end{multicols}
        \question p013e12 - Aplica el teorema del resto para calcular el resto de las siguientes divisiones:
        \begin{multicols}{3} 
        \begin{parts} \part[]  $ ( {7{x^3} - 4x + 9} ):( {x+1} ) $  \begin{solution}  $ 6 $  \end{solution} \part[]  $ ( {7{x^3} - 4x - 3} ):( {x-1} ) $  \begin{solution}  $ 0 $  \end{solution} \part[]  $ ( {x^2 - 1} ):( {x + 1}) $  \begin{solution}  $ 0 $  \end{solution}
        \end{parts}
        \end{multicols}
        \question p013e17-18 - Descomponer en factores
        \begin{multicols}{2} 
        \begin{parts} \part[]  $ x^2-81 $  \begin{solution}  $ \left(x - 9\right) \left(x + 9\right) $  \end{solution} \part[]  $ x^2-2 $  \begin{solution}  $ \left(x - \sqrt{2}\right) \left(x + \sqrt{2}\right) $  \end{solution} \part[]  $ 4x^2-9 $  \begin{solution}  $ 4 \left(x - \frac{3}{2}\right) \left(x + \frac{3}{2}\right) $  \end{solution} \part[]  $ x^3-x $  \begin{solution}  $ x \left(x - 1\right) \left(x + 1\right) $  \end{solution} \part[]  $ x^2-3x $  \begin{solution}  $ x \left(x - 3\right) $  \end{solution} \part[]  $ x^2-2x+1 $  \begin{solution}  $ \left(x - 1\right)^{2} $  \end{solution} \part[]  $ x^5-3x^4+2x^3 $  \begin{solution}  $ x^{3} \left(x - 2\right) \left(x - 1\right) $  \end{solution} \part[]  $ x^2-x-30 $  \begin{solution}  $ \left(x - 6\right) \left(x + 5\right) $  \end{solution} \part[]  $ x^2+2x+1 $  \begin{solution}  $ \left(x + 1\right)^{2} $  \end{solution} \part[]  $ x^3-x^2-x+1 $  \begin{solution}  $ \left(x - 1\right)^{2} \left(x + 1\right) $  \end{solution} \part[]  $ x^3-2x^2-5x+6 $  \begin{solution}  $ \left(x - 3\right) \left(x - 1\right) \left(x + 2\right) $  \end{solution} \part[]  $ x^5+4x^4+x^3-10x^2-4x+8 $  \begin{solution}  $ \left(x - 1\right)^{2} \left(x + 2\right)^{3} $  \end{solution} \part[]  $ x^3+3x^2-2x-6 $  \begin{solution}  $ \left(x + 3\right) \left(x - \sqrt{2}\right) \left(x + \sqrt{2}\right) $  \end{solution} \part[]  $ x^3-3x^2-13x+15 $  \begin{solution}  $ \left(x - 5\right) \left(x - 1\right) \left(x + 3\right) $  \end{solution} \part[]  $ x^3+x^2-6x $  \begin{solution}  $ x \left(x - 2\right) \left(x + 3\right) $  \end{solution} \part[]  $ 3x^3+x^2-12x-4 $  \begin{solution}  $ 3 \left(x - 2\right) \left(x + \frac{1}{3}\right) \left(x + 2\right) $  \end{solution} \part[]  $ x^4+2x^3-x^2-2x $  \begin{solution}  $ x \left(x - 1\right) \left(x + 1\right) \left(x + 2\right) $  \end{solution} \part[]  $ x^4-2x^3+2x^2-2x+1 $  \begin{solution}  $ \left(x - 1\right)^{2} \left(x^{2} + 1\right) $  \end{solution} \part[]  $ x^4+2x^3-3x^2-4x+4 $  \begin{solution}  $ \left(x - 1\right)^{2} \left(x + 2\right)^{2} $  \end{solution} \part[]  $ x^3+4x^2+x-6 $  \begin{solution}  $ \left(x - 1\right) \left(x + 2\right) \left(x + 3\right) $  \end{solution} \part[]  $ x^5- 4x^3-x^2+4 $  \begin{solution}  $ \left(x - 2\right) \left(x - 1\right) \left(x + 2\right) \left(x^{2} + x + 1\right) $  \end{solution}
        \end{parts}
        \end{multicols}
        \question p013e21 - Halla el valor numérico del polinomio $x^4-2x^3-x^2+3$,  para los valores:
        \begin{multicols}{3} 
        \begin{parts} \part[]  $ x=0 $  \begin{solution}  $ 3 $  \end{solution} \part[]  $ x=1 $  \begin{solution}  $ 1 $  \end{solution} \part[]  $ x=2 $  \begin{solution}  $ -1 $  \end{solution} \part[]  $ x=\frac{2}{3} $  \begin{solution}  $ \frac{175}{81} $  \end{solution}
        \end{parts}
        \end{multicols}
        
    \end{questions}
    \end{document}
    