
        \documentclass[spanish, 11pt]{exam}

        %These tell TeX which packages to use.
        \usepackage{array,epsfig}
        \usepackage{amsmath, textcomp}
        \usepackage{amsfonts}
        \usepackage{amssymb}
        \usepackage{amsxtra}
        \usepackage{amsthm}
        \usepackage{mathrsfs}
        \usepackage{color}
        \usepackage{multicol, xparse}
        \usepackage{verbatim}


        \usepackage[utf8]{inputenc}
        \usepackage[spanish]{babel}
        \usepackage{eurosym}

        \usepackage{graphicx}
        \graphicspath{{../img/}}



        \printanswers
        \nopointsinmargin
        \pointformat{}

        %Pagination stuff.
        %\setlength{\topmargin}{-.3 in}
        %\setlength{\oddsidemargin}{0in}
        %\setlength{\evensidemargin}{0in}
        %\setlength{\textheight}{9.in}
        %\setlength{\textwidth}{6.5in}
        %\pagestyle{empty}

        \let\multicolmulticols\multicols
        \let\endmulticolmulticols\endmulticols
        \RenewDocumentEnvironment{multicols}{mO{}}
         {%
          \ifnum#1=1
            #2%
          \else % More than 1 column
            \multicolmulticols{#1}[#2]
          \fi
         }
         {%
          \ifnum#1=1
          \else % More than 1 column
            \endmulticolmulticols
          \fi
         }
        \renewcommand{\solutiontitle}{\noindent\textbf{Sol:}\enspace}

        \newcommand{\samedir}{\mathbin{\!/\mkern-5mu/\!}}

        \newcommand{\class}{1º Bachillerato}
        \newcommand{\examdate}{\today}

        \newcommand{\tipo}{A}


        \newcommand{\timelimit}{50 minutos}



        \pagestyle{head}
        \firstpageheader{\includegraphics[width=0.2\columnwidth]{header_left}}{\textbf{Departamento de Matemáticas\linebreak \class}\linebreak \examnum}{\includegraphics[width=0.1\columnwidth]{header_right}}
        \runningheader{\class}{\examnum}{Página \thepage\ of \numpages}
        \runningheadrule

        \newcommand{\examnum}{24 - Complejos}
        \begin{document}
        \begin{questions}
        \question p045e01 - Calcula:
        \begin{multicols}{2} 
        \begin{parts} \part[1]  $ (7-2i)^2 + (3+4i)(5-2i) $  \begin{solution}  $ 68 - 14 i $  \end{solution} \part[1]  $ (2+i)^2(3-2i) + (5-i)i^2 $  \begin{solution}  $ 12 + 7 i $  \end{solution} \part[1]  $ (\sqrt{3}  - 2i)^2 + (2\sqrt{3}  -  5i)({1  -  2i}) $  \begin{solution}  $ -11 + 2 \sqrt{3} + i \left(- 8 \sqrt{3} - 5\right) $  \end{solution} \part[1]  $ (i^7-1)(i^{16} + i^3-i^9)^5 + (1-2i)^5(1+i) $  \begin{solution}  $ 0 $  \end{solution} \part[1]  $ (1+i)^2+\frac{1+i}{1-i} $  \begin{solution}  $ 3 i $  \end{solution}
        \end{parts}
        \end{multicols}
        \question p045e02 - Halla el valor de k, sabiendo que se cumple:
        \begin{multicols}{2} 
        \begin{parts} \part[1]  $ (k+5i) + (3+i) = (1+5i) + (-k+i) $  \begin{solution}  $ \left [ -1\right ] $  \end{solution} \part[1]  $ (1+3i)(k+2i) = 13+59i $  \begin{solution}  $ \left \{ k : 19\right \} $  \end{solution} \part[1]  $ k+\frac{4}{5}i=\frac{5+i}{3-i} $  \begin{solution}  $ \left [ \frac{7}{5}\right ] $  \end{solution}
        \end{parts}
        \end{multicols}
        \question p045e03 - Calcula el inverso de los siguientes números complejos:
        \begin{multicols}{2} 
        \begin{parts} \part[1]  $ -1 + 2i $  \begin{solution}  $ - \frac{1}{5} - \frac{2 i}{5} $  \end{solution} \part[1]  $ 3-\sqrt{2}i $  \begin{solution}  $ \frac{3}{11} + \frac{\sqrt{2} i}{11} $  \end{solution} \part[1]  $ \frac{1}{3}-\frac{1}{2}i $  \begin{solution}  $ \frac{12}{13} + \frac{18 i}{13} $  \end{solution}
        \end{parts}
        \end{multicols}
        
    \end{questions}
    \end{document}
    