
        \documentclass[spanish, 11pt]{exam}

        %These tell TeX which packages to use.
        \usepackage{array,epsfig}
        \usepackage{amsmath, textcomp}
        \usepackage{amsfonts}
        \usepackage{amssymb}
        \usepackage{amsxtra}
        \usepackage{amsthm}
        \usepackage{mathrsfs}
        \usepackage{color}
        \usepackage{multicol, xparse}
        \usepackage{verbatim}


        \usepackage[utf8]{inputenc}
        \usepackage[spanish]{babel}
        \usepackage{eurosym}

        \usepackage{graphicx}
        \graphicspath{{../img/}}



        \printanswers
        \nopointsinmargin
        \pointformat{}

        %Pagination stuff.
        %\setlength{\topmargin}{-.3 in}
        %\setlength{\oddsidemargin}{0in}
        %\setlength{\evensidemargin}{0in}
        %\setlength{\textheight}{9.in}
        %\setlength{\textwidth}{6.5in}
        %\pagestyle{empty}

        \let\multicolmulticols\multicols
        \let\endmulticolmulticols\endmulticols
        \RenewDocumentEnvironment{multicols}{mO{}}
         {%
          \ifnum#1=1
            #2%
          \else % More than 1 column
            \multicolmulticols{#1}[#2]
          \fi
         }
         {%
          \ifnum#1=1
          \else % More than 1 column
            \endmulticolmulticols
          \fi
         }
        \renewcommand{\solutiontitle}{\noindent\textbf{Sol:}\enspace}

        \newcommand{\samedir}{\mathbin{\!/\mkern-5mu/\!}}

        \newcommand{\class}{1º Bachillerato}
        \newcommand{\examdate}{\today}

        \newcommand{\tipo}{A}


        \newcommand{\timelimit}{50 minutos}



        \pagestyle{head}
        \firstpageheader{\includegraphics[width=0.2\columnwidth]{header_left}}{\textbf{Departamento de Matemáticas\linebreak \class}\linebreak \examnum}{\includegraphics[width=0.1\columnwidth]{header_right}}
        \runningheader{\class}{\examnum}{Página \thepage\ of \numpages}
        \runningheadrule

        \newcommand{\examnum}{24 - Complejos}
        \begin{document}
        \begin{questions}
        \question p045e01 - Calcula:
        \begin{multicols}{2} 
        \begin{parts} \part[1]  $ (7-2i)^2 + (3+4i)(5-2i) $  \begin{solution}  $ 68 - 14 i $  \end{solution} \part[1]  $ (2+i)^2(3-2i) + (5-i)i^2 $  \begin{solution}  $ 12 + 7 i $  \end{solution} \part[1]  $ (\sqrt{3}  - 2i)^2 + (2\sqrt{3}  -  5i)({1  -  2i}) $  \begin{solution}  $ -11 + 2 \sqrt{3} + i \left(- 8 \sqrt{3} - 5\right) $  \end{solution} \part[1]  $ (i^7-1)(i^{16} + i^3-i^9)^5 + (1-2i)^5(1+i) $  \begin{solution}  $ 0 $  \end{solution} \part[1]  $ (1+i)^2+\frac{1+i}{1-i} $  \begin{solution}  $ 3 i $  \end{solution}
        \end{parts}
        \end{multicols}
        \question p045e02 - Halla el valor de k, sabiendo que se cumple:
        \begin{multicols}{2} 
        \begin{parts} \part[1]  $ (k+5i) + (3+i) = (1+5i) + (-k+i) $  \begin{solution}  $ \left [ -1\right ] $  \end{solution} \part[1]  $ (1+3i)(k+2i) = 13+59i $  \begin{solution}  $ \left \{ k : 19\right \} $  \end{solution} \part[1]  $ k+\frac{4}{5}i=\frac{5+i}{3-i} $  \begin{solution}  $ \left [ \frac{7}{5}\right ] $  \end{solution}
        \end{parts}
        \end{multicols}
        \question p045e03 - Calcula el inverso de los siguientes números complejos:
        \begin{multicols}{2} 
        \begin{parts} \part[1]  $ -1 + 2i $  \begin{solution}  $ - \frac{1}{5} - \frac{2 i}{5} $  \end{solution} \part[1]  $ 3-\sqrt{2}i $  \begin{solution}  $ \frac{3}{11} + \frac{\sqrt{2} i}{11} $  \end{solution} \part[1]  $ \frac{1}{3}-\frac{1}{2}i $  \begin{solution}  $ \frac{12}{13} + \frac{18 i}{13} $  \end{solution}
        \end{parts}
        \end{multicols}
        \question p045e04y14 - Calcular el valor de k para que la siguiente expresión sea a) real y b) imaginario:
        \begin{multicols}{2} 
        \begin{parts} \part[1]  $ \frac{{k - 2i}}{{3 + 4i}} $  \begin{solution}  $ \frac{3 k}{25} - \frac{4 i k}{25} - \frac{8}{25} - \frac{6 i}{25}\to\left [ - \frac{3}{2}\right ]\land\left [ \frac{8}{3}\right ] $  \end{solution} \part[1]  $ k-2+(\frac{1}{4}+k)i $  \begin{solution}  $ k + i k - 2 + \frac{i}{4}\to\left [ - \frac{1}{4}\right ]\land\left [ 2\right ] $  \end{solution}
        \end{parts}
        \end{multicols}
        \question p045e05 - Determina el valor que debe tener k para que la siguiente expresión sea un número real.
        \begin{multicols}{2} 
        \begin{parts} \part[1]  $ (k-i)^3 $  \begin{solution}  $ k^{3} - 3 i k^{2} - 3 k + i\to\left [ - \frac{\sqrt{3}}{3}, \quad \frac{\sqrt{3}}{3}\right ] $  \end{solution}
        \end{parts}
        \end{multicols}
        \question p045e06y7 - Dados los siguientes números complejos, indica sus afijos:
        \begin{multicols}{2} 
        \begin{parts} \part[1]  $ 1+i $  \begin{solution}  $ \left ( 1, \quad 1\right ) $  \end{solution} \part[1]  $ (1+i)i $  \begin{solution}  $ \left ( -1, \quad 1\right ) $  \end{solution} \part[1]  $ (1+i)i\cdot i $  \begin{solution}  $ \left ( -1, \quad -1\right ) $  \end{solution} \part[1]  $ (1+i)i\cdot i \cdot i $  \begin{solution}  $ \left ( 1, \quad -1\right ) $  \end{solution} \part[1]  $ i $  \begin{solution}  $ \left ( 0, \quad 1\right ) $  \end{solution} \part[1]  $ \frac{1}{2}(-1+\sqrt{3}i) \cdot i $  \begin{solution}  $ \left ( - 0.5 \sqrt{3}, \quad -0.5\right ) $  \end{solution} \part[1]  $ \frac{1}{2}(-1+\sqrt{3}i) \cdot \frac{1}{2}(-1+\sqrt{3}i) \cdot i $  \begin{solution}  $ \left ( 0.5 \sqrt{3}, \quad -0.5\right ) $  \end{solution}
        \end{parts}
        \end{multicols}
        \question p045e11 - Dado el siguiente número $z$, calcula el valor de $z\cdot\overline{z}$
        \begin{multicols}{2} 
        \begin{parts} \part[1]  $ \sqrt{3}-2i $  \begin{solution}  $ 7 $  \end{solution} \part[1]  $ 4-2i $  \begin{solution}  $ - \frac{i}{2} $  \end{solution}
        \end{parts}
        \end{multicols}
        \question p045e17 - Calcula
        \begin{multicols}{2} 
        \begin{parts} \part[1]  $ (5-i)(3+2i) $  \begin{solution}  $ 17 + 7 i $  \end{solution} \part[1]  $ (2+\frac{1}{3}i)(-5-i) $  \begin{solution}  $ - \frac{29}{3} - \frac{11 i}{3} $  \end{solution} \part[1]  $ (2-i)(2+i) $  \begin{solution}  $ 5 $  \end{solution} \part[1]  $ (3-\frac{1}{4}i)(2-i)(3+2i) $  \begin{solution}  $ \frac{97}{4} + i $  \end{solution} \part[1]  $ \frac{2-i}{1+3i} $  \begin{solution}  $ - \frac{1}{10} - \frac{7 i}{10} $  \end{solution} \part[1]  $ \frac{{\sqrt{2} - 3i}}{{2 + i}} $  \begin{solution}  $ - \frac{3}{5} + \frac{2 \sqrt{2}}{5} - \frac{6 i}{5} - \frac{\sqrt{2} i}{5} $  \end{solution} \part[1]  $ \frac{1}{{3 - i}} $  \begin{solution}  $ \frac{3}{10} + \frac{i}{10} $  \end{solution} \part[1]  $ \frac{{3i}}{{2 - 4i}} $  \begin{solution}  $ - \frac{3}{5} + \frac{3 i}{10} $  \end{solution} \part[1]  $ \frac{{5 - i}}{i} $  \begin{solution}  $ -1 - 5 i $  \end{solution} \part[1]  $ \frac{1 + 2i}{3 + 3i} $  \begin{solution}  $ \frac{1}{2} + \frac{i}{6} $  \end{solution} \part[1]  $ ({\sqrt {2}  - i})\frac{{\sqrt{ 2} +i}}{{1 - 2i}} $  \begin{solution}  $ \frac{3}{5} + \frac{6 i}{5} $  \end{solution} \part[1]  $ ( {2\sqrt {3} - i})\frac{{\sqrt {3} i}}{{1 + i}} $  \begin{solution}  $ \frac{\sqrt{3}}{2} + 3 - \frac{\sqrt{3} i}{2} + 3 i $  \end{solution} \part[1]  $ \frac{{1 - i}}{{3 + 2i}}\frac{{2i}}{{1 + i}} $  \begin{solution}  $ \frac{6}{13} - \frac{4 i}{13} $  \end{solution} \part[1]  $ \frac{{\sqrt {2} }}{{ - 2 - i}}\frac{1}{{2 + 3i}} $  \begin{solution}  $ - \frac{\sqrt{2}}{65} + \frac{8 \sqrt{2} i}{65} $  \end{solution}
        \end{parts}
        \end{multicols}
        \question p046e31y 32 - Calcular el módulo y el argumento (en radianes) de los siguientes números complejos:
        \begin{multicols}{2} 
        \begin{parts} \part[1]  $ 2 - 2\sqrt {3}i $  \begin{solution}  $ 4_{- \frac{\pi}{3}} $  \end{solution} \part[1]  $ -1-i $  \begin{solution}  $ \sqrt{2}_{- \frac{3 \pi}{4}} $  \end{solution} \part[1]  $ \sqrt {3}  + i $  \begin{solution}  $ 2_{\frac{\pi}{6}} $  \end{solution} \part[1]  $ 2\sqrt{3}+2i $  \begin{solution}  $ 4_{\frac{\pi}{6}} $  \end{solution} \part[1]  $ 2-2i $  \begin{solution}  $ 2 \sqrt{2}_{- \frac{\pi}{4}} $  \end{solution} \part[1]  $  -5-5i $  \begin{solution}  $ 5 \sqrt{2}_{- \frac{3 \pi}{4}} $  \end{solution} \part[1]  $ 5i $  \begin{solution}  $ 5_{\frac{\pi}{2}} $  \end{solution} \part[1]  $ 4 $  \begin{solution}  $ 4_{0} $  \end{solution} \part[1]  $ 1+i $  \begin{solution}  $ \sqrt{2}_{\frac{\pi}{4}} $  \end{solution} \part[1]  $ -9i $  \begin{solution}  $ 9_{- \frac{\pi}{2}} $  \end{solution} \part[1]  $ -3+3i $  \begin{solution}  $ 3 \sqrt{2}_{\frac{3 \pi}{4}} $  \end{solution} \part[1]  $ \sqrt{3}+i $  \begin{solution}  $ 2_{\frac{\pi}{6}} $  \end{solution}
        \end{parts}
        \end{multicols}
        \question p046e34 - Escribe en forma binómica los siguientes números complejos:
        \begin{multicols}{2} 
        \begin{parts} \part[1]  $ 2_{\frac{\pi}{4}} $  \begin{solution}  $ \sqrt{2} + \sqrt{2} i $  \end{solution} \part[1]  $ 3_{\frac{\pi}{6}} $  \begin{solution}  $ \frac{3 \sqrt{3}}{2} + \frac{3 i}{2} $  \end{solution} \part[1]  $ \sqrt{2}_{\pi} $  \begin{solution}  $ - \sqrt{2} $  \end{solution} \part[1]  $ 17_{0} $  \begin{solution}  $ 17 $  \end{solution} \part[1]  $ 1_{\frac{\pi}{2}} $  \begin{solution}  $ i $  \end{solution} \part[1]  $ 5_{\frac{3 \pi}{2}} $  \begin{solution}  $ - 5 i $  \end{solution} \part[1]  $ 1_{\frac{5 \pi}{6}} $  \begin{solution}  $ - \frac{\sqrt{3}}{2} + \frac{i}{2} $  \end{solution} \part[1]  $ 4_{\frac{2 \pi}{3}} $  \begin{solution}  $ -2 + 2 \sqrt{3} i $  \end{solution}
        \end{parts}
        \end{multicols}
        
    \end{questions}
    \end{document}
    