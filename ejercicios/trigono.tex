
        \documentclass[spanish, 11pt]{exam}

        %These tell TeX which packages to use.
        \usepackage{array,epsfig}
        \usepackage{amsmath, textcomp}
        \usepackage{amsfonts}
        \usepackage{amssymb}
        \usepackage{amsxtra}
        \usepackage{amsthm}
        \usepackage{mathrsfs}
        \usepackage{color}
        \usepackage{multicol, xparse}
        \usepackage{verbatim}


        \usepackage[utf8]{inputenc}
        \usepackage[spanish]{babel}
        \usepackage{eurosym}

        \usepackage{graphicx}
        \graphicspath{{../img/}}



        \printanswers
        \nopointsinmargin
        \pointformat{}

        %Pagination stuff.
        %\setlength{\topmargin}{-.3 in}
        %\setlength{\oddsidemargin}{0in}
        %\setlength{\evensidemargin}{0in}
        %\setlength{\textheight}{9.in}
        %\setlength{\textwidth}{6.5in}
        %\pagestyle{empty}

        \let\multicolmulticols\multicols
        \let\endmulticolmulticols\endmulticols
        \RenewDocumentEnvironment{multicols}{mO{}}
         {%
          \ifnum#1=1
            #2%
          \else % More than 1 column
            \multicolmulticols{#1}[#2]
          \fi
         }
         {%
          \ifnum#1=1
          \else % More than 1 column
            \endmulticolmulticols
          \fi
         }
        \renewcommand{\solutiontitle}{\noindent\textbf{Sol:}\enspace}

        \newcommand{\samedir}{\mathbin{\!/\mkern-5mu/\!}}

        \newcommand{\class}{1º Bachillerato}
        \newcommand{\examdate}{\today}

        \newcommand{\tipo}{A}


        \newcommand{\timelimit}{50 minutos}



        \pagestyle{head}
        \firstpageheader{\includegraphics[width=0.2\columnwidth]{header_left}}{\textbf{Departamento de Matemáticas\linebreak \class}\linebreak \examnum}{\includegraphics[width=0.1\columnwidth]{header_right}}
        \runningheader{\class}{\examnum}{Página \thepage\ of \numpages}
        \runningheadrule

        \newcommand{\examnum}{23 - Trigonometría}
        \begin{document}
        \begin{questions}
        \question p039e01 - Expresa en radianes los siguientes ángulos, dados en grados:
        \begin{multicols}{3} 
        \begin{parts} \part[1] 45º  \begin{solution}  $ \frac{\pi}{4} $  \end{solution} \part[1] 75º  \begin{solution}  $ \frac{5 \pi}{12} $  \end{solution} \part[1] 105º  \begin{solution}  $ \frac{7 \pi}{12} $  \end{solution} \part[1] 230º  \begin{solution}  $ \frac{23 \pi}{18} $  \end{solution}
        \end{parts}
        \end{multicols}
        \question p039e02 - Expresa en grados los siguientes ángulos dados en radianes:
        \begin{multicols}{3} 
        \begin{parts} \part[1]  $ \frac{{3\pi }}{4} $  \begin{solution}  $ 135 $  \end{solution} \part[1]  $ \frac{{{5}\pi }}{{3}} $  \begin{solution}  $ 300 $  \end{solution} \part[1]  $ \frac{{{{3}}\pi }}{{{2}}} $  \begin{solution}  $ 270 $  \end{solution} \part[1]  $ \frac{{{{9}}\pi }}{{{2}}} $  \begin{solution}  $ 810 $  \end{solution} \part[1]  $ \frac{{{{4}}\pi }}{{{3}}} $  \begin{solution}  $ 240 $  \end{solution}
        \end{parts}
        \end{multicols}
        \question p039e05y6 - Demostrar si son verdaderas o falsas las siguientes ecuaciones:
        \begin{multicols}{3} 
        \begin{parts} \part[1]  $ \frac{\tan\alpha+\tan\beta}{\cot\alpha+\cot\beta}={\tan\alpha} \cdot {\tan\beta} $  \begin{solution}  $ \mathrm{True} $  \end{solution}
        \end{parts}
        \end{multicols}
        
    \end{questions}
    \end{document}
    