
        \documentclass[spanish, 11pt]{exam}

        %These tell TeX which packages to use.
        \usepackage{array,epsfig}
        \usepackage{amsmath, textcomp}
        \usepackage{amsfonts}
        \usepackage{amssymb}
        \usepackage{amsxtra}
        \usepackage{amsthm}
        \usepackage{mathrsfs}
        \usepackage{color}
        \usepackage{multicol, xparse}
        \usepackage{verbatim}


        \usepackage[utf8]{inputenc}
        \usepackage[spanish]{babel}
        \usepackage{eurosym}

        \usepackage{graphicx}
        \graphicspath{{../img/}}



        \printanswers
        \nopointsinmargin
        \pointformat{}

        %Pagination stuff.
        %\setlength{\topmargin}{-.3 in}
        %\setlength{\oddsidemargin}{0in}
        %\setlength{\evensidemargin}{0in}
        %\setlength{\textheight}{9.in}
        %\setlength{\textwidth}{6.5in}
        %\pagestyle{empty}

        \let\multicolmulticols\multicols
        \let\endmulticolmulticols\endmulticols
        \RenewDocumentEnvironment{multicols}{mO{}}
         {%
          \ifnum#1=1
            #2%
          \else % More than 1 column
            \multicolmulticols{#1}[#2]
          \fi
         }
         {%
          \ifnum#1=1
          \else % More than 1 column
            \endmulticolmulticols
          \fi
         }
        \renewcommand{\solutiontitle}{\noindent\textbf{Sol:}\enspace}

        \newcommand{\samedir}{\mathbin{\!/\mkern-5mu/\!}}

        \newcommand{\class}{1º Bachillerato}
        \newcommand{\examdate}{\today}

        \newcommand{\tipo}{A}


        \newcommand{\timelimit}{50 minutos}



        \pagestyle{head}
        \firstpageheader{\includegraphics[width=0.2\columnwidth]{header_left}}{\textbf{Departamento de Matemáticas\linebreak \class}\linebreak \examnum}{\includegraphics[width=0.1\columnwidth]{header_right}}
        \runningheader{\class}{\examnum}{Página \thepage\ of \numpages}
        \runningheadrule

        \newcommand{\examnum}{23 - Trigonometría}
        \begin{document}
        \begin{questions}
        \question p039e01 - Expresa en radianes los siguientes ángulos, dados en grados:
        \begin{multicols}{3} 
        \begin{parts} \part[1] 45º  \begin{solution}  $ \frac{\pi}{4} $  \end{solution} \part[1] 75º  \begin{solution}  $ \frac{5 \pi}{12} $  \end{solution} \part[1] 105º  \begin{solution}  $ \frac{7 \pi}{12} $  \end{solution} \part[1] 230º  \begin{solution}  $ \frac{23 \pi}{18} $  \end{solution}
        \end{parts}
        \end{multicols}
        \question p039e02 - Expresa en grados los siguientes ángulos dados en radianes:
        \begin{multicols}{3} 
        \begin{parts} \part[1]  $ \frac{{3\pi }}{4} $  \begin{solution}  $ 135 $  \end{solution} \part[1]  $ \frac{{{5}\pi }}{{3}} $  \begin{solution}  $ 300 $  \end{solution} \part[1]  $ \frac{{{{3}}\pi }}{{{2}}} $  \begin{solution}  $ 270 $  \end{solution} \part[1]  $ \frac{{{{9}}\pi }}{{{2}}} $  \begin{solution}  $ 810 $  \end{solution} \part[1]  $ \frac{{{{4}}\pi }}{{{3}}} $  \begin{solution}  $ 240 $  \end{solution}
        \end{parts}
        \end{multicols}
        \question p039e05y6 - Demostrar si son verdaderas o falsas las siguientes ecuaciones:
        \begin{multicols}{1} 
        \begin{parts} \part[1]  $ \sec^2\alpha+\csc^2\alpha={\sec^2\alpha}\cdot{\csc^2\alpha} $  \begin{solution}  $ \left [ \frac{8}{- \cos{\left (4 \alpha \right )} + 1}, \quad \frac{8}{- \cos{\left (4 \alpha \right )} + 1}\right ] \to \mathrm{True} $  \end{solution} \part[1]  $ \frac{\tan\alpha+\tan\beta}{\cot\alpha+\cot\beta}={\tan\alpha} \cdot {\tan\beta} $  \begin{solution}  $ \left [ \tan{\left (\alpha \right )} \tan{\left (\beta \right )}, \quad \tan{\left (\alpha \right )} \tan{\left (\beta \right )}\right ] \to \mathrm{True} $  \end{solution} \part[1]  $ \frac{{\sin\alpha}\cdot{\cos\alpha}}{{\cos^2\alpha}-{\sin^2\alpha}}=\frac{\tan\alpha}{1-{\tan^2\alpha}} $  \begin{solution}  $ \left [ \frac{\tan{\left (2 \alpha \right )}}{2}, \quad \frac{\tan{\left (2 \alpha \right )}}{2}\right ] \to \mathrm{True} $  \end{solution} \part[1]  $ \cot\alpha-\frac{{\cot^2\alpha}-1}{\cot\alpha}=\tan\alpha $  \begin{solution}  $ \left [ \tan{\left (\alpha \right )}, \quad \tan{\left (\alpha \right )}\right ] \to \mathrm{True} $  \end{solution} \part[1]  $ \frac{\sin\alpha+ \cot \alpha}{\tan\alpha+ \csc \alpha}=\cos\alpha $  \begin{solution}  $ \left [ \cos{\left (\alpha \right )}, \quad \cos{\left (\alpha \right )}\right ] \to \mathrm{True} $  \end{solution} \part[1]  $ \cot^2\alpha-\cos^2\alpha={\cot^2\alpha}\cdot{\cos^2\alpha} $  \begin{solution}  $ \left [ - \cos^{2}{\left (\alpha \right )} + \cot^{2}{\left (\alpha \right )}, \quad \cos^{2}{\left (\alpha \right )} \cot^{2}{\left (\alpha \right )}\right ] \to \mathrm{True} $  \end{solution} \part[1]  $ \sin\alpha\cos\alpha\tan\alpha\cot\alpha\sec\alpha\csc\alpha=1 $  \begin{solution}  $ \left [ 1, \quad 1\right ] \to \mathrm{True} $  \end{solution} \part[1]  $ \frac{1+\tan\alpha}{1-\tan\alpha}=\frac{\cos\alpha+\sin\alpha}{\cos\alpha-\sin\alpha} $  \begin{solution}  $ \left [ \frac{\tan{\left (\alpha \right )} + 1}{- \tan{\left (\alpha \right )} + 1}, \quad \tan{\left (\alpha + \frac{\pi}{4} \right )}\right ] \to \mathrm{True} $  \end{solution} \part[1]  $ \frac{1+\tan^2\alpha}{\cot\alpha}=\frac{\tan\alpha}{\cos^2\alpha} $  \begin{solution}  $ \left [ \frac{\tan{\left (\alpha \right )}}{\cos^{2}{\left (\alpha \right )}}, \quad \frac{\tan{\left (\alpha \right )}}{\cos^{2}{\left (\alpha \right )}}\right ] \to \mathrm{True} $  \end{solution}
        \end{parts}
        \end{multicols}
        \question p039e07 - Simplificar las siguientes expresiones:
        \begin{multicols}{1} 
        \begin{parts} \part[1]  $ {\sin\alpha}\cdot{\frac{1}{\tan\alpha}} $  \begin{solution}  $ \cos{\left (\alpha \right )} $  \end{solution} \part[1]  $ \sin^3\alpha+{\sin\alpha}\cdot{\cos^2\alpha} $  \begin{solution}  $ \sin{\left (\alpha \right )} $  \end{solution} \part[1]  $ \sqrt{(1-\sin\alpha)\cdot(1+\sin\alpha)} $  \begin{solution}  $ \sqrt{\cos^{2}{\left (\alpha \right )}} $  \end{solution} \part[1]  $ \sin^4\alpha-\cos^4\alpha $  \begin{solution}  $ - \cos{\left (2 \alpha \right )} $  \end{solution} \part[1]  $ \cos^3\alpha+{\cos^2\alpha}\cdot{\sin\alpha}+{\cos\alpha}\cdot{\sin^2\alpha}+{\sin^3\alpha} $  \begin{solution}  $ \sqrt{2} \sin{\left (\alpha + \frac{\pi}{4} \right )} $  \end{solution} \part[1]  $ {\sin\alpha}\cdot{\cos\alpha}\cdot(\tan\alpha+\frac{1}{\tan\alpha}) $  \begin{solution}  $ 1 $  \end{solution} \part[1]  $ \frac{\cos^2\alpha-\sin^2\alpha}{\cos^4\alpha-\sin^4\alpha} $  \begin{solution}  $ 1 $  \end{solution} \part[1]  $ \frac{\sec^2\alpha+\cos^2\alpha}{\sec^2\alpha-\cos^2\alpha} $  \begin{solution}  $ \frac{\left(- \cos^{2}{\left (\alpha \right )} + 1\right)^{2} + 2 \cos^{2}{\left (\alpha \right )}}{- \cos^{4}{\left (\alpha \right )} + 1} $  \end{solution} \part[1]  $ \frac{\cos^2\alpha}{1-\sin\alpha} $  \begin{solution}  $ \sin{\left (\alpha \right )} + 1 $  \end{solution} \part[1]  $ \frac{\csc\alpha}{1+\cot^2\alpha} $  \begin{solution}  $ \sin{\left (\alpha \right )} $  \end{solution}
        \end{parts}
        \end{multicols}
        \question p039e08 - Calcular las restantes razones trigonométricas de $\alpha$, conocida:
        \begin{multicols}{1} 
        \begin{parts} \part[1]  $ \cos\alpha=\frac{4}{5} \land \alpha \in I $  \begin{solution}  $ \left [ 36.86989764584401, \quad \frac{3}{5}, \quad \frac{4}{5}, \quad \frac{3}{4}\right ] $  \end{solution} \part[1]  $ \sin\alpha=\frac{3}{5} \land \alpha \in II $  \begin{solution}  $ \left [ 36.86989764584402, \quad \frac{3}{5}, \quad - \frac{4}{5}, \quad - \frac{3}{4}\right ] $  \end{solution} \part[1]  $ \tan\alpha=-\frac{3}{4} \land \alpha \in II $  \begin{solution}  $ \left [ 36.86989764584402, \quad \frac{3}{5}, \quad - \frac{4}{5}, \quad - \frac{3}{4}\right ] $  \end{solution} \part[1]  $ \sec\alpha=2 \land \alpha \in IV $  \begin{solution}  $ \left [ 60.0, \quad - \frac{\sqrt{3}}{2}, \quad \frac{1}{2}, \quad - \sqrt{3}\right ] $  \end{solution} \part[1]  $ \csc\alpha=-2 \land \alpha \in III $  \begin{solution}  $ \left [ 30.0, \quad - \frac{1}{2}, \quad - \frac{\sqrt{3}}{2}, \quad \frac{\sqrt{3}}{3}\right ] $  \end{solution} \part[1]  $ \cot\alpha=-2 \land \alpha \in IV $  \begin{solution}  $ \left [ 26.56505117707799, \quad - \frac{\sqrt{5}}{5}, \quad \frac{2 \sqrt{5}}{5}, \quad - \frac{1}{2}\right ] $  \end{solution}
        \end{parts}
        \end{multicols}
        \question p039e09 - Expresa las siguientes razones trigonométricas en función de ángulos del primer cuadrante:
        \begin{multicols}{2} 
        \begin{parts} \part[1]  $ \sin(-120) $  \begin{solution}  $ \left [ 60, \quad - \frac{\sqrt{3}}{2}\right ] $  \end{solution} \part[1]  $ \sin(2700) $  \begin{solution}  $ \left [ 0, \quad 0\right ] $  \end{solution} \part[1]  $ \cos(-30) $  \begin{solution}  $ \left [ 30, \quad \frac{\sqrt{3}}{2}\right ] $  \end{solution} \part[1]  $ \cos(3000) $  \begin{solution}  $ \left [ 60, \quad - \frac{1}{2}\right ] $  \end{solution} \part[1]  $ \tan(-275) $  \begin{solution}  $ \left [ \frac{180 \operatorname{atan}{\left (\frac{\cos{\left (\frac{\pi}{18} \right )} + 1}{\cos{\left (\frac{4 \pi}{9} \right )}} \right )}}{\pi}, \quad \frac{\cos{\left (\frac{\pi}{18} \right )} + 1}{\cos{\left (\frac{4 \pi}{9} \right )}}\right ] $  \end{solution} \part[1]  $ \tan(10330) $  \begin{solution}  $ \left [ 70, \quad \tan{\left (\frac{7 \pi}{18} \right )}\right ] $  \end{solution} \part[1]  $ \cot(-150) $  \begin{solution}  $ \left [ 30, \quad \sqrt{3}\right ] $  \end{solution} \part[1]  $ \cot(4500) $  \begin{solution}  $ \left [ 0, \quad \tilde{\infty}\right ] $  \end{solution} \part[1]  $ \sec(-25) $  \begin{solution}  $ \left [ 25, \quad \sec{\left (\frac{5 \pi}{36} \right )}\right ] $  \end{solution} \part[1]  $ \sec(745) $  \begin{solution}  $ \left [ 25, \quad \sec{\left (\frac{149 \pi}{36} \right )}\right ] $  \end{solution} \part[1]  $ \csc(-155) $  \begin{solution}  $ \left [ 25, \quad - \csc{\left (\frac{5 \pi}{36} \right )}\right ] $  \end{solution} \part[1]  $ \csc(4420) $  \begin{solution}  $ \left [ 80, \quad \csc{\left (\frac{4 \pi}{9} \right )}\right ] $  \end{solution}
        \end{parts}
        \end{multicols}
        \question p039e10 - Si sen 37º = 0,6. Calcula, sin usar la calculadora, las razones trigonométricas de los siguientes ángulos dados en grados:
        \begin{multicols}{2} 
        \begin{parts} \part[1]  $ 53 $  \begin{solution}  $ \left [ 0.8, \quad 0.6, \quad 1.33\right ] $  \end{solution} \part[1]  $ 127 $  \begin{solution}  $ \left [ 0.8, \quad -0.6, \quad -1.33\right ] $  \end{solution} \part[1]  $ 143 $  \begin{solution}  $ \left [ 0.6, \quad -0.8, \quad -0.75\right ] $  \end{solution}
        \end{parts}
        \end{multicols}
        \question p041e27 - Resolver las siguientes ecuaciones para ángulos en el primer cuadrante:
        \begin{multicols}{1} 
        \begin{parts} \part[1]  $ \sin{2x}=\frac{1}{2} $  \begin{solution}  $ \left [ \frac{\pi}{12}, \quad \frac{5 \pi}{12}\right ] $  \end{solution} \part[1]  $ \tan{\frac{x}{2}}=\frac{{\sqrt {3} }}{3} $  \begin{solution}  $ \left [ \frac{\pi}{3}\right ] $  \end{solution} \part[1]  $ \sin{(3x-\frac{\pi}{2})}=-\frac{1}{2} $  \begin{solution}  $ \left [ \frac{\pi}{9}, \quad \frac{5 \pi}{9}\right ] $  \end{solution}
        \end{parts}
        \end{multicols}
        \question p041e28 - Resolver las siguientes ecuaciones:
        \begin{multicols}{1} 
        \begin{parts} \part[1]  $ 2\sin{x}+\csc{x}=2\sqrt{2} $  \begin{solution}  $ \left [ 45, \quad 135\right ] $  \end{solution} \part[1]  $ \sin{x}=\cos^2{x}+1 $  \begin{solution}  $ \left [ 90\right ] $  \end{solution} \part[1]  $ \sin{x}\cos{x}=0 $  \begin{solution}  $ \left [ 0, \quad 90, \quad 180, \quad 270\right ] $  \end{solution} \part[1]  $ \tan{x}-\sin{x}=0 $  \begin{solution}  $ \left [ 0, \quad -180, \quad 180, \quad 360\right ] $  \end{solution} \part[1]  $ \sin{x}\cos{x}=2\sin{x} $  \begin{solution}  $ \left [ 0\right ] $  \end{solution} \part[1]  $ 2\cos{x}-3\tan{x}=0 $  \begin{solution}  $ \left [ 150, \quad 30, \quad - \frac{180 i \log{\left (- i \left(- \sqrt{3} + 2\right) \right )}}{\pi}, \quad - \frac{180 i \log{\left (- i \left(\sqrt{3} + 2\right) \right )}}{\pi}\right ] $  \end{solution} \part[1]  $ \sin{2x}=2\cos{x} $  \begin{solution}  $ \left [ -90, \quad 90\right ] $  \end{solution} \part[1]  $ 4\tan{x}=\frac{\sqrt{3}}{\cos^2{x}} $  \begin{solution}  $ \left [ -120, \quad -150, \quad 60, \quad 30\right ] $  \end{solution} \part[1]  $ \sin{x}+\cos{x}=\sqrt{2} $  \begin{solution}  $ \left [ 45\right ] $  \end{solution} \part[1]  $ \sin{2x}\cos{x}=6\sin^3{x} $  \begin{solution}  $ \left [ 0, \quad 180, \quad -150, \quad 150, \quad -30, \quad 30\right ] $  \end{solution} \part[1]  $ 4\sin{\frac{x}{2}}\cos{x}=3 $  \begin{solution}  $ \left [ \right ] $  \end{solution} \part[1]  $ \tan{x}\tan{2x}=1 $  \begin{solution}  $ \left [ -150, \quad 150, \quad -30, \quad 30\right ] $  \end{solution} \part[1]  $ 4\cos{2x}+3\cos{x}=1 $  \begin{solution}  $ \left [ 180, \quad - \frac{180 i \log{\left (\frac{5}{8} - \frac{\sqrt{39} i}{8} \right )}}{\pi}, \quad - \frac{180 i \log{\left (\frac{5}{8} + \frac{\sqrt{39} i}{8} \right )}}{\pi}\right ] $  \end{solution} \part[1]  $ \tan{x}+3\cot{x}=4 $  \begin{solution}  $ \left [ 45, \quad \frac{180 \operatorname{atan}{\left (3 \right )}}{\pi}\right ] $  \end{solution} \part[1]  $ 4\sin{(x-30)}\cos{(x-30)}=\sqrt{3} $  \begin{solution}  $ \left [ \frac{180 \left(- \frac{2 \pi}{3} + 30\right)}{\pi}, \quad \frac{180 \left(\frac{\pi}{6} + 30\right)}{\pi}, \quad \frac{180 \left(\frac{\pi}{3} + 30\right)}{\pi}, \quad \frac{180 \left(- 2 \operatorname{atan}{\left (\sqrt{3} + 2 \right )} + 30\right)}{\pi}\right ] $  \end{solution}
        \end{parts}
        \end{multicols}
        \question p042e01 - Calcular los restantes elementos de un triángulo del que se conocen:
        \begin{multicols}{1} 
        \begin{parts} \part[1] El lado a = 6, y los ángulos B=45º, C=105º  \begin{solution}  $ 6 \sqrt{2},\ 3 \sqrt{2} + 3 \sqrt{6},\ 30º $  \end{solution} \part[1] El lado a = 8, y los ángulos B=30º, C=60º  \begin{solution}  $ 4,\ 4 \sqrt{3},\ 90º $  \end{solution}
        \end{parts}
        \end{multicols}
        \question p042e02 - Calcular los restantes elementos de un triángulo del que se conocen:
        \begin{multicols}{1} 
        \begin{parts} \part[1] Los lados a=10y b=7, y C=30º  \begin{solution}  $ \sqrt{- 70 \sqrt{3} + 149} = 5.268438428052338,\ 108.36878841450955º,\ 41.63121158549045º $  \end{solution}
        \end{parts}
        \end{multicols}
        \question p042e03 - Determina si se puede construir un triángulo ABC sabiendo que
        \begin{multicols}{1} 
        \begin{parts} \part[1] El lado a = 52 y b = 32 y que B = 40.5º.  \begin{solution}  $ distancia_{a recta}=33.77129851316955  \to \mathrm{False} $  \end{solution} \part[1] El lado a = 50 y b = 32 y que B = 39.5º.  \begin{solution}  $ distancia_{a recta}=31.803911013888197  \to \mathrm{True} $  \end{solution}
        \end{parts}
        \end{multicols}
        \question p042e04 - Calcula los ángulo del triángulo ABC, si se conocen:
        \begin{multicols}{1} 
        \begin{parts} \part[1] Los lados a=22, b=17, y c=15  \begin{solution}  $ \left [ 86.62771331656609, \quad 50.47880364135783, \quad 42.89348304207606\right ]º $  \end{solution}
        \end{parts}
        \end{multicols}
        \question p042e07 - Calcular el área de un triángulo sabiendo que:
        \begin{multicols}{1} 
        \begin{parts} \part[1] El lado a=8, y los ángulos B=30º, y C=105º  \begin{solution}  $ 2 \sqrt{2} \left(2 \sqrt{2} + 2 \sqrt{6}\right) \to 21.856406460551018 $  \end{solution}
        \end{parts}
        \end{multicols}
        \question p042e08 - Calcular los lados de un triángulo sabiendo que:
        \begin{multicols}{1} 
        \begin{parts} \part[1] Su área mide=18cm², y los ángulos A=30º, y B=45º  \begin{solution}  $ 6, 6, 6 \sqrt{2} $  \end{solution}
        \end{parts}
        \end{multicols}
        \question p042e09 - Tres puntos A, B y C están unidos por carreteras rectas y llanas. ¿Cuánto distan A y C?, si:
        \begin{multicols}{1} 
        \begin{parts} \part[1] La distancia AB es de 6km, la BC es=9km, y el ángulo que forman AB y BC es de 120º  \begin{solution}  $ 3 \sqrt{19}=13.076696830622021 $  \end{solution}
        \end{parts}
        \end{multicols}
        \question p042e10 - Calcular el área de un triángulo ABC sabiendo que
        \begin{multicols}{1} 
        \begin{parts} \part[1] a=25km, b=28, y sen(C)=0.96, siendo C < 90º  \begin{solution}  $ 336.0 $  \end{solution}
        \end{parts}
        \end{multicols}
        \question p042e11 - Resuelve
        \begin{multicols}{1} 
        \begin{parts} \part[1] Un barco se encuentra a una distancia de 3.5 km del espigón del puerto en el instante en que otro barco se
encuentra a 3 km del primero. Si ambos son observados desde el espigón bajo un ángulo de 43º, ¿a qué
distancia se encuentra el segundo barco del puerto?  \begin{solution}  $ b=\left [ 0.742526244918245, \quad 4.37694966641595\right ] $  \end{solution}
        \end{parts}
        \end{multicols}
        \question p042e12 - Resuelve
        \begin{multicols}{1} 
        \begin{parts} \part[1] Las visuales a lo alto de una torre desde dos puntos A y B del plano horizontal,
    separados 300m entre sí,
forman con el segmento AB ángulos de 50º y 45º, respectivamente. Calcula la distancia desde lo alto de la
torre a los dos puntos A y B.  \begin{solution}  $ A:212.9423442649438\ B:230.69118253280297 $  \end{solution} \part[1] Las visuales a lo alto de una torre desde dos puntos A y B del plano horizontal,
    separados 300m entre sí,
forman con el segmento AB ángulos de 50º y 45º, respectivamente. Calcula la distancia desde lo alto de la
torre a los dos puntos A y B.  \begin{solution}  $ A:\left [ \frac{300 \sin{\left (\frac{5 \pi}{18} \right )}}{\sin{\left (\frac{\pi}{36} \right )}}\right ]\ B:2433.9421324209197 $  \end{solution}
        \end{parts}
        \end{multicols}
        
    \end{questions}
    \end{document}
    