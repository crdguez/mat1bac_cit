
	\documentclass[spanish, 11pt]{exam}
	
	%These tell TeX which packages to use.
	\usepackage{array,epsfig}
	\usepackage{amsmath, textcomp}
	\usepackage{amsfonts}
	\usepackage{amssymb}
	\usepackage{amsxtra}
	\usepackage{amsthm}
	\usepackage{mathrsfs}
	\usepackage{color}
	\usepackage{multicol}
	\usepackage{verbatim}
	
	
	\usepackage[utf8]{inputenc}
	\usepackage[spanish]{babel}
	\usepackage{eurosym}
	
	\usepackage{graphicx}
	\graphicspath{{../img/}}
	
	
	
	\printanswers
	\nopointsinmargin
	\pointformat{}
	
	%Pagination stuff.
	%\setlength{\topmargin}{-.3 in}
	%\setlength{\oddsidemargin}{0in}
	%\setlength{\evensidemargin}{0in}
	%\setlength{\textheight}{9.in}
	%\setlength{\textwidth}{6.5in}
	%\pagestyle{empty}
	
	\renewcommand{\solutiontitle}{\noindent\textbf{Sol:}\enspace}
	
	\newcommand{\samedir}{\mathbin{\!/\mkern-5mu/\!}}
	
	\newcommand{\class}{1º Bachillerato}
	\newcommand{\examdate}{\today}
	
	\newcommand{\tipo}{A}
	
	
	\newcommand{\timelimit}{50 minutos}
	
	
	
	\pagestyle{head}
	\firstpageheader{\includegraphics[width=0.2\columnwidth]{header_left}}{\textbf{Departamento de Matemáticas\linebreak \class}\linebreak \examnum}{\includegraphics[width=0.1\columnwidth]{header_right}}
	\runningheader{\class}{\examnum}{Página \thepage\ of \numpages}
	\runningheadrule
	
	\newcommand{\examnum}{Números combinatorios. Binomio de Newton}
    \begin{document}
    \begin{questions}
    \question p9e2 - Simplifica los cocientes entre factoriales:
        \begin{multicols}{3} 
        \begin{parts} \part[]  $ \frac{{7!}}{{6!}} $  \begin{solution}  $ 7 $  \end{solution} \part[]  $ \frac{{8!}}{{9!}} $  \begin{solution}  $ \frac{1}{9} $  \end{solution} \part[]  $ \frac{{9!}}{{5!\cdot 4!}} $  \begin{solution}  $ 126 $  \end{solution} \part[]  $ \frac{{m!}}{{(m - 1)!}} $  \begin{solution}  $ m $  \end{solution} \part[]  $ \frac{{( {m + 1} )!}}{{( {m - 1} )!}} $  \begin{solution}  $ m \left(m + 1\right) $  \end{solution}
        \end{parts}
        \end{multicols}
        \question p9e3 - Calcula las siguientes operaciones:
        \begin{multicols}{3} 
        \begin{parts} \part[]  $ \binom{252}{250} $  \begin{solution}  $ 31626 $  \end{solution} \part[]  $ \binom{25}{3} + \binom{25}{4} $  \begin{solution}  $ 14950 $  \end{solution} \part[]  $ \binom{9}{6} + \binom{9}{7} + \binom{10}{2} $  \begin{solution}  $ 165 $  \end{solution} \part[]  $ \binom{4}{2} + \binom{4}{3} + \binom{5}{4}+ \binom{6}{5} + \binom{7}{6} + \binom{8}{7} $  \begin{solution}  $ 36 $  \end{solution} \part[]  $ \binom{4}{0} + \binom{4}{1} + \binom{4}{2}+\binom{4}{3} $  \begin{solution}  $ 15 $  \end{solution}
        \end{parts}
        \end{multicols}
        \question p9e5 - Simplifica:
        \begin{multicols}{3} 
        \begin{parts} \part[]  $ \frac{{6!}}{{5!}} + \frac{{8!}}{{6!}} $  \begin{solution}  $ 62 $  \end{solution} \part[]  $ \frac{{n!}}{{(n - 1)!}} + \frac{{(n + 2)!}}{{n!}} $  \begin{solution}  $ n^{2} + 4 n + 2 $  \end{solution} \part[]  $ \frac{\binom{n+3}{n}+\binom{n+2}{n}}{\frac{n+6}{6}} $  \begin{solution}  $ \frac{n \left(n + 1\right) \left(n + 2\right) \left(n + 6\right)}{6 \left(n^{2} + 6\right)} $  \end{solution}
        \end{parts}
        \end{multicols}
        
    \end{questions}
    \end{document}
    