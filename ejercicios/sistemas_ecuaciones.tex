
        \documentclass[spanish, 11pt]{exam}

        %These tell TeX which packages to use.
        \usepackage{array,epsfig}
        \usepackage{amsmath, textcomp}
        \usepackage{amsfonts}
        \usepackage{amssymb}
        \usepackage{amsxtra}
        \usepackage{amsthm}
        \usepackage{mathrsfs}
        \usepackage{color}
        \usepackage{multicol, xparse}
        \usepackage{verbatim}


        \usepackage[utf8]{inputenc}
        \usepackage[spanish]{babel}
        \usepackage{eurosym}

        \usepackage{graphicx}
        \graphicspath{{../img/}}



        \printanswers
        \nopointsinmargin
        \pointformat{}

        %Pagination stuff.
        %\setlength{\topmargin}{-.3 in}
        %\setlength{\oddsidemargin}{0in}
        %\setlength{\evensidemargin}{0in}
        %\setlength{\textheight}{9.in}
        %\setlength{\textwidth}{6.5in}
        %\pagestyle{empty}

        \let\multicolmulticols\multicols
        \let\endmulticolmulticols\endmulticols
        \RenewDocumentEnvironment{multicols}{mO{}}
         {%
          \ifnum#1=1
            #2%
          \else % More than 1 column
            \multicolmulticols{#1}[#2]
          \fi
         }
         {%
          \ifnum#1=1
          \else % More than 1 column
            \endmulticolmulticols
          \fi
         }
        \renewcommand{\solutiontitle}{\noindent\textbf{Sol:}\enspace}

        \newcommand{\samedir}{\mathbin{\!/\mkern-5mu/\!}}

        \newcommand{\class}{1º Bachillerato}
        \newcommand{\examdate}{\today}

        \newcommand{\tipo}{A}


        \newcommand{\timelimit}{50 minutos}



        \pagestyle{head}
        \firstpageheader{\includegraphics[width=0.2\columnwidth]{header_left}}{\textbf{Departamento de Matemáticas\linebreak \class}\linebreak \examnum}{\includegraphics[width=0.1\columnwidth]{header_right}}
        \runningheader{\class}{\examnum}{Página \thepage\ of \numpages}
        \runningheadrule

        \newcommand{\examnum}{7 - Sistemas de ecuaciones}
        \begin{document}
        \begin{questions}
        \question p019e01 - Resuelve los sistemas:
        \begin{multicols}{2} 
        \begin{parts} \part[1]  $ \left\{\begin{matrix}3x - 2y = 1\\ x + 6y = 7\\ \end{matrix}\right. $  \begin{solution}  $ \left[\begin{matrix}3 & -2 & 1\\0 & \frac{20}{3} & \frac{20}{3}\end{matrix}\right] \rightarrow \\ \left \{ x : 1, \quad y : 1\right \} $  \end{solution} \part[1]  $ \left\{\begin{matrix}6x - 2y = 14\\ 3x - y = 7 \\ \end{matrix}\right. $  \begin{solution}  $ \left[\begin{matrix}6 & -2 & 14\\0 & 0 & 0\end{matrix}\right] \rightarrow \\ \left \{ x : \frac{y}{3} + \frac{7}{3}\right \} $  \end{solution} \part[1]  $ \left\{\begin{matrix}6x - 2y = 9\\ 3x - y = 10\\ \end{matrix}\right. $  \begin{solution}  $ \left[\begin{matrix}6 & -2 & 9\\0 & 0 & \frac{11}{2}\end{matrix}\right] \rightarrow \\ \left [ \right ] $  \end{solution} \part[1]  $ \left\{\begin{matrix}4x + 7y =  - 3\\ 7x + 4y = 36\\ \end{matrix}\right. $  \begin{solution}  $ \left[\begin{matrix}4 & 7 & -3\\0 & - \frac{33}{4} & \frac{165}{4}\end{matrix}\right] \rightarrow \\ \left \{ x : 8, \quad y : -5\right \} $  \end{solution} \part[1]  $ \left\{\begin{matrix} 4x + 16 = 5y\\  5y - 19 = 3x\\ \end{matrix}\right. $  \begin{solution}  $ \left[\begin{matrix}-4 & 5 & 16\\0 & - \frac{5}{4} & -7\end{matrix}\right] \rightarrow \\ \left \{ x : 3, \quad y : \frac{28}{5}\right \} $  \end{solution} \part[1]  $ \left\{\begin{matrix}x - 5 = y + 2\\ 1 + 3x + 2y = x - 4\\ \end{matrix}\right. $  \begin{solution}  $ \left[\begin{matrix}1 & -1 & 7\\0 & -4 & 19\end{matrix}\right] \rightarrow \\ \left \{ x : \frac{9}{4}, \quad y : - \frac{19}{4}\right \} $  \end{solution} \part[1]  $ \left\{\begin{matrix}x - 5 = y + 2\\ 3x - 2y = x - 5\\ \end{matrix}\right. $  \begin{solution}  $ \left[\begin{matrix}1 & -1 & 7\\0 & 0 & 19\end{matrix}\right] \rightarrow \\ \left [ \right ] $  \end{solution} \part[1]  $ \left\{\begin{matrix}x + 3y = 6\\ 6y - 5 = 7 - 2x\\ \end{matrix}\right. $  \begin{solution}  $ \left[\begin{matrix}1 & 3 & 6\\0 & 0 & 0\end{matrix}\right] \rightarrow \\ \left \{ x : - 3 y + 6\right \} $  \end{solution} \part[1]  $ \left\{\begin{matrix}x - y = 8\\  x + y = 24\\ \end{matrix}\right. $  \begin{solution}  $ \left[\begin{matrix}1 & -1 & 8\\0 & 2 & 16\end{matrix}\right] \rightarrow \\ \left \{ x : 16, \quad y : 8\right \} $  \end{solution} \part[1]  $ \left\{\begin{matrix}x + 2y = 11\\ 2x - y = 2\\ \end{matrix}\right. $  \begin{solution}  $ \left[\begin{matrix}1 & 2 & 11\\0 & -5 & -20\end{matrix}\right] \rightarrow \\ \left \{ x : 3, \quad y : 4\right \} $  \end{solution} \part[1]  $ \left\{\begin{matrix}3x - 4y =  - 9\\ 2x + y = 5\\ \end{matrix}\right. $  \begin{solution}  $ \left[\begin{matrix}3 & -4 & -9\\0 & \frac{11}{3} & 11\end{matrix}\right] \rightarrow \\ \left \{ x : 1, \quad y : 3\right \} $  \end{solution} \part[1]  $ \left\{\begin{matrix}10( {x - 2} ) + y = 1\\ x + 3( {x - y} ) = 5\\ \end{matrix}\right. $  \begin{solution}  $ \left[\begin{matrix}10 & 1 & 21\\0 & - \frac{17}{5} & - \frac{17}{5}\end{matrix}\right] \rightarrow \\ \left \{ x : 2, \quad y : 1\right \} $  \end{solution} \part[1]  $ \left\{\begin{matrix}\frac{{x - y}}{2} + \frac{{x - y}}{3} = 5\\ \frac{{x + 7}}{4} + y = 3\\ \end{matrix}\right. $  \begin{solution}  $ \left[\begin{matrix}\frac{5}{6} & - \frac{5}{6} & 5\\0 & \frac{5}{4} & - \frac{1}{4}\end{matrix}\right] \rightarrow \\ \left \{ x : \frac{29}{5}, \quad y : - \frac{1}{5}\right \} $  \end{solution} \part[1]  $ \left\{\begin{matrix}\frac{{3( {y + 2x + 2} )}}{4} = \frac{{4x + y - 1}}{3} \\  \frac{1}{3}( {x + y} ) - \frac{1}{6}( {x - y} ) = \frac{{y - 1}}{6}\\ \end{matrix}\right. $  \begin{solution}  $ \left[\begin{matrix}- \frac{1}{6} & - \frac{5}{12} & \frac{11}{6}\\0 & \frac{1}{12} & - \frac{5}{3}\end{matrix}\right] \rightarrow \\ \left \{ x : 39, \quad y : -20\right \} $  \end{solution} \part[1]  $ \left\{\begin{matrix} x - 2( {x + y} ) = 3y - 2\\  \frac{x}{3} + \frac{y}{2} = 3\\ \end{matrix}\right. $  \begin{solution}  $ \left[\begin{matrix}1 & 5 & 2\\0 & - \frac{7}{6} & \frac{7}{3}\end{matrix}\right] \rightarrow \\ \left \{ x : 12, \quad y : -2\right \} $  \end{solution} \part[1]  $ \left\{\begin{matrix}\frac{{3 - 2y}}{4} - \frac{1}{4} = \frac{{1 - 2x}}{6}\\  \frac{{25}}{8} - 1 = \frac{{x + 3}}{2} - \frac{{3( {1 + y} )}}{8}\\ \end{matrix}\right. $  \begin{solution}  $ \left[\begin{matrix}- \frac{1}{3} & \frac{1}{2} & \frac{1}{3}\\0 & \frac{3}{8} & \frac{3}{2}\end{matrix}\right] \rightarrow \\ \left \{ x : 5, \quad y : 4\right \} $  \end{solution} \part[1]  $ \left\{\begin{matrix}\frac{{4y - 5x}}{6} + \frac{{3x - 2y}}{2} = 1 - \frac{2}{9}( {x + y} )\\ \frac{{4y + x - 8}}{8} - x = \frac{{2( {y - 2x} )}}{3}\\ \end{matrix}\right. $  \begin{solution}  $ \left[\begin{matrix}\frac{8}{9} & - \frac{1}{9} & 1\\0 & \frac{7}{64} & - \frac{31}{64}\end{matrix}\right] \rightarrow \\ \left \{ x : \frac{4}{7}, \quad y : - \frac{31}{7}\right \} $  \end{solution}
        \end{parts}
        \end{multicols}
        \question p020e02 - Resuelve mediante expresiones algebraicas:
        \begin{multicols}{1} 
        \begin{parts} \part[1] Halla dos números sabiendo que al dividir el mayor por el menor, obtenemos 3 de cociente y 4 de resto,
mientras que la razón entre los dos después de aumentarlos en 9 unidades es 2.  \begin{solution}  $ \left\{\begin{matrix}y=3x+4\\ \frac{y+9}{x+9}=2\\ \end{matrix}\right.  \rightarrow  \\\left[\begin{matrix}-3 & 1 & 4\\0 & 0 & 2\end{matrix}\right] \rightarrow  \left \{ x : 5, \quad y : 19\right \} $  \end{solution} \part[1] Un automóvil sale de una población A a 60 km/h. Tres horas después sale a su alcance otro automóvil, que
marcha a 75 km/h. Halla la distancia del punto en que se verifica el encuentro a A, y el tiempo que han
tardado en encontrarse.  \begin{solution}  $ \left\{\begin{matrix}75y=60(y+3)\\ x=75y\\ \end{matrix}\right.  \rightarrow  \\\left[\begin{matrix}1 & -75 & 0\\0 & 15 & 180\end{matrix}\right] \rightarrow  \left \{ x : 900, \quad y : 12\right \} $  \end{solution} \part[1] En un corral hay conejos y gallinas, en total 50 cabezas y 134 patas. 
    ¿Cuántos animales hay de cada clase?  \begin{solution}  $ \left\{\begin{matrix}50=x+y\\ 134=4x+2y\\ \end{matrix}\right.  \rightarrow  \\\left[\begin{matrix}1 & 1 & 50\\0 & -2 & -66\end{matrix}\right] \rightarrow  \left \{ x : 17, \quad y : 33\right \} $  \end{solution} \part[1] Se tienen 140 euros, en 20 billetes, unos de 5 euros y de 10 los restantes. 
    ¿Cuántos billetes hay de cada clase?  \begin{solution}  $ \left\{\begin{matrix}140=5x+10y\\ 20=x+y\\ \end{matrix}\right.  \rightarrow  \\\left[\begin{matrix}5 & 10 & 140\\0 & -1 & -8\end{matrix}\right] \rightarrow  \left \{ x : 12, \quad y : 8\right \} $  \end{solution} \part[1] Un librero vendió 84 libros, unos a 45 euros y otros a 36 y obtuvo de la venta 3.105 euros. ¿Cuántos vendió de
cada clase?  \begin{solution}  $ \left\{\begin{matrix}3105=45x+36y\\ 84=x+y\\ \end{matrix}\right.  \rightarrow  \\\left[\begin{matrix}45 & 36 & 3105\\0 & \frac{1}{5} & 15\end{matrix}\right] \rightarrow  \left \{ x : 9, \quad y : 75\right \} $  \end{solution} \part[1] En una clase los 2/3 del número de alumnas es igual a los 5/7 del número de alumnos. Si el número de
alumnas aumenta en 26, entonces es igual al doble del número de alumnos. ¿Cuántos alumnos y alumnas
tiene la clase?  \begin{solution}  $ \left\{\begin{matrix}\frac{2x}{3}=\frac{5y}{7}\\ x+26=2y\\ \end{matrix}\right.  \rightarrow  \\\left[\begin{matrix}\frac{2}{3} & - \frac{5}{7} & 0\\0 & \frac{13}{14} & 26\end{matrix}\right] \rightarrow  \left \{ x : 30, \quad y : 28\right \} $  \end{solution} \part[1] Un comerciante vendió 18 m de una pieza de tela y 20 m de otra de distinta longitud, y le quedó un sobrante,
entre las dos, que es los 2/5 de la longitud de la segunda pieza. Si hubiera vendido un metro más de cada
pieza, el sobrante hubiera sido 1/3 de la longitud de la segunda pieza. ¿Cuántos metros tenía cada pieza?  \begin{solution}  $ \left\{\begin{matrix}(x-18)+(y-20)=\frac{2y}{5}\\ (x-19)+(y-21)=\frac{y}{3}\\ \end{matrix}\right.  \rightarrow  \\\left[\begin{matrix}-1 & - \frac{3}{5} & -38\\0 & - \frac{1}{15} & -2\end{matrix}\right] \rightarrow  \left \{ x : 20, \quad y : 30\right \} $  \end{solution} \part[1] Un padre para estimular a su hijo a estudiar matemáticas, le dice: Por cada ejercicio que resuelvas bien te
daré 70 céntimos de euro y por cada uno que hagas mal me darás 50 céntimos. Después de hacer 25
ejercicios, el muchacho se encuentra con 5,5 euros. ¿Cuántos ejercicios ha resuelto bien?  \begin{solution}  $ \left\{\begin{matrix}550=70x-50y\\ 25=x+y\\ \end{matrix}\right.  \rightarrow  \\\left[\begin{matrix}70 & -50 & 550\\0 & \frac{12}{7} & \frac{120}{7}\end{matrix}\right] \rightarrow  \left \{ x : 15, \quad y : 10\right \} $  \end{solution} \part[1] El dueño de un estanco necesita hacer una compra de cerillas y mecheros, y dispone para ello de 325 euros.
Si compra al proveedor 2.600 cajas de cerillas y 300 mecheros, éste le devuelve 15 euros. Si compra 1.600 cajas
de cerillas y 400 mecheros le devuelve 5 euros. ¿Cuánto cuesta cada caja de cerillas? ¿y cada mechero?  \begin{solution}  $ \left\{\begin{matrix}310=2600x+300y\\ 320=1600x+400y\\ \end{matrix}\right.  \rightarrow  \\\left[\begin{matrix}2600 & 300 & 310\\0 & \frac{2800}{13} & \frac{1680}{13}\end{matrix}\right] \rightarrow  \left \{ x : \frac{1}{20}, \quad y : \frac{3}{5}\right \} $  \end{solution} \part[1] Un comerciante está programando las rebajas de enero. Si descuenta un 30% sobre el precio de una marca de
camisas, aún gana 3 euros sobre el precio de coste. Si rebaja un poco más, descontando el 40%, entonces
pierde 2 euros en cada camisa de esa marca. ¿Cuál era el precio de coste y el de venta de ese tipo de camisas?  \begin{solution}  $ \left\{\begin{matrix}\frac{70y}{100}-x=3\\ x - \frac{60y}{100}=2\\ \end{matrix}\right.  \rightarrow  \\\left[\begin{matrix}-1 & \frac{7}{10} & 3\\0 & \frac{1}{10} & 5\end{matrix}\right] \rightarrow  \left \{ x : 32, \quad y : 50\right \} $  \end{solution} \part[1] Se desea mezclar vino de 11 euros con otro de 8 euros litro de modo que la mezcla resulte a 9 euros el litro.
¿Cuántos litros de cada clase deben mezclarse para obtener 300 litro de la mezcla?  \begin{solution}  $ \left\{\begin{matrix}\frac{11x+8y}{300}=9\\ x+y=300\\ \end{matrix}\right.  \rightarrow  \\\left[\begin{matrix}\frac{11}{300} & \frac{2}{75} & 9\\0 & \frac{3}{11} & \frac{600}{11}\end{matrix}\right] \rightarrow  \left \{ x : 100, \quad y : 200\right \} $  \end{solution} \part[1] Miguel y Ana tiene un perro. Averigua el peso de cada uno de los tres sabiendo que Miguel y Ana pesan 50
kg juntos, y Ana y su perro 29 kg y, finalmente, Miguel y el perro 35 kg.  \begin{solution}  $ \left\{\begin{matrix}x+y = 50\\ y+z=29\\ x+z=35\\ \end{matrix}\right.  \rightarrow  \\\left[\begin{matrix}1 & 0 & 1 & 29\\0 & 1 & 1 & 50\\0 & 0 & -2 & -44\end{matrix}\right] \rightarrow  \left \{ x : 28, \quad y : 22, \quad z : 7\right \} $  \end{solution} \part[1] Un grupo de personas comen en un restaurante y les cobran 240 euros. Si hubiesen asistido 2 personas menos
y cada una hubiese hecho un gasto de 5 euros más, la cuenta habría sido de 250 euros. Halla el número de
personas y el gasto de cada una de ellas.  \begin{solution}  $ \left\{\begin{matrix}xy=240\\ (x-2)(y+5)=250\\ \end{matrix}\right.  \rightarrow  \\\left[\begin{matrix}y & x & 240\\0 & 0 & 250\end{matrix}\right] \rightarrow  \left [ \left \{ x : -8, \quad y : -30\right \}, \quad \left \{ x : 12, \quad y : 20\right \}\right ] $  \end{solution} \part[1] El área de un triángulo es 78 cm 2 y entre la base y la altura suman 25 cm. 
    Calcula la base y la altura.  \begin{solution}  $ \left\{\begin{matrix}\frac{xy}{2}=78\\ x+y=25\\ \end{matrix}\right.  \rightarrow  \\\left[\begin{matrix}1 & 1 & 25\\0 & \frac{x}{2} - \frac{y}{2} & - \frac{25 y}{2} + 78\end{matrix}\right] \rightarrow  \left [ \left \{ x : 12, \quad y : 13\right \}, \quad \left \{ x : 13, \quad y : 12\right \}\right ] $  \end{solution} \part[1] El área de un rectángulo es de 50 cm 2 y la diagonal mide 10 cm. Halla los lados.  \begin{solution}  $ \left\{\begin{matrix}xy=50\\ x^2+y^2=10^2\\ \end{matrix}\right.  \rightarrow  \\\left[\begin{matrix}y & x & 50\\0 & 0 & 100\end{matrix}\right] \rightarrow  \left [ \left \{ x : \frac{\sqrt{2} \left(-10 - 5 \sqrt{2}\right) \left(- 5 \sqrt{2} + 10\right)}{10}, \quad y : - 5 \sqrt{2}\right \}, \quad \left \{ x : - \frac{\sqrt{2} \left(-10 + 5 \sqrt{2}\right) \left(5 \sqrt{2} + 10\right)}{10}, \quad y : 5 \sqrt{2}\right \}\right ] $  \end{solution}
        \end{parts}
        \end{multicols}
        \question p021e23 - Resuelve los sistemas:
        \begin{multicols}{2} 
        \begin{parts} \part[1]  $ \left\{\begin{matrix}x - 2y + 5z = 13\\ 2x - 5y + z = 19\\ x + 3y - 2z =  - 4\\ \end{matrix}\right. $  \begin{solution}  $ \left[\begin{matrix}5 & 1 & -2 & 13\\0 & \frac{9}{5} & - \frac{23}{5} & \frac{82}{5}\\0 & 0 & \frac{52}{9} & - \frac{104}{9}\end{matrix}\right] \rightarrow  \\ \left \{ x : 4, \quad y : -2, \quad z : 1\right \} $  \end{solution} \part[1]  $ \left\{\begin{matrix}x - y + z = 7\\ x + y - 3z = 1\\ 2x + y - 4z = 5\\ \end{matrix}\right. $  \begin{solution}  $ \left[\begin{matrix}1 & 1 & -1 & 7\\0 & 4 & -2 & 22\\0 & 0 & 0 & 0\end{matrix}\right] \rightarrow  \\ \left \{ x : z + 4, \quad y : 2 z - 3\right \} $  \end{solution} \part[1]  $ \left\{\begin{matrix}x - 2y + z = 13\\ 3x - 4y + 2z = 1\\ 2x - 2y + z = 0\\ \end{matrix}\right. $  \begin{solution}  $ \left[\begin{matrix}1 & 1 & -2 & 13\\0 & 1 & 0 & -25\\0 & 0 & 0 & 12\end{matrix}\right] \rightarrow  \\ \left [ \right ] $  \end{solution} \part[1]  $ \left\{\begin{matrix}x - y + z = 1\\  x + z = 4\\ y - 3z =  - 15\\ \end{matrix}\right. $  \begin{solution}  $ \left[\begin{matrix}1 & 1 & -1 & 1\\0 & 3 & -2 & -12\\0 & 0 & 1 & 3\end{matrix}\right] \rightarrow  \\ \left \{ x : -2, \quad y : 3, \quad z : 6\right \} $  \end{solution} \part[1]  $ \left\{\begin{matrix}2x - y + z = 6\\ x + y - 2z = 1\\ x - 2y + 3z = 0\\ \end{matrix}\right. $  \begin{solution}  $ \left[\begin{matrix}1 & 2 & -1 & 6\\0 & 5 & -1 & 13\\0 & 0 & 0 & -5\end{matrix}\right] \rightarrow  \\ \left [ \right ] $  \end{solution} \part[1]  $ \left\{\begin{matrix}x + 2y - 3z = 9\\ 2x - y = 6\\ 4x + 3y - 6z = 24\\ \end{matrix}\right. $  \begin{solution}  $ \left[\begin{matrix}-3 & 1 & 2 & 9\\0 & 2 & -1 & 6\\0 & 0 & 0 & 0\end{matrix}\right] \rightarrow  \\ \left \{ x : \frac{3 z}{5} + \frac{21}{5}, \quad y : \frac{6 z}{5} + \frac{12}{5}\right \} $  \end{solution} \part[1]  $ \left\{\begin{matrix}4x - 2y = 2\\  6y - 3z = 1\\ 3z - 4x =  - 1\\ \end{matrix}\right. $  \begin{solution}  $ \left[\begin{matrix}-3 & 0 & 6 & 1\\0 & 4 & -2 & 2\\0 & 0 & 4 & 2\end{matrix}\right] \rightarrow  \\ \left \{ x : \frac{3}{4}, \quad y : \frac{1}{2}, \quad z : \frac{2}{3}\right \} $  \end{solution} \part[1]  $ \left\{\begin{matrix}x + 2y = 5\\  2x + y =  - 1\\  - x + 3y = 6\\ \end{matrix}\right. $  \begin{solution}  $ \left[\begin{matrix}1 & 2 & 5\\0 & -3 & -11\\0 & 0 & - \frac{22}{3}\end{matrix}\right] \rightarrow  \\ \left [ \right ] $  \end{solution} \part[1]  $ \left\{\begin{matrix}x - 3y = 1\\ 4y - z = 1\\ 2x - z = 1\\ \end{matrix}\right. $  \begin{solution}  $ \left[\begin{matrix}-1 & 0 & 4 & 1\\0 & 1 & -3 & 1\\0 & 0 & 2 & -2\end{matrix}\right] \rightarrow  \\ \left \{ x : -2, \quad y : -1, \quad z : -5\right \} $  \end{solution} \part[1]  $ \left\{\begin{matrix}x + 2 =  - y\\ - y + 3 = 2x\\ 4x - y = 6\\ \end{matrix}\right. $  \begin{solution}  $ \left[\begin{matrix}-1 & -1 & 2\\0 & -1 & 7\\0 & 0 & -21\end{matrix}\right] \rightarrow  \\ \left [ \right ] $  \end{solution} \part[1]  $ \left\{\begin{matrix}x - 2y + 3z = 2\\ 2x - 3y + z = 1\\ 3x - y + 2z = 9\\ \end{matrix}\right. $  \begin{solution}  $ \left[\begin{matrix}3 & 1 & -2 & 2\\0 & \frac{5}{3} & - \frac{7}{3} & \frac{1}{3}\\0 & 0 & \frac{18}{5} & \frac{36}{5}\end{matrix}\right] \rightarrow  \\ \left \{ x : 3, \quad y : 2, \quad z : 1\right \} $  \end{solution} \part[1]  $ \left\{\begin{matrix} x - 6 + y = 0\\  - 3y + x - 2 = 0\\ 5x - 26 + y = 0 \\ \end{matrix}\right. $  \begin{solution}  $ \left[\begin{matrix}1 & 1 & 6\\0 & -4 & -4\\0 & 0 & 0\end{matrix}\right] \rightarrow  \\ \left \{ x : 5, \quad y : 1\right \} $  \end{solution} \part[1]  $ \left\{\begin{matrix}2x + 2y =  - 2\\ x + 6 = y\\ 3x + 5y = 1\\ \end{matrix}\right. $  \begin{solution}  $ \left[\begin{matrix}2 & 2 & -2\\0 & 2 & 5\\0 & 0 & -1\end{matrix}\right] \rightarrow  \\ \left [ \right ] $  \end{solution} \part[1]  $ \left\{\begin{matrix}x + y + z = 4\\ x - 2y + 3z = 13\\ x + 3y + 4z = 11\\ \end{matrix}\right. $  \begin{solution}  $ \left[\begin{matrix}1 & 1 & 1 & 4\\0 & -2 & -5 & 1\\0 & 0 & \frac{13}{2} & - \frac{13}{2}\end{matrix}\right] \rightarrow  \\ \left \{ x : 2, \quad y : -1, \quad z : 3\right \} $  \end{solution} \part[1]  $ \left\{\begin{matrix}z - 2( {x + y} ) =  - 9\\ 3x - y = 3\\ 3y - z = 9\\ \end{matrix}\right. $  \begin{solution}  $ \left[\begin{matrix}1 & -2 & -2 & -9\\0 & 3 & -1 & 3\\0 & 0 & \frac{1}{3} & 2\end{matrix}\right] \rightarrow  \\ \left \{ x : 3, \quad y : 6, \quad z : 9\right \} $  \end{solution} \part[1]  $ \left\{\begin{matrix}\frac{x}{2} + \frac{y}{3} + z = 7\\ x - \frac{y}{2} + \frac{z}{3} = 11\\  \frac{x}{3} - y - \frac{z}{2} = 5\\ \end{matrix}\right. $  \begin{solution}  $ \left[\begin{matrix}1 & \frac{1}{2} & \frac{1}{3} & 7\\0 & \frac{5}{6} & - \frac{11}{18} & \frac{26}{3}\\0 & 0 & - \frac{73}{180} & \frac{73}{30}\end{matrix}\right] \rightarrow  \\ \left \{ x : 6, \quad y : -6, \quad z : 6\right \} $  \end{solution} \part[1]  $ \left\{\begin{matrix}\frac{x}{2} + \frac{y}{3} + \frac{z}{3} = 9\\ \frac{x}{3} - \frac{y}{9} + \frac{z}{3} = 6\\ \frac{x}{6} + \frac{y}{2} + \frac{z}{2} = 13\\ \end{matrix}\right. $  \begin{solution}  $ \left[\begin{matrix}\frac{1}{3} & \frac{1}{2} & \frac{1}{3} & 9\\0 & - \frac{1}{6} & - \frac{4}{9} & -3\\0 & 0 & \frac{14}{9} & 10\end{matrix}\right] \rightarrow  \\ \left \{ x : \frac{6}{7}, \quad y : \frac{45}{7}, \quad z : \frac{135}{7}\right \} $  \end{solution} \part[1]  $ \left\{\begin{matrix}x - y + z = 5\\ \frac{{x - 1}}{2} + \frac{y}{3} = 1\\ \frac{{2x + y}}{6} - \frac{{3z + y}}{8} = 4\\ \end{matrix}\right. $  \begin{solution}  $ \left[\begin{matrix}1 & 1 & -1 & 5\\0 & \frac{1}{2} & \frac{1}{3} & \frac{3}{2}\\0 & 0 & - \frac{29}{36} & \frac{15}{4}\end{matrix}\right] \rightarrow  \\ \left \{ x : \frac{177}{29}, \quad y : - \frac{135}{29}, \quad z : - \frac{167}{29}\right \} $  \end{solution}
        \end{parts}
        \end{multicols}
        
    \end{questions}
    \end{document}
    