
        \documentclass[spanish, 11pt]{exam}

        %These tell TeX which packages to use.
        \usepackage{array,epsfig}
        \usepackage{amsmath, textcomp}
        \usepackage{amsfonts}
        \usepackage{amssymb}
        \usepackage{amsxtra}
        \usepackage{amsthm}
        \usepackage{mathrsfs}
        \usepackage{color}
        \usepackage{multicol, xparse}
        \usepackage{verbatim}


        \usepackage[utf8]{inputenc}
        \usepackage[spanish]{babel}
        \usepackage{eurosym}

        \usepackage{graphicx}
        \graphicspath{{../img/}}



        \printanswers
        \nopointsinmargin
        \pointformat{}

        %Pagination stuff.
        %\setlength{\topmargin}{-.3 in}
        %\setlength{\oddsidemargin}{0in}
        %\setlength{\evensidemargin}{0in}
        %\setlength{\textheight}{9.in}
        %\setlength{\textwidth}{6.5in}
        %\pagestyle{empty}

        \let\multicolmulticols\multicols
        \let\endmulticolmulticols\endmulticols
        \RenewDocumentEnvironment{multicols}{mO{}}
         {%
          \ifnum#1=1
            #2%
          \else % More than 1 column
            \multicolmulticols{#1}[#2]
          \fi
         }
         {%
          \ifnum#1=1
          \else % More than 1 column
            \endmulticolmulticols
          \fi
         }
        \renewcommand{\solutiontitle}{\noindent\textbf{Sol:}\enspace}

        \newcommand{\samedir}{\mathbin{\!/\mkern-5mu/\!}}

        \newcommand{\class}{1º Bachillerato}
        \newcommand{\examdate}{\today}

        \newcommand{\tipo}{A}


        \newcommand{\timelimit}{50 minutos}



        \pagestyle{head}
        \firstpageheader{\includegraphics[width=0.2\columnwidth]{header_left}}{\textbf{Departamento de Matemáticas\linebreak \class}\linebreak \examnum}{\includegraphics[width=0.1\columnwidth]{header_right}}
        \runningheader{\class}{\examnum}{Página \thepage\ of \numpages}
        \runningheadrule

        \newcommand{\examnum}{25 - Geometría Plana}
        \begin{document}
        \begin{questions}
        \question p050e01 - Hallar las ecuaciones paramétricas, continua, general y explícita de la recta r 
determinada por:
        \begin{multicols}{2} 
        \begin{parts} \part[1]  $ A(-1, 3) \ y \overrightarrow{u}=(2, 5) $  \begin{solution}  $ - 5 x + 2 y - 11 = 0 y Point2D(2*t - 1, 5*t + 3) = \left ( x, \quad y\right ) $  \end{solution}
        \end{parts}
        \end{multicols}
        \question p050e02 - Halla un vector direccional y un vector perpendicular a la recta:
        \begin{multicols}{2} 
        \begin{parts} \part[1]  $ 3x+2y+8=0 $  \begin{solution}  $ (Point2D(1, -3/2), Point2D(3/2, 1)) $  \end{solution} \part[1]  $ \frac{x-1}{3}=\frac{2-y}{1} $  \begin{solution}  $ (Point2D(1, -1/3), Point2D(1/3, 1)) $  \end{solution} \part[1]  $ y=5 $  \begin{solution}  $ (Point2D(1, 0), Point2D(0, 1)) $  \end{solution}
        \end{parts}
        \end{multicols}
        \question p050e04 - Comprobar que es isósceles el triángulo de vértices:
        \begin{multicols}{2} 
        \begin{parts} \part[1]  $ A=(2, 1), \ B=(1, 2),\  y \ C=(3, 3) $  \begin{solution}  $ True $  \end{solution}
        \end{parts}
        \end{multicols}
        \question p050e05 - Determinar m con la condición de que disten 1 unidad los siguiente puntos
        \begin{multicols}{2} 
        \begin{parts} \part[1]  $ A=(0, m)\  y \ B=(1, 2) $  \begin{solution}  $ [2] $  \end{solution} \part[1]  $ r\equiv4x-y-1=0\  y \ s\equiv2x+7y-6=0 $  \begin{solution}  $ 91.90915243299638 $  \end{solution} \part[1]  $ r\equiv-x+2y+1=0\  y \ s\equiv3x+y+5=0 $  \begin{solution}  $ 98.13010235415598 $  \end{solution}
        \end{parts}
        \end{multicols}
        \question p050e07y8 - Hallar la recta r que:
        \begin{multicols}{2} 
        \begin{parts} \part[1]  $ Pasa \ por \ A(1, -2)\  y \ forma \ 120 \ grados \  con \ s\equiv y=0 $  \begin{solution}  $ \frac{\sqrt{3} x}{2} + \frac{y}{2} - \frac{\sqrt{3}}{2} + 1 = 0 $  \end{solution} \part[1]  $ Pasa \ por \ A(2, 3)\  y \ forma \ 45 \ grados \  con \ s\equiv 2x+y-1=0 $  \begin{solution}  $ \frac{\sqrt{2} \left(- x - 3 y + 11\right)}{2} = 0 $  \end{solution}
        \end{parts}
        \end{multicols}
        
    \end{questions}
    \end{document}
    