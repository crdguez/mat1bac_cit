
        \documentclass[spanish, 11pt]{exam}

        %These tell TeX which packages to use.
        \usepackage{array,epsfig}
        \usepackage{amsmath, textcomp}
        \usepackage{amsfonts}
        \usepackage{amssymb}
        \usepackage{amsxtra}
        \usepackage{amsthm}
        \usepackage{mathrsfs}
        \usepackage{color}
        \usepackage{multicol, xparse}
        \usepackage{verbatim}


        \usepackage[utf8]{inputenc}
        \usepackage[spanish]{babel}
        \usepackage{eurosym}

        \usepackage{graphicx}
        \graphicspath{{../img/}}



        \printanswers
        \nopointsinmargin
        \pointformat{}

        %Pagination stuff.
        %\setlength{\topmargin}{-.3 in}
        %\setlength{\oddsidemargin}{0in}
        %\setlength{\evensidemargin}{0in}
        %\setlength{\textheight}{9.in}
        %\setlength{\textwidth}{6.5in}
        %\pagestyle{empty}

        \let\multicolmulticols\multicols
        \let\endmulticolmulticols\endmulticols
        \RenewDocumentEnvironment{multicols}{mO{}}
         {%
          \ifnum#1=1
            #2%
          \else % More than 1 column
            \multicolmulticols{#1}[#2]
          \fi
         }
         {%
          \ifnum#1=1
          \else % More than 1 column
            \endmulticolmulticols
          \fi
         }
        \renewcommand{\solutiontitle}{\noindent\textbf{Sol:}\enspace}

        \newcommand{\samedir}{\mathbin{\!/\mkern-5mu/\!}}

        \newcommand{\class}{1º Bachillerato}
        \newcommand{\examdate}{\today}

        \newcommand{\tipo}{A}


        \newcommand{\timelimit}{50 minutos}



        \pagestyle{head}
        \firstpageheader{\includegraphics[width=0.2\columnwidth]{header_left}}{\textbf{Departamento de Matemáticas\linebreak \class}\linebreak \examnum}{\includegraphics[width=0.1\columnwidth]{header_right}}
        \runningheader{\class}{\examnum}{Página \thepage\ of \numpages}
        \runningheadrule

        \newcommand{\examnum}{25 - Geometría Plana}
        \begin{document}
        \begin{questions}
        \question p050e01 - Hallar las ecuaciones paramétricas, continua, general y explícita de la recta r 
determinada por:
        \begin{multicols}{2} 
        \begin{parts} \part[1]  $ A(-1, 3) \ y \overrightarrow{u}=(2, 5) $  \begin{solution}  $ - 5 x + 2 y - 11 = 0 \ y \ Point2D(2*t - 1, 5*t + 3) = \left ( x, \quad y\right ) $  \end{solution}
        \end{parts}
        \end{multicols}
        \question p050e02 - Halla un vector direccional y un vector perpendicular a la recta:
        \begin{multicols}{2} 
        \begin{parts} \part[1]  $ 3x+2y+8=0 $  \begin{solution}  $ (Point2D(1, -3/2), Point2D(3/2, 1)) $  \end{solution} \part[1]  $ \frac{x-1}{3}=\frac{2-y}{1} $  \begin{solution}  $ (Point2D(1, -1/3), Point2D(1/3, 1)) $  \end{solution} \part[1]  $ y=5 $  \begin{solution}  $ (Point2D(1, 0), Point2D(0, 1)) $  \end{solution}
        \end{parts}
        \end{multicols}
        \question p050e04 - Comprobar que es isósceles el triángulo de vértices:
        \begin{multicols}{2} 
        \begin{parts} \part[1]  $ A=(2, 1), \ B=(1, 2),\  y \ C=(3, 3) $  \begin{solution}  $ True $  \end{solution}
        \end{parts}
        \end{multicols}
        \question p050e05 - Determinar m con la condición de que disten 1 unidad los siguiente puntos
        \begin{multicols}{2} 
        \begin{parts} \part[1]  $ A=(0, m)\  y \ B=(1, 2) $  \begin{solution}  $ [2] $  \end{solution}
        \end{parts}
        \end{multicols}
        \question p050e06 - Determinar el ángulo formado por las rectas:
        \begin{multicols}{2} 
        \begin{parts} \part[1]  $ r\equiv4x-y-1=0\  y \ s\equiv2x+7y-6=0 $  \begin{solution}  $ 91.90915243299638 $  \end{solution} \part[1]  $ r\equiv-x+2y+1=0\  y \ s\equiv3x+y+5=0 $  \begin{solution}  $ 98.13010235415598 $  \end{solution}
        \end{parts}
        \end{multicols}
        \question p050e07,8y28 - Hallar la recta r que:
        \begin{multicols}{2} 
        \begin{parts} \part[1]  $ Pasa \ por \ A(2, 3)\  y \ forma \ 45 \ grados \  con \ s\equiv 2x+y-1=0 $  \begin{solution}  $ \frac{x}{3} + y - \frac{11}{3} = 0 $  \end{solution} \part[1]  $ Pasa \ por \ A(1, 2)\  y \ forma \ 45 \ grados \  con \ s\equiv 2x+y-1=0 $  \begin{solution}  $ \frac{x}{3} + y - \frac{7}{3} = 0 $  \end{solution} \part[1]  $ Pasa \ por \ A(1, -2)\  y \ forma \ 120 \ grados \  con \ s\equiv y=0 $  \begin{solution}  $ \sqrt{3} x + y - \sqrt{3} + 2 = 0 $  \end{solution}
        \end{parts}
        \end{multicols}
        \question p051e20 - Calcula el vértice C de un triángulo isósceles ABC, sabiendo que:
        \begin{multicols}{1} 
        \begin{parts} \part[1]  $  A(4, 0)\  , \ B=(6, 2) \ y \  C \in r\equiv 3x+y-1=0 $  \begin{solution}  $ \left [ \left \{ x : - \frac{5}{2}, \quad y : \frac{17}{2}\right \}\right ] $  \end{solution}
        \end{parts}
        \end{multicols}
        \question p051e21 - Determinar el punto simétrico al punto y respecto de la recta siguientes:
        \begin{multicols}{1} 
        \begin{parts} \part[1]  $  A(2, 5) \ y \   r\equiv 5x+y=2 $  \begin{solution}  $ \left [ - x + 5 y - 23, \quad Point2D(-1/2, 9/2), \quad Point2D(-3, 4)\right ] $  \end{solution}
        \end{parts}
        \end{multicols}
        \question p051e23 - Hallar la ecuación de la recta paralela y que dista una unidad a la recta:
        \begin{multicols}{1} 
        \begin{parts} \part[1]  $ r\equiv 4x-3y=0 $  \begin{solution}  $ \frac{\left|{4 x - 3 y}\right|}{5} - 1 = 0 $  \end{solution}
        \end{parts}
        \end{multicols}
        \question p051e29 - Halla el valor del ángulo que forma con el eje de abscisas la mediatriz del segmento determinado por los puntos:
        \begin{multicols}{1} 
        \begin{parts} \part[1]  $ A=(1, -3)\ y \ B=(4, 5) $  \begin{solution}  $ 159.44395478041653 $  \end{solution}
        \end{parts}
        \end{multicols}
        \question p051e33y58 - Calcula el área del triángulo de vértices:
        \begin{multicols}{1} 
        \begin{parts} \part[1]  $ A=(-1, 1), \ B=(1, 4),\  y \ C=(2, -3) $  \begin{solution}  $ \frac{17}{2} $  \end{solution} \part[1]  $ A=(0, -1), \ B=(2, 0),\  y \ C=(1, 1) $  \begin{solution}  $ \frac{3}{2} $  \end{solution}
        \end{parts}
        \end{multicols}
        \question p051e35 - Hallar las ecuaciones de las alturas y las coordenadas del ortocentro del triángulo de vértices:
        \begin{multicols}{1} 
        \begin{parts} \part[1]  $ A=(1, 0), \ B=(-2, 5),\  y \ C=(-1, -3) $  \begin{solution}  $ \left ( \left [ \frac{19 x}{65} - \frac{152 y}{65} - \frac{19}{65} = 0, \quad \frac{38 x}{13} + \frac{57 y}{13} - \frac{209}{13} = 0, \quad - \frac{57 x}{34} + \frac{95 y}{34} + \frac{114}{17} = 0\right ], \quad Point2D(91/19, 9/19)\right ) $  \end{solution}
        \end{parts}
        \end{multicols}
        \question p051e38 - Hallar la ecuación de la recta paralela a la bisectriz del segundo cuadrante y que pasar por el punto:
        \begin{multicols}{1} 
        \begin{parts} \part[1]  $ A=(3, 5) $  \begin{solution}  $ x + y - 8 = 0 $  \end{solution}
        \end{parts}
        \end{multicols}
        \question p051e45 - Hallar el punto de la bisectriz de los cuadrantes 2 y 4, que equidista de los puntos:
        \begin{multicols}{1} 
        \begin{parts} \part[1]  $ A=(4, -2)\ y \ B=(10, 0) $  \begin{solution}  $ \left [ \left \{ x : 10, \quad y : -10\right \}\right ] $  \end{solution}
        \end{parts}
        \end{multicols}
        \question p052e46 - Hallar la longitud de la altura del triángulo ABC que pasa por C, y su área, si:
        \begin{multicols}{1} 
        \begin{parts} \part[1]  $ A=(2, -1), \ B=(-5, 1),\  y \ C=(0, 3) $  \begin{solution}  $ \left [ \frac{24 \sqrt{53}}{53}, \quad 12\right ] $  \end{solution}
        \end{parts}
        \end{multicols}
        \question p052e47 - Hallar las ecuaciones de las rectas de pendiente finita que:
        \begin{multicols}{1} 
        \begin{parts} \part[1]  $ Pasen \ por \ A=(1, -2), \ y \  disten \ 2 \ de \ B=(3, 1) $  \begin{solution}  $ \left [ - \frac{5 x}{12} + y + \frac{29}{12} = 0\right ] $  \end{solution}
        \end{parts}
        \end{multicols}
        \question p052e57 - Dado el triángulo, hallar la mediana correspondiente al vértice A, la altura correspondiente al vértice B y la mediatriz correspondiente al lado AB. Siendo:
        \begin{multicols}{1} 
        \begin{parts} \part[1]  $ A=(2, -3), \ B=(-2, -2),\  y \ C=(0, 3) $  \begin{solution}  $ \left [ - \frac{7 x}{2} - 3 y - 2 = 0, \quad - \frac{11 \left(x - 3 y - 4\right)}{10} = 0, \quad 4 x - y - \frac{5}{2} = 0\right ] $  \end{solution}
        \end{parts}
        \end{multicols}
        \question p052e58 - Determina el valor de m para que el área del triángulo ABC sea:
        \begin{multicols}{1} 
        \begin{parts} \part[1]  $ 6\ unidades \ cuadradas, \ siendo \ A=(2, 1), \ B=(-3, 5)\  y \ C=(4, m) $  \begin{solution}  $ \left [ -3\right ] $  \end{solution}
        \end{parts}
        \end{multicols}
        
    \end{questions}
    \end{document}
    