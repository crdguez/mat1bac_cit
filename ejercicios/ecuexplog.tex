
        \documentclass[spanish, 11pt]{exam}

        %These tell TeX which packages to use.
        \usepackage{array,epsfig}
        \usepackage{amsmath, textcomp}
        \usepackage{amsfonts}
        \usepackage{amssymb}
        \usepackage{amsxtra}
        \usepackage{amsthm}
        \usepackage{mathrsfs}
        \usepackage{color}
        \usepackage{multicol, xparse}
        \usepackage{verbatim}


        \usepackage[utf8]{inputenc}
        \usepackage[spanish]{babel}
        \usepackage{eurosym}

        \usepackage{graphicx}
        \graphicspath{{../img/}}



        \printanswers
        \nopointsinmargin
        \pointformat{}

        %Pagination stuff.
        %\setlength{\topmargin}{-.3 in}
        %\setlength{\oddsidemargin}{0in}
        %\setlength{\evensidemargin}{0in}
        %\setlength{\textheight}{9.in}
        %\setlength{\textwidth}{6.5in}
        %\pagestyle{empty}

        \let\multicolmulticols\multicols
        \let\endmulticolmulticols\endmulticols
        \RenewDocumentEnvironment{multicols}{mO{}}
         {%
          \ifnum#1=1
            #2%
          \else % More than 1 column
            \multicolmulticols{#1}[#2]
          \fi
         }
         {%
          \ifnum#1=1
          \else % More than 1 column
            \endmulticolmulticols
          \fi
         }
        \renewcommand{\solutiontitle}{\noindent\textbf{Sol:}\enspace}

        \newcommand{\samedir}{\mathbin{\!/\mkern-5mu/\!}}

        \newcommand{\class}{1º Bachillerato}
        \newcommand{\examdate}{\today}

        \newcommand{\tipo}{A}


        \newcommand{\timelimit}{50 minutos}



        \pagestyle{head}
        \firstpageheader{\includegraphics[width=0.2\columnwidth]{header_left}}{\textbf{Departamento de Matemáticas\linebreak \class}\linebreak \examnum}{\includegraphics[width=0.1\columnwidth]{header_right}}
        \runningheader{\class}{\examnum}{Página \thepage\ of \numpages}
        \runningheadrule

        \newcommand{\examnum}{9 - Ecuaciones exponenciales y logarítmicas}
        \begin{document}
        \begin{questions}
        \question p027e04 - Resuelve las siguientes ecuaciones exponenciales:
        \begin{multicols}{2} 
        \begin{parts} \part[1]  $ {10^{3 - x}} = 1 $  \begin{solution}  $ \left [ 3\right ] $  \end{solution} \part[1]  $ {5^{x + 3}} = 125 $  \begin{solution}  $ \left [ 0\right ] $  \end{solution} \part[1]  $ 5^{1-x^2}=\frac{1}{125} $  \begin{solution}  $ \left [ -2.0, \quad 2.0\right ] $  \end{solution} \part[1]  $ {5^{{x^2} - 5x + 6}} = 1 $  \begin{solution}  $ \left [ 2, \quad 3\right ] $  \end{solution} \part[1]  $ {2^{1 - x}} = \frac{1}{8} $  \begin{solution}  $ \left [ 4.0\right ] $  \end{solution} \part[1]  $ 2^{x + 3} = 4^{- x} $  \begin{solution}  $ \left [ -1\right ] $  \end{solution} \part[1]  $ 3+x=-2x $  \begin{solution}  $ \left [ -1\right ] $  \end{solution} \part[1]  $ {9^{x - 1}} = {3^{x + 1}} $  \begin{solution}  $ \left [ 3\right ] $  \end{solution} \part[1]  $ {4^{4x + 3}} = {2^{ - x}} $  \begin{solution}  $ \left [ - \frac{2}{3}\right ] $  \end{solution}
        \end{parts}
        \end{multicols}
        \question p028e05 - Resuelve las siguientes ecuaciones exponenciales:
        \begin{multicols}{2} 
        \begin{parts} \part[1]  $ {3^{x + 1}} + {3^x} + {3^{x - 1}} = 117 $  \begin{solution}  $ \left [ 3\right ] $  \end{solution} \part[1]  $ {3^x} + {3^{x - 1}} + {3^{x - 2}} + {3^{x - 3}} + {3^{x - 4}} = 363 $  \begin{solution}  $ \left [ 5\right ] $  \end{solution} \part[1]  $ {2^{3x}} - \frac{3}{{{2^{3x + 2}}}} + 1 = 0 $  \begin{solution}  $ \left [ - \frac{1}{3}\right ] $  \end{solution} \part[1]  $ {3^{x - 1}} + {3^{2 - x}} = 4 $  \begin{solution}  $ \left [ 1, \quad 2\right ] $  \end{solution} \part[1]  $ {2^{x + 1}} + {4^x} = 80 $  \begin{solution}  $ \left [ 3\right ] $  \end{solution} \part[1]  $ 2^{2x} - 3 \cdot {2^{x + 1}} + 8 = 0 $  \begin{solution}  $ \left [ 1, \quad 2\right ] $  \end{solution} \part[1]  $ {3^{2x - 3}} + 1 = 4 \cdot {3^{x - 2}} $  \begin{solution}  $ \left [ 1, \quad 2\right ] $  \end{solution} \part[1]  $ {2^{2x}} - 10 \cdot {2^x} + 16 = 0 $  \begin{solution}  $ \left [ 1, \quad 3\right ] $  \end{solution} \part[1]  $ {16^x} - {4^x} = 240 $  \begin{solution}  $ \left [ 2\right ] $  \end{solution} \part[1]  $ 9^x - 6 \cdot {3^{x + 1}} + 81 = 0 $  \begin{solution}  $ \left [ 2\right ] $  \end{solution} \part[1]  $ {3^{x + 2}} + {9^{x + 1}} = 810 $  \begin{solution}  $ \left [ 2\right ] $  \end{solution} \part[1]  $ {5^{x - 1}} = 2 + \frac{3}{{{5^{x - 2}}}} $  \begin{solution}  $ \left [ 2\right ] $  \end{solution} \part[1]  $ {3^{x + 1}} + {3^{x - 2}} = \frac{{15}}{{{3^{x - 1}}}} + \frac{{247}}{{{3^{x - 2}}}} $  \begin{solution}  $ \left [ 3\right ] $  \end{solution} \part[1]  $ {4^{2x}} + 16 \cdot {4^{ - 2x}} - 10 = 0 $  \begin{solution}  $ \left [ \frac{\log{\left (\sqrt{2} \right )}}{\log{\left (4 \right )}}, \quad \frac{\log{\left (2 \sqrt{2} \right )}}{\log{\left (4 \right )}}\right ] $  \end{solution}
        \end{parts}
        \end{multicols}
        \question p028e07 - Calcula:
        \begin{multicols}{4} 
        \begin{parts} \part[1]  $ \log 100 $  \begin{solution}  $ 2 $  \end{solution} \part[1]  $ \log_{5}(625) $  \begin{solution}  $ 4 $  \end{solution} \part[1]  $ \log_{2}(32) $  \begin{solution}  $ 5 $  \end{solution} \part[1]  $ \log_{3}(81) $  \begin{solution}  $ 4 $  \end{solution} \part[1]  $ \log_{2}(1024) $  \begin{solution}  $ 10 $  \end{solution} \part[1]  $ \log 1000 $  \begin{solution}  $ 3 $  \end{solution} \part[1]  $ \log 10000 $  \begin{solution}  $ 4 $  \end{solution} \part[1]  $ \log 1000000 $  \begin{solution}  $ 6 $  \end{solution} \part[1]  $ \log (10^6) $  \begin{solution}  $ 6 $  \end{solution} \part[1]  $ \log 0.1 $  \begin{solution}  $ -1 $  \end{solution} \part[1]  $ \log 0.01 $  \begin{solution}  $ -2 $  \end{solution} \part[1]  $ \log 0.001 $  \begin{solution}  $ -3 $  \end{solution} \part[1]  $ \log 0.000001 $  \begin{solution}  $ -6 $  \end{solution} \part[1]  $ \log_{5}(625) $  \begin{solution}  $ 4 $  \end{solution} \part[1]  $ \log_{2}(4) $  \begin{solution}  $ 2 $  \end{solution} \part[1]  $ \log_{2}(64) $  \begin{solution}  $ 6 $  \end{solution}
        \end{parts}
        \end{multicols}
        \question p028e07b - Calcula (continuación):
        \begin{multicols}{4} 
        \begin{parts} \part[1]  $ \log_{2}(\frac{1}{2}) $  \begin{solution}  $ -1 $  \end{solution} \part[1]  $ \log_{2}(\frac{1}{4}) $  \begin{solution}  $ -2 $  \end{solution} \part[1]  $ \log_{2}(\sqrt{2}) $  \begin{solution}  $ \frac{1}{2} $  \end{solution} \part[1]  $ \log_{2}(\sqrt{8}) $  \begin{solution}  $ \frac{3}{2} $  \end{solution} \part[1]  $ \log_{3}(3) $  \begin{solution}  $ 1 $  \end{solution} \part[1]  $ \log_{3}(27) $  \begin{solution}  $ 3 $  \end{solution} \part[1]  $ \log_{3}(27) $  \begin{solution}  $ 3 $  \end{solution} \part[1]  $ \log_{3}(\frac{1}{3}) $  \begin{solution}  $ -1 $  \end{solution} \part[1]  $ \log_{3}(\frac{1}{9}) $  \begin{solution}  $ -2 $  \end{solution} \part[1]  $ \log_{3}{\sqrt[3]{3}} $  \begin{solution}  $ \frac{1}{3} $  \end{solution} \part[1]  $ \log_{\frac{1}{3}}(81) $  \begin{solution}  $ -4 $  \end{solution} \part[1]  $ \log_{0.8}(1) $  \begin{solution}  $ 0 $  \end{solution} \part[1]  $ \log_{0.01}({10}^{-3}) $  \begin{solution}  $ \frac{3}{2} $  \end{solution} \part[1]  $ \log_{\frac{1}{49}}(7) $  \begin{solution}  $ - \frac{1}{2} $  \end{solution} \part[1]  $ \log_{\frac{1}{5}}({\frac{1}{25}})^{\frac{1}{5}} $  \begin{solution}  $ \sqrt[5]{2} $  \end{solution}
        \end{parts}
        \end{multicols}
        
    \end{questions}
    \end{document}
    