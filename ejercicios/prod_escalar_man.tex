
        \documentclass[spanish, 11pt]{exam}

        %These tell TeX which packages to use.
        \usepackage{array,epsfig}
        \usepackage{amsmath, textcomp}
        \usepackage{amsfonts}
        \usepackage{amssymb}
        \usepackage{amsxtra}
        \usepackage{amsthm}
        \usepackage{mathrsfs}
        \usepackage{color}
        \usepackage{multicol, xparse}
        \usepackage{verbatim}


        \usepackage[utf8]{inputenc}
        \usepackage[spanish]{babel}
        \usepackage{eurosym}

        \usepackage{graphicx}
        \graphicspath{{../img/}}



        \printanswers
        \nopointsinmargin
        \pointformat{}

        %Pagination stuff.
        %\setlength{\topmargin}{-.3 in}
        %\setlength{\oddsidemargin}{0in}
        %\setlength{\evensidemargin}{0in}
        %\setlength{\textheight}{9.in}
        %\setlength{\textwidth}{6.5in}
        %\pagestyle{empty}

        \let\multicolmulticols\multicols
        \let\endmulticolmulticols\endmulticols
        \RenewDocumentEnvironment{multicols}{mO{}}
         {%
          \ifnum#1=1
            #2%
          \else % More than 1 column
            \multicolmulticols{#1}[#2]
          \fi
         }
         {%
          \ifnum#1=1
          \else % More than 1 column
            \endmulticolmulticols
          \fi
         }
        \renewcommand{\solutiontitle}{\noindent\textbf{Sol:}\enspace}

        \newcommand{\samedir}{\mathbin{\!/\mkern-5mu/\!}}

        \newcommand{\class}{1º Bachillerato}
        \newcommand{\examdate}{\today}

        \newcommand{\tipo}{A}


        \newcommand{\timelimit}{50 minutos}



        \pagestyle{head}
        \firstpageheader{\includegraphics[width=0.2\columnwidth]{header_left}}{\textbf{Departamento de Matemáticas\linebreak \class}\linebreak \examnum}{\includegraphics[width=0.1\columnwidth]{header_right}}
        \runningheader{\class}{\examnum}{Página \thepage\ of \numpages}
        \runningheadrule

        \newcommand{\examnum}{22 - Producto escalar}
    \begin{document}
    \begin{questions}
    \question Sea $\left\lbrace {\vec i\,,\,\vec j} \right\}$ la base canónica de V2, y los vectores:	$\vec u\, = \, - 2\vec i\, + \,\vec j$, 	$\vec v\, = \,2\vec i\, - 3\,\vec j$, 	$\vec w\, = \,\vec i\, + \,\vec j$, 	$\vec z\, = \, - \vec i\, - \,3\vec j$
	Calcular: $\left\lbrace\overrightarrow{ i},\overrightarrow{ j}\right\rbrace$

    \begin{multicols}{1} 
        \begin{parts} 
        \part[1] Las coordenadas de cada uno de ellos respecto de la base canónica  \begin{solution}  $ \left [ \left ( -2, \quad 1\right ), \quad \left ( 2, \quad -3\right ), \quad \left ( 1, \quad 1\right ), \quad \left ( -1, \quad -3\right )\right ] $ \\
         \end{solution}
        
        \part[1] Las coordenadas de los vectores:$\vec u\, + \,2\vec v$, 	$5\vec u\, - \,\vec w$, 	$ - 3\vec v\, + 4\vec w$,		$\vec w - 2\vec z$
        \begin{solution} $\left [ \left ( 2, \quad -5\right ), \quad \left ( 4, \quad -11\right ), \quad \left ( 13, \quad -2\right ), \quad \left ( 3, \quad 7\right )\right ] $
        \end{solution}
        \end{parts}
    \end{multicols}
    \end{questions}    
    \end{document}
    