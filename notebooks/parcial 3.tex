
        \documentclass[spanish, 11pt]{exam}

        %These tell TeX which packages to use.
        \usepackage{array,epsfig}
        \usepackage{amsmath, textcomp}
        \usepackage{amsfonts}
        \usepackage{amssymb}
        \usepackage{amsxtra}
        \usepackage{amsthm}
        \usepackage{mathrsfs}
        \usepackage{color}
        \usepackage{multicol, xparse}
        \usepackage{verbatim}


        \usepackage[utf8]{inputenc}
        \usepackage[spanish]{babel}
        \usepackage{eurosym}

        \usepackage{graphicx}
        \graphicspath{{../img/}}
        \usepackage{pgf}



        \printanswers
        \nopointsinmargin
        \pointformat{}

        %Pagination stuff.
        %\setlength{\topmargin}{-.3 in}
        %\setlength{\oddsidemargin}{0in}
        %\setlength{\evensidemargin}{0in}
        %\setlength{\textheight}{9.in}
        %\setlength{\textwidth}{6.5in}
        %\pagestyle{empty}

        \let\multicolmulticols\multicols
        \let\endmulticolmulticols\endmulticols
        \RenewDocumentEnvironment{multicols}{mO{}}
         {%
          \ifnum#1=1
            #2%
          \else % More than 1 column
            \multicolmulticols{#1}[#2]
          \fi
         }
         {%
          \ifnum#1=1
          \else % More than 1 column
            \endmulticolmulticols
          \fi
         }
        \renewcommand{\solutiontitle}{\noindent\textbf{Sol:}\enspace}

        \newcommand{\samedir}{\mathbin{\!/\mkern-5mu/\!}}

        \newcommand{\class}{1º Bachillerato}
        \newcommand{\examdate}{\today}

        \newcommand{\tipo}{A}


        \newcommand{\timelimit}{50 minutos}



        \pagestyle{head}
        \firstpageheader{\includegraphics[width=0.2\columnwidth]{header_left}}{\textbf{Departamento de Matemáticas\linebreak \class}\linebreak \examnum}{\includegraphics[width=0.1\columnwidth]{header_right}}
        \runningheader{\class}{\examnum}{Página \thepage\ of \numpages}
        \runningheadrule

        \newcommand{\examnum}{Parcial 3ª Evaluación}
        \begin{document}
        \begin{questions}
        \question ex21e02 - En Utebo, el 50\% de sus habitantes es mayor de 60 años, el 40\% fuma y el 60\% 
fuma o es mayor de 60 años. 
Calcula la probabilidad de los siguientes sucesos:
        \begin{multicols}{2}
        \begin{parts} \part[1] i) Ser mayor de 60 años y fumar \\ ii) No fumar  \begin{solution}  $ \left[ \frac{3}{10}, \  \frac{3}{5}\right] $  \end{solution}
        \end{parts}
        \end{multicols}
        \question ex21e03 - Dos fábricas producen el mismo televisor. La fábrica A produce el 60\% de todos los televisore.
El 1\% de todos los televisores producidos por la fábrica A salen defectuosos, mientras que el 2\% de los
televisores producidos por la fábrica B son defectuosos. Se selecciona un televisor al azar de entre
todos los fabricados:

        \begin{multicols}{2}
        \begin{parts} \part[1] i) calcular la probabilidad de que sea defectuoso
ii) Si sabemos que el televisor es defectuoso, calcula la probabilidad de que haya sido producido por la
planta A.  \begin{solution}  $ \left[ \frac{7}{500}, \  \frac{3}{7}\right] $  \end{solution}
        \end{parts}
        \end{multicols}
        \question ex21e04-0 - Un jugador de baloncesto tiene un porcentaje de acierto en tiros de 2 del 70 \%. Si tira 5
veces:\\ 
    
        \begin{multicols}{1}
        \begin{parts} \part[1] Describe la variable del ejercicio  \begin{solution}   $\left\{ 0 : 0.00243, \  1 : 0.02835, \  2 : 0.1323, \  3 : 0.3087, \  4 : 0.36015, \  5 : 0.16807\right\}$   \end{solution} \part[1] Calcula la probabilidad de que enceste 4  \begin{solution}   $P{\left(X = 4 \right)}=0.36015$   \end{solution} \part[1] Calcula la probabilidad de que enceste al menos 1  \begin{solution}   $P{\left(X \geq 1 \right)}=0.99757$   \end{solution} \part[1] Calcula la probabilidad de que enceste más de 4  \begin{solution}   $P{\left(X > 4 \right)}=0.16807$   \end{solution}
        \end{parts}
        \end{multicols}
        \question ex21e05-0 - La duración media de un televisor es de 10 años, con una desviación típica igual a 2 años. Si la
vida útil del electrodoméstico se distribuye normalmente. Halla la probabilidad de que al comprar
un televisor: 
    
        \begin{multicols}{1}
        \begin{parts} \part[1]  este dure más de 12 años  \begin{solution}   $P{\left(X \geq 12 \right)}=0.158655253931457$   \end{solution} \part[1]  este dure entre 8 y 12 años  \begin{solution}   $P{\left(X \geq 8 \wedge X \leq 12 \right)}=0.682689492137086$   \end{solution} \part[1]  este dure más de 12 años  \begin{solution}   $P{\left(X \geq 12 \right)}=0.158655253931457$   \end{solution} \part[1]  este dure entre 8 y 12 años  \begin{solution}   $P{\left(X \geq 8 \wedge X \leq 12 \right)}=0.682689492137086$   \end{solution}
        \end{parts}
        \end{multicols}
        
    \end{questions}
    \end{document}
    