
	\documentclass[spanish, 11pt]{exam}
	
	%These tell TeX which packages to use.
	\usepackage{array,epsfig}
	\usepackage{amsmath, textcomp}
	\usepackage{amsfonts}
	\usepackage{amssymb}
	\usepackage{amsxtra}
	\usepackage{amsthm}
	\usepackage{mathrsfs}
	\usepackage{color}
	\usepackage{multicol}
	\usepackage{verbatim}
	
	
	\usepackage[utf8]{inputenc}
	\usepackage[spanish]{babel}
	\usepackage{eurosym}
	
	\usepackage{graphicx}
	\graphicspath{{../img/}}
	
	
	
	\printanswers
	\nopointsinmargin
	\pointformat{}
	
	%Pagination stuff.
	%\setlength{\topmargin}{-.3 in}
	%\setlength{\oddsidemargin}{0in}
	%\setlength{\evensidemargin}{0in}
	%\setlength{\textheight}{9.in}
	%\setlength{\textwidth}{6.5in}
	%\pagestyle{empty}
	
	\renewcommand{\solutiontitle}{\noindent\textbf{Sol:}\enspace}
	
	\newcommand{\samedir}{\mathbin{\!/\mkern-5mu/\!}}
	
	\newcommand{\class}{1º Bachillerato}
	\newcommand{\examdate}{\today}
	
	\newcommand{\tipo}{A}
	
	
	\newcommand{\timelimit}{50 minutos}
	
	
	
	\pagestyle{head}
	\firstpageheader{\includegraphics[width=0.2\columnwidth]{header_left}}{\textbf{Departamento de Matemáticas\linebreak \class}\linebreak \examnum}{\includegraphics[width=0.1\columnwidth]{header_right}}
	\runningheader{\class}{\examnum}{Página \thepage\ of \numpages}
	\runningheadrule
	
	\newcommand{\examnum}{Autoevaluación}
    \begin{document}
    \begin{questions}
    \question p6e13 - Calcula los siguientes radicales:
        \begin{multicols}{3} 
        \begin{parts} \part[]  $ \sqrt {1225} $  \begin{solution}  $ 35 $  \end{solution}
        \end{parts}
        \end{multicols}
        \question p6e14 - Resuelve las siguientes ecuaciones:
        \begin{multicols}{3} 
        \begin{parts} \part[]  $ x^2 = -6 $  \begin{solution}  $ x^{2} = -6 $  \end{solution} \part[]  $ x^4 = 81 $  \begin{solution}  $ x^{4} = 81 $  \end{solution}
        \end{parts}
        \end{multicols}
        \question p6e15 - Calcula y expresa el resultado de la forma más simple:
        \begin{multicols}{3} 
        \begin{parts} \part[]  $ \sqrt {27}  \cdot \sqrt {243}  \cdot \sqrt {81} $  \begin{solution}  $ 729 $  \end{solution}
        \end{parts}
        \end{multicols}
        \question p6e17 - Calcula y expresa el resultado como potencia de exponente racional:
        \begin{multicols}{3} 
        \begin{parts} \part[]  $ \sqrt[3]{\sqrt{a} } $  \begin{solution}  $ \sqrt[6]{a} $  \end{solution}
        \end{parts}
        \end{multicols}
        \question p6e18 - Calcula:
        \begin{multicols}{3} 
        \begin{parts} \part[]  $ \frac{1}{4}\sqrt {3125}  - 2\sqrt {20}  - \frac{3}{2}\sqrt {45} $  \begin{solution}  $ - \frac{9 \sqrt{5}}{4} $  \end{solution}
        \end{parts}
        \end{multicols}
        \question p6e25 - Calcula:
        \begin{multicols}{3} 
        \begin{parts} \part[]  $ 5\sqrt {8}  - 2\sqrt {50}  + \sqrt {32}  - \sqrt {2} $  \begin{solution}  $ 3 \sqrt{2} $  \end{solution}
        \end{parts}
        \end{multicols}
        \question p6e27 - Efectúa:
        \begin{multicols}{3} 
        \begin{parts} \part[]  $ \sqrt [3] {{x^2}}  \cdot \frac{{\sqrt [5] {xy} }}{{\sqrt {x{y^3}} }} $  \begin{solution}  $ \frac{\sqrt[5]{x y} \sqrt{x y^{3}} \sqrt[3]{x^{2}}}{x y^{3}} $  \end{solution}
        \end{parts}
        \end{multicols}
        \question p6e28 - Racionaliza:
        \begin{multicols}{3} 
        \begin{parts} \part[]  $ \frac{{\sqrt {2}  - \sqrt {3} }}{{\sqrt{ 2}  + \sqrt {3} }} $  \begin{solution}  $ - \left(- \sqrt{3} + \sqrt{2}\right)^{2} $  \end{solution}
        \end{parts}
        \end{multicols}
        \question p9e2 - Simplifica los cocientes entre factoriales:
        \begin{multicols}{3} 
        \begin{parts} \part[]  $ \frac{{7!}}{{6!}} $  \begin{solution}  $ 7 $  \end{solution}
        \end{parts}
        \end{multicols}
        \question p9e3 - Calcula las siguientes operaciones:
        \begin{multicols}{3} 
        \begin{parts} \part[]  $ \binom{252}{250} $  \begin{solution}  $ 31626 $  \end{solution} \part[]  $ \left|{2 x + 3}\right| - 4< 0 $  \begin{solution}  $ - \frac{7}{2} < x \wedge x < \frac{1}{2} $  \end{solution}
        \end{parts}
        \end{multicols}
        \question p9e5 - Simplifica:
        \begin{multicols}{3} 
        \begin{parts} \part[]  $ \frac{{6!}}{{5!}} + \frac{{8!}}{{6!}} $  \begin{solution}  $ 62 $  \end{solution}
        \end{parts}
        \end{multicols}
        \question p9e6 - Realiza los desarrollos de los siguientes binomios:
        \begin{multicols}{3} 
        \begin{parts} \part[]  $ ( {5\sqrt {2} - 2\sqrt {3} } )^4 $  \begin{solution}  $ - 2480 \sqrt{6} + 6244 $  \end{solution}
        \end{parts}
        \end{multicols}
        \question p9e7-13 - Realiza los desarrollos de los siguientes binomios para identificar determinados términos y coeficientes:
        \begin{multicols}{3} 
        \begin{parts} \part[]  $ ( {{x^2} + \frac{1}{x}} )^8 $  \begin{solution}  $ x^{16} + 8 x^{13} + 28 x^{10} + 56 x^{7} + 70 x^{4} + 56 x + \frac{28}{x^{2}} + \frac{8}{x^{5}} + \frac{1}{x^{8}} $  \end{solution}
        \end{parts}
        \end{multicols}
        
    \end{questions}
    \end{document}
    