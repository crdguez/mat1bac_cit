
	\documentclass[spanish, 11pt]{exam}
	
	%These tell TeX which packages to use.
	\usepackage{array,epsfig}
	\usepackage{amsmath, textcomp}
	\usepackage{amsfonts}
	\usepackage{amssymb}
	\usepackage{amsxtra}
	\usepackage{amsthm}
	\usepackage{mathrsfs}
	\usepackage{color}
	\usepackage{multicol}
	\usepackage{verbatim}
	
	
	\usepackage[utf8]{inputenc}
	\usepackage[spanish]{babel}
	\usepackage{eurosym}
	
	\usepackage{graphicx}
	\graphicspath{{../img/}}
	
	
	
	\printanswers
	\nopointsinmargin
	\pointformat{}
	
	%Pagination stuff.
	%\setlength{\topmargin}{-.3 in}
	%\setlength{\oddsidemargin}{0in}
	%\setlength{\evensidemargin}{0in}
	%\setlength{\textheight}{9.in}
	%\setlength{\textwidth}{6.5in}
	%\pagestyle{empty}
	
	\renewcommand{\solutiontitle}{\noindent\textbf{Sol:}\enspace}
	
	\newcommand{\samedir}{\mathbin{\!/\mkern-5mu/\!}}
	
	\newcommand{\class}{1º Bachillerato}
	\newcommand{\examdate}{\today}
	
	\newcommand{\tipo}{A}
	
	
	\newcommand{\timelimit}{50 minutos}
	
	
	
	\pagestyle{head}
	\firstpageheader{\includegraphics[width=0.2\columnwidth]{header_left}}{\textbf{Departamento de Matemáticas\linebreak \class}\linebreak \examnum}{\includegraphics[width=0.1\columnwidth]{header_right}}
	\runningheader{\class}{\examnum}{Página \thepage\ of \numpages}
	\runningheadrule
	
	\newcommand{\examnum}{Potencias y radicales}
    \begin{document}
    \begin{questions}
    
        \question Calcula: 
        \begin{multicols}{3} 
        \begin{parts} \part[]  $ \frac{{{3^{ - 2}} \cdot {3^5} \cdot {2^3}}}{{{{( {3 \cdot 2} )}^4}}} $  \begin{solution}  $ \frac{1}{6} $  \end{solution} \part[]  $ {3^{ - 5}} \cdot {( {\frac{1}{3}} )^{ - 2}} \cdot 81 $  \begin{solution}  $ 3 $  \end{solution} \part[]  $ {( {\frac{5}{4}} )^5} \cdot \frac{{{2^6}}}{{{5^2}}} $  \begin{solution}  $ \frac{125}{16} $  \end{solution} \part[]  $ \frac{{{2^{ - 2}} \cdot {{( {{2^2}} )}^3}}}{{{2^{ - 3}}}} $  \begin{solution}  $ 128 $  \end{solution} \part[]  $ \frac{{{5^{ - 3}} \cdot {5^{ - 1}} \cdot {5^2}}}{{{5^0} + {5^6}}} $  \begin{solution}  $ \frac{1}{390650} $  \end{solution} \part[]  $ {( {\frac{2}{3}} )^{ - 2}} \cdot {( {\frac{3}{2}} )^4} $  \begin{solution}  $ \frac{729}{64} $  \end{solution} \part[]  $ \frac{{\sqrt {2}  \cdot {{( {\sqrt {2} } )}^3} \cdot {{( {\sqrt {5} } )}^3}}}{{{{( {5\sqrt{2} } )}^2}}} $  \begin{solution}  $ \frac{2 \sqrt{5}}{5} $  \end{solution} \part[]  $ \frac{{{9^{\frac{1}{2}}} \cdot {3^{ - 1}} \cdot {2^{\frac{3}{2}}}}}{{\sqrt {2} }} $  \begin{solution}  $ 2 $  \end{solution}
        \end{parts}
        \end{multicols}
        
        \question Calcula: 
        \begin{multicols}{3} 
        \begin{parts} \part[]  $ \frac{{{3^{ - 2}} \cdot {3^5} \cdot {2^3}}}{{{{( {3 \cdot 2} )}^4}}} $  \begin{solution}  $ \frac{1}{6} $  \end{solution} \part[]  $ {3^{ - 5}} \cdot {( {\frac{1}{3}} )^{ - 2}} \cdot 81 $  \begin{solution}  $ 3 $  \end{solution} \part[]  $ {( {\frac{5}{4}} )^5} \cdot \frac{{{2^6}}}{{{5^2}}} $  \begin{solution}  $ \frac{125}{16} $  \end{solution} \part[]  $ \frac{{{2^{ - 2}} \cdot {{( {{2^2}} )}^3}}}{{{2^{ - 3}}}} $  \begin{solution}  $ 128 $  \end{solution} \part[]  $ \frac{{{5^{ - 3}} \cdot {5^{ - 1}} \cdot {5^2}}}{{{5^0} + {5^6}}} $  \begin{solution}  $ \frac{1}{390650} $  \end{solution} \part[]  $ {( {\frac{2}{3}} )^{ - 2}} \cdot {( {\frac{3}{2}} )^4} $  \begin{solution}  $ \frac{729}{64} $  \end{solution} \part[]  $ \frac{{\sqrt {2}  \cdot {{( {\sqrt {2} } )}^3} \cdot {{( {\sqrt {5} } )}^3}}}{{{{( {5\sqrt{2} } )}^2}}} $  \begin{solution}  $ \frac{2 \sqrt{5}}{5} $  \end{solution} \part[]  $ \frac{{{9^{\frac{1}{2}}} \cdot {3^{ - 1}} \cdot {2^{\frac{3}{2}}}}}{{\sqrt {2} }} $  \begin{solution}  $ 2 $  \end{solution} \part[]  $ \sqrt {16} $  \begin{solution}  $ 4 $  \end{solution} \part[]  $ \sqrt[4]{ - 16} $  \begin{solution}  $ 2 \sqrt[4]{-1} $  \end{solution} \part[]  $ \sqrt[3]{27} $  \begin{solution}  $ 3 $  \end{solution} \part[]  $ \sqrt[5]{-1} $  \begin{solution}  $ \sqrt[5]{-1} $  \end{solution} \part[]  $ \sqrt {1225} $  \begin{solution}  $ 35 $  \end{solution} \part[]  $ \sqrt[7]{1} $  \begin{solution}  $ 1 $  \end{solution}
        \end{parts}
        \end{multicols}
        
    \end{questions}
    \end{document}
    